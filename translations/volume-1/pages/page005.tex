\chapter*{}

\section*{Introductory Report on the Enterprise of Publishing the Encyclopedia of Mathematical Sciences}

In September of 1894, Felix Klein and Heinrich Weber met with Franz Meyer, then professor at the Mining Academy in Clausthal, on a journey to the Harz Mountains. There, the first plan for the Encyclopedia of Mathematical Sciences was drafted. Franz Meyer developed his idea of composing a dictionary of pure and applied mathematics.

The ending century has, as in many areas of human knowledge, given rise to the desire for a comprehensive presentation of the scientific work accomplished during its course, which should also include the manifold applications to natural science and technology. Exhaustive, of course, in the sense of a complete presentation delving into all details of the widely branched structure, indicating all paths in both historical and methodological directions, such a work could not be planned, given the lack of comprehensive preliminary work, if one did not want to jeopardize its implementation. Thus, it was initially the intention to compile and characterize only the "most necessary", the fundamental "concepts" of our mathematical knowledge in the form of a lexicon.

"It should" — as Franz Meyer explained in a first draft — "provide the explanation of the concept falling under a given keyword in the form in which it first appeared, along with indication of the literary source, as far as possible. While this was mainly intended for newer concepts, the old and even obsolete expressions should nevertheless be mentioned, to preserve them as in a museum. This would be followed by the historical development of the concept"
