\thispagestyle{fancy}

\vspace{0.5cm}

follow up to the most recent times. Almost every concept differentiates and splits over time, takes on different nuances and applications, branches according to the uses made of it, deepens and generalizes. The respective technical term undergoes corresponding changes, additions, and compositions. The most important sections in this concept's career should again be provided with evidence." Thus, the developmental history of each individual concept should, in its part, provide a picture of progressive science.

The plan found full approval from Klein and Weber.

Fresh courage to execute it might strengthen during the wandering through mountain and forest. A great goal had been brought before their eyes, worth investing the effort and enduring the difficulties that the path would present. The enterprise exceeded the power of the individual; it was to become a collective effort of our German mathematicians, to which each would contribute according to their special field of work, and beyond that, where development brought it with it, researchers from abroad were to be recruited as well.

At that time, the Cartel of German Academies had just been formed, determined to implement and promote large scientific enterprises in collaborative work. The task set here appeared genuinely as a task for the Cartel. Through the academies, not only financial support should be offered, but also in scientific terms, the progress of work that would not be completed quickly — at that time, they thought of implementation in six to seven years — should be secured.

The German Mathematical Association, however, should primarily make the enterprise their own through the cooperation of their members. For them, the successfully begun plan of large detailed scientific reports on all current areas of mathematics, which were to be recorded in the annual reports, was complemented by this new comprehensive task, for which preliminary work could be drawn from those, at least in part.*)

\vfill
\leftline{\rule{2in}{0.4pt}}
\vspace{0.2cm}
{\footnotesize *) Already at the first meeting of the German Mathematical Association in Halle, autumn 1891, Felix Müller during the discussion of "literary under-}
