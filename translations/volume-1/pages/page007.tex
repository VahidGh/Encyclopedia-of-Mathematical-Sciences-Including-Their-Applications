\thispagestyle{fancy}

\vspace{0.5cm}

Thus, the importance and need for comprehensive presentation of widely branched knowledge was naturally accompanied by the necessity of uniting their representatives for collaborative work.

\vspace{0.3cm}
\centerline{\textbf{* * *}}
\vspace{0.3cm}

At the \textit{Natural Scientists' Meeting} in \textit{Vienna} in September 1894, the \textit{German Mathematical Association} decided to adopt the plan of composing a dictionary of pure and applied mathematics and commissioned \textit{Franz Meyer} to seek scientific and financial support from the academies and learned societies of \textit{Göttingen}, \textit{Leipzig}, \textit{Munich}, and \textit{Vienna} united in the Cartel.

At the beginning of 1895, the first draft of the book, combined with a preliminary financing plan (which was established with the involvement of \textit{B. G. Teubner} in \textit{Leipzig}) was presented to the academies and received principal approval from \textit{Göttingen}, \textit{Munich}, and \textit{Vienna}, while the \textit{Society of Sciences} in \textit{Leipzig}, due to lack of available funds, found itself compelled to abstain from participation in the enterprise for the time being.

The learned societies commissioned \textit{F. Klein} (\textit{Göttingen}), \textit{W. v. Dyck} (\textit{Munich}), \textit{G. v. Escherich} (\textit{Vienna}) to initiate discussions with the editorial board and with a publisher to be considered, and to draft a detailed plan of the enterprise regarding both its scientific and financial aspects. This academic commission subsequently stood as a permanent institution alongside the editorial board. It strengthened itself right at the beginning through \textit{H. Weber} (\textit{Strasbourg}) as representative of the \textit{German Mathematical Association} and \textit{L. Boltzmann} (\textit{Vienna}) as advisor in scientific matters. Later, \textit{H. v. Seeliger} (\textit{Munich}) and more recently \textit{O. Hölder} (\textit{Leipzig}), who will be mentioned later, joined as well.

In detailed preliminary work, which concerned the organization of the material and its classification into larger comprehensive as well as smaller

\vfill
\leftline{\rule{2in}{0.4pt}}
\vspace{0.2cm}
{
\footnotesize takings suitable for facilitating the study of mathematics" (1st Annual Report of the \textit{G.M.A.}, p. 59), in connection with the presentation of the draft of his (meanwhile published) mathematical vocabulary, pointed to such an alphabetically arranged mathematical encyclopedia.

}
