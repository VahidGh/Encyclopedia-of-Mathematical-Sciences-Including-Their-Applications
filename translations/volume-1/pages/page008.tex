\thispagestyle{fancy}

\vspace{0.5cm}

individual articles, and then the anticipated scope of the entire work, occupied the summer of 1895. The decision about the feasibility of the enterprise, however, came from a conference of the academy delegates with Franz Meyer in September 1896 in Leipzig, in which A. Wangerin, in place of H. Weber, as well as publisher Alfred Ackermann-Teubner participated. Alongside a first draft of a keyword-based content arrangement, the manuscript of Felix Müller's previously mentioned lexicon of mathematical terminology was present — and it became apparent that for the intended purposes of an encyclopedia, an alphabetical arrangement could not be maintained. If one wanted to link the presentation of our current mathematical knowledge to individual concepts and technical terms and their transformation, as was the original plan mentioned at the beginning, the proper selection of keywords to be included, free from unnecessary ballast, around which the entire presentation would have to be grouped, would already present considerable difficulties. Nevertheless, such an arrangement would result in extensive fragmentation of the content, while on the other hand, particularly in presenting research results and methods, repetitions would hardly be avoidable. The lexicon would moreover acquire a completely inhomogeneous character, because alongside coherent developments about individual areas, very short sections, mere explanations, and countless cross-references would have to be inserted. 

Thus in Leipzig, upon Dyck's proposal, the decision was made to abandon the idea of a proper lexicon and replace the artificial system of alphabetical ordering with the natural system of a purely subject-based arrangement and presentation of mathematical fields of knowledge. Even in such an arrangement, the manifold connections between individual disciplines are often enough severed, the mutual interweaving in subject matter or methodological terms can only partially be expressed, and the sequential presentation must completely replace the simultaneity of facts. But it is still possible to follow the main thread of guiding thoughts in the simply laid out presentation and incorporate into it the development of individual areas with their further elaboration.
