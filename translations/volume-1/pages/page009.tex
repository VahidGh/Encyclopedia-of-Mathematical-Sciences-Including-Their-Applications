\thispagestyle{fancy}

\vspace{0.5cm}

Based on this new principle, the arrangement for the volumes dedicated to pure mathematics was first established. For its development, as well as for the preparation of two articles on "Surfaces of Third Order" and "Potential Theory," they succeeded in gaining \textit{Heinrich Burkhardt}, then lecturer at the University of Göttingen, alongside \textit{Franz Meyer}, and persuaded the former to join the editorial board, for it became apparent from the start that the editorial task could not be managed by a single person. Specifically, \textit{Franz Meyer} later took on the editorship of Volume I (Arithmetic and Algebra) and Volume III (Geometry), while \textit{Heinrich Burkhardt} took that of Volume II (Analysis).

It cannot be denied that with the change in the system of presentation, there was also a shift in content or at least a different emphasis of the same. Not the individual concept, but the structure of content in the results and methods of mathematical research forms the principle of grouping. Thus, the following was established as the task of the "\textit{Encyclopedia of Mathematical Sciences}," as the work was called from then on:

\begin{quote}
"The task of the \textit{Encyclopedia} shall be to provide, in a concise form suitable for quick orientation, but with the greatest possible completeness, a comprehensive presentation of the mathematical sciences according to their current content of established results, and at the same time to demonstrate through careful literature references the historical development of mathematical methods since the beginning of the 19th century. It shall not limit itself to so-called pure mathematics, but shall also consider applications to mechanics and physics, astronomy and geodesy, the various branches of technology and other fields, and thereby give an overall picture of the position that mathematics holds within today's culture."
\end{quote}

A further difficulty now lay in measuring the scope of the entire work and in a proper distribution of space across the individual areas. Comparisons with earlier works of similar nature, with analogous ones from other disciplines, offered only slight
