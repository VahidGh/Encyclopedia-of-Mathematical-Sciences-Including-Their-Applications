\thispagestyle{fancy}

\vspace{0.5cm}

guidance. Here, an initial approach could only be established as a desirable limitation, not as a reliable norm, yet such an estimate had to form the basis for measuring the resources to be contributed by the academies as well as for negotiations with the publishing house.

It was agreed to set six large octavo volumes of forty sheets each as the starting point for space allocation. Three volumes were to serve pure mathematics, two applied mathematics, and another was to be dedicated to historical, philosophical, and didactic questions. Each volume was to be provided with its own index. The final volume should also contain a comprehensive overview and, to make the work usable as a reference work, a detailed alphabetically arranged index.

For the entire implementation of the enterprise, the editorial board was to work together with the commission appointed by the academies:

The editorial board was tasked with structuring the material in detail based on the work's arrangement established in joint consultations with the commission; to gain contributors, to reach understanding with them about the distribution of areas and to mediate the mutual reference of reviewers concerning neighboring areas; to ensure a unified character of the various articles; to oversee the printing; to compile the indexes; finally, through the commission, to provide regular reports to the participating academies about the work's progress.

The academic commission was to be responsible for maintaining the special interest of the academies in the work's prosperity and providing vigorous scientific support to the editorial board. In particular, this commission's approval should be required for any changes proving necessary in the work's plan or in the composition of the editorial board, as well as for the selection of contributors.

In spring 1896, the presented plans and proposals of the commission and editorial board received the approval of the aca-
