\thispagestyle{fancy}

\vspace{0.5cm}

demies of Göttingen, Munich, and Vienna, and the contract for publication was concluded with B. G. Teubner publishing house in Leipzig.

And now the work began — under favorable auspices, for right from the start, the editorial board succeeded in securing a large, significant circle of contributors, ready to put their work in service of the common cause, setting aside their special interests. "General Principles" had been issued, which were intended to ensure as much as possible a common basis for the structure of articles and uniform treatment of the material, without overly restricting the scientific freedom and individuality of the individual who bears full responsibility for their presentation.*)

Regarding the arrangement of individual volumes, as it gradually took shape based on these foundations, this will be reported in the special introductions by the editorial board. Here it should only be emphasized how the establishment and gradual completion of the comprehensive arrangement, the mutual alignment of the content of individual essays, and the clarification of their mutual relationships were particularly promoted in the frequent personal conferences between contributors, editors, and commission members. They represent a sacrifice by all participants that must be acknowledged with the utmost gratitude, but also a lasting

\vfill
\leftline{\rule{2in}{0.4pt}}
\vspace{0.2cm}
{\footnotesize *) We believe we should reproduce them at this point with those modifications and additions which they later received, particularly when undertaking the volumes of applied mathematics.

\vspace{0.25cm}
\textbf{General Principles for the Preparation of Articles.}
\vspace{0.25cm}

1. Within each article, the mathematical concepts peculiar to the respective field, their most important properties, the most fundamental theorems, and the investigation methods that have proven fruitful are presented.

2. The execution of proofs of the communicated theorems must be omitted; only where principally important proof methods are concerned can a brief indication of them be given.

3. The parts of the work relating to applications should fulfill a dual purpose: they should, on one hand, orient the mathematician about what questions the applications pose to them, and on the other hand, inform the astronomer, physicist, engineer about what answer mathematics gives to these questions. Accordingly, they limit themselves to the mathematical side of applications; instrumentation, observation techniques, collection of constants, regulations fall outside the framework of the work.

}
