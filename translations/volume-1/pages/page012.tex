\thispagestyle{fancy}

\vspace{0.5cm}

benefit for the entire work and for all who have participated in it. The Natural Scientists' Meetings of the last decade, starting from the Vienna meeting of 1894, where the foundation stone of the work was laid, the national Mathematics Congress in Zurich (1897), as well as other conferences of the academic commission and editorial board, which were almost regularly combined with the annual meetings of the Cartel of German Academies, offered important opportunities for joint consultation about the work's progress and exchange of ideas about its detailed development.

The necessity of personal discussion became particularly apparent when, in 1897, after the most essential steps for the arrangement and implementation of the first three volumes of pure mathematics had been taken, it was time to approach the volumes dedicated to applied mathematics. From the outset, it had become clear that only an expansion of the editorial board could ensure the implementation of the enterprise and likewise, that — if one did not want to delay completion

\vfill
\leftline{\rule{2in}{0.4pt}}
\vspace{0.2cm}
{\footnotesize If the first of these goals is to be achieved, it will be necessary to: briefly indicate the considerations that led to the mathematical formulation of the problem in question; explicitly establish this formulation; indicate the limits within which the occurring constants lie in practical cases; indicate the degree of accuracy up to which the formulation in question is to be considered correct.

If the second goal is also to be achieved, one must not limit oneself to mere references to those places in the first three volumes where the problem in question is treated; one must briefly state the result of the required mathematical operations (equation solving, geometric construction, integration). However, repetition of literature references is not necessary.

4. Strictly chronological arrangement of the material would necessitate many repetitions for which there is no space; but the gradual development of concepts and methods will be explained at appropriate points and documented through precise literature references.

5. The existing historical monographs and bibliographic resources will provide good initial orientation services to the contributors; however, the first principle of all historical criticism requires that the presentation ultimately be based on personal study of the original works.

6. While results from older developmental periods should be included, specific proof of their origin will have to be omitted; otherwise, following principle (5) would delay the completion of the work beyond measure, as the required orienting works are still lacking, especially for the 18th and partly also for the 17th century

}
