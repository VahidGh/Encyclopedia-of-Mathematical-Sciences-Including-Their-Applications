\thispagestyle{fancy}

\vspace{0.5cm}

of the whole into the distant future — it was necessary to tackle the work from all sides. The academic commission hoped to persuade F. Klein to join the editorial board and specifically to take charge of the volume relating to mechanics, and likewise to gain A. Sommerfeld (then private lecturer in Göttingen) for the editing of the mathematical-physical part. Initially, Klein undertook several major journeys (to England, France, Holland, Italy, and Austria), for which the academies had granted the necessary means in a liberal way, to make the necessary preliminary work for the arrangement, development, and collaboration on these volumes. While the participation of non-German authors had already become of essential importance for the character of the reports in the first volumes, here, with the volumes dedicated to applied mathematics, it is particularly important to be able to count on the collaboration of non-German authors according to the development of individual areas.

As much as we want to claim the entire enterprise as German in its foundation and execution, it is from

\vfill
\leftline{\rule{2in}{0.4pt}}
\vspace{0.2cm}
{\footnotesize lacking. Accordingly, the historical presentation will generally begin with the start of the nineteenth century. Insofar as citations to earlier times are given at all, they are to be understood in the sense that no guarantee is provided whether an even earlier source could have been cited.

7. The individual mathematical subjects are not considered as isolated from each other; on the contrary, it is one of the main tasks of the work to bring general awareness to the manifold interweaving and overlapping of the most diverse areas.

8. One-sided emphasis of a particular school standpoint runs counter to the work's purpose. The most desirable would be if everywhere it were possible to integrate the results obtained by different paths into an objective presentation; where this appears unfeasible, at least each of the opposing views should be given a voice.

9. The Encyclopedia is not called upon to decide pending disputes, particularly those about priority.

10. If concepts or theorems belonging to another area are used in one field, reference is simply made to the section treating the latter area (using the signature used in the arrangement), even if it appears at a later point in the Encyclopedia. Moreover, things about which one can doubt whether they belong in an earlier or later section will generally be included at the earlier point.

11. As far as it can be done without compromising principles (7) and (10), the requirements for readers' prior knowledge will be kept such that

}
