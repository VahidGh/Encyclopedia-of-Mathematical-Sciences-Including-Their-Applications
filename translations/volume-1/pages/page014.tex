\thispagestyle{fancy}

\vspace{0.5cm}

of utmost importance, if it is not to represent a one-sided viewpoint, that in the conception and presentation of individual areas, all voices that have contributed to the uniqueness of their development are heard. The permanent holdings of any science are an international good, gained from the collective work of scholars of all times and all countries. But in different directions, with varying emphasis and appreciation of individual areas, with characteristic differences in methods and form of presentation, different nations and different epochs have participated in this work. This must be expressed in the Encyclopedia in the presentation of content according to its historical development as well as in the recruitment of contributors. Indeed, the enterprise today counts, alongside its foundation of German authors, scholars from America, Belgium, England, France, Holland, Italy, from Norway, Austria, Russia, Sweden among its contributors.

In the years 1898 and 1899, particularly through the personal efforts and connections of \textit{F. Klein}, the implementation

\vfill
\leftline{\rule{2in}{0.4pt}}
\vspace{0.2cm}
{\footnotesize 

the work can also be useful to someone who seeks orientation about only a specific area.

12. Bibliographic completeness of \textit{literature references} is neither possible nor even desirable, just as exhaustive enumeration of all main proposed theorems or suggested technical terms.

13. However, all important \textit{technical terms} actually in use should appear and find explanation, so they can later be included in the index. Cases should be noted particularly where the same term or symbol is used by different authors with different meanings, especially those where the meaning of a term has imperceptibly expanded over time. Among obsolete terms, a sparing selection should be made.

14. Wherever necessary for understanding, \textit{figures} will be included \textit{in the text}.

15. The enterprise does not have the means for including extensive \textit{collections of formulas}, or similar \textit{tables of numerical} values of the functions treated — which should not be copied from other works without prior verification anyway. However, information about where such can be found is desirable, if necessary with a warning against uncritical use. — Very small tables can find space, which illustrate the behavior of a function through a few appropriately chosen numerical values; often a graphical representation will serve the same purpose even better.

16. \textit{Citations} to frequently used journals will be given in uniform abbreviated

}
