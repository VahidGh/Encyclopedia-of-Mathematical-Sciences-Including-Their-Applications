\thispagestyle{fancy}

\vspace{0.5cm}

of the volumes dedicated to applied mathematics could be secured and a first arrangement of the same could be drafted. In doing so, it proved necessary to distribute the entire abundant material of applications across three volumes instead of two as planned, of which the fourth would encompass mechanics, the fifth mathematical physics, and the sixth geodesy, geophysics and astronomy, while a seventh volume was reserved for historical, philosophical and didactic questions.

In 1899, Klein definitively took over the editorship of the volume dedicated to mechanics, soon after Sommerfeld took the editorship of the fifth volume, mathematical physics.

An arrangement of the sixth volume could only be approached after multiple preliminary negotiations in 1900. It was undertaken by E. Wiechert in Göttingen for geodesy and geophysics, and by R. Lehmann-Filhés in Berlin for astronomy, both thereby joining the encyclopedia's editorial board. Unfortunately, the latter found himself compelled to step down from the editorial board in 1902, where he had conducted the initial negotiations with the selected contributors in a most commendable way. In his

\vfill
\leftline{\rule{2in}{0.4pt}}
{\footnotesize 
form (according to a specially established scheme); books are cited, where they appear in an article for the first time, with family name and abbreviated first name of the author, main part of the title, place and year, in case of multiple occurrences the later times in shorter form. Where the matter doesn't seem important enough for more detailed information, mere enumerations of names usually have little use for the reader.

17. Generally meaningless ornamental epithets, such as groundbreaking, ingenious, magnificent, classical etc. should be avoided. Instead, it will be indicated in which direction progress lies in each case: whether in finding new results — or in rigorous foundation of previously only conjecturally proposed or insufficiently proven theorems — or in shortening cumbersome developments through the use of new tools — or finally in systematic arrangement of an entire theory.

Specifically for the preparation of volumes for applied mathematics, the following remarks apply:

1. Since the encyclopedia essentially addresses a mathematical audience, it must place emphasis on the mathematical side of theories. This will include, on one hand, the mathematical formulation of the tasks under consideration, and on the other hand, their mathematical implementation. The latter viewpoint, which often recedes in specifically physical and scientific books, will be essentially kept in mind here. On the other hand, however, in contrast to the presentation in the majority of mathematical works, the experimental foundation of individual

}
