\thispagestyle{fancy}

\vspace{0.5cm}

place, K. Schwarzschild, who had just been appointed to the Göttingen Observatory, joined in 1903.

At Easter 1904, Conrad H. Müller, who had been involved in the editorial work for some time, was appointed by the academic commission as co-editor of the fourth volume to support F. Klein; finally in July 1904, Ph. Furtwängler (in Potsdam) was called to edit the first part of Volume VI (Geodesy and Geophysics) in collaboration with E. Wiechert.

Meanwhile, on November 7, 1898, the first issue of the first volume was published, containing H. Schubert's report on the foundations of arithmetic, E. Netto's report on combinatorics, and A. Pringsheim's extensive work on irrational numbers and convergence of infinite processes. In August 1899, the publication of the second volume then began with Pringsheim's foundations of general function theory, followed by A. Voss's essay on differential and

\vfill
\leftline{\rule{2in}{0.4pt}}
\vspace{0.2cm}
{\footnotesize areas should be described, namely to the extent that the reader gains a general judgment about the foundation and accuracy limit of the mathematical theory.

2. The general plan of the encyclopedia corresponds, as emphasized in the writing, to the historical arrangement of material and provision of the main moments of historical development. However, for the present volumes in this regard, it should be noted that the results of applied mathematics become outdated more quickly than those of pure mathematics, and therefore the historical development here does not have the same importance for understanding the current state of theory as there. Nevertheless, historical presentation will generally be desirable in the following volumes as well, insofar as it is compatible with systematics and clarity.

3. In the fields of applied mathematics, the literature is often very scattered and disconnected. The editorial board has therefore made it their concern to establish connections in advance in as many directions as possible, with mathematicians, physicists, technicians, ... astronomers, as well as in different countries; they will be gladly ready to communicate or at least point out otherwise difficult-to-obtain literature to the contributors based on these connections.

4. Finally, it does not seem necessary that each article be prepared by a single author. Rather, smaller contributions that cover only part of the material to be treated in the respective article are sometimes desirable. Such contributions can be printed as an appendix to the comprehensive article or, if the author agrees, be made available to the main reviewer of the area and incorporated by them. The authorship of such contributions will be appropriately expressed in the article's heading.

}
