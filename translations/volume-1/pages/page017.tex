\thispagestyle{fancy}

\vspace{0.5cm}

integral calculus, as well as that of the late G. Brunel on definite integrals. In October 1902, the first issue of the third volume appeared with essays by H. v. Mangoldt and R. v. Lilienthal on differential geometry. The publication of parts dedicated to applied mathematics began in June 1901 with the fourth volume with M. Abraham's presentation of geometric concepts for mechanics of deformable bodies and two treatises by A.E.H. Love on hydrodynamics. At Easter 1903, Volume V (Mathematical Physics) followed, introduced by C. Runge's essays on measurement and measuring and J. Zenneck's on gravitation, to which G. H. Bryant's general foundation of dynamics is joined. Still in the course of this year, the publication of the first issues of both parts of the sixth volume will begin. They will contain, on one hand, essays by C. Reinhertz and P. Pizzetti on geodesy, by S. Finsterwalder on photogrammetry, and on the other hand (in the astronomical part) treatises by E. Anding and F. Cohn on the theory of coordinates.

Not everyone has approved of this activity beginning on all sides of the work, fearing that the completion of individual volumes might be delayed too much. Also, the reader currently receives a not easily overlooked patchwork of individual issues, which libraries are also reluctant to release for use. However, it must be said that a delay in publication due to the broad scope of editorial activity does not occur, because it almost always involves different editors and contributors; on the contrary, the uniform progress of the whole is of essential importance for utilizing the mutual relationships between individual volumes and individual essays. On the other hand, the publishing house has recently accounted for the ease of use of individual issues through special equipment and binding of the issues.

Here is the place to emphasize with special thanks the extremely great accommodation of the publishing house B. G. Teubner. On one hand, the firm has most willingly fulfilled all extensive wishes and requirements of the editors and authors regarding printing, and on the other hand, through its own commitment to the honoraria to be expended, has made it possible

