\thispagestyle{fancy}

\vspace{0.5cm}

to meet the increasingly strong need arising in the course of development to carry out the work in a scope considerably expanded compared to the initial plan.

One may regret that the original approach of presenting a very concise overview of our current mathematical knowledge in six manageable volumes has been abandoned, and may not without concern see how from volume to volume the work extends beyond the boundaries drawn at the beginning. However, the striving for greatest possible completeness in individual sections and the desire to be clear and comprehensible, even at the expense of brevity — a desire that has repeatedly been expressed to us from reader circles — form the immediate reason for the growth in scope. But the essential reason lies perhaps deeper: The work is a \textit{first} according to its task, so it cannot be a complete one in fulfilling it. Only when the vast field it encompasses lies before us in this first version as a whole, when the circle of problems it has to present has been measured once, will one be able to see how much remains to be done for deepening its content, for simplifying and making the presentation more concise, for aligning and interconnecting all individual parts.

\vspace{0.5cm}
\centerline{\textbf{* * *}}
\vspace{0.5cm}

Two circumstances significant for the recognition of the scientific work accomplished so far are still to be mentioned in our historical report:

The first is the publication of a French adaptation of the Encyclopedia, for which the publishers \textit{B. G. Teubner} in Leipzig and \textit{Gauthier-Villars et fils} in Paris received authorization from the academies in 1900. \textit{J. Molk}, Professor at the Faculty of Sciences in Nancy, was entrusted with directing this edition initially for the volumes dedicated to pure mathematics, while for publishing the volumes of applied mathematics he collaborated with \textit{P. Appell}, member of the Institut de France (mechanics), as well as with \textit{A. Potier} (physics), \textit{Ch. Lallemand} (geodesy and geophysics) and \textit{H. Andoyer} (astronomy).

It is not merely a translation, but an adaptation envisioned, in which the leading French scholars have promised their participation
