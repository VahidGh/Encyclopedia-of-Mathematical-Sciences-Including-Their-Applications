\thispagestyle{fancy}

\vspace{0.5cm}

have promised. While fully preserving the character of the German original, this edition shall take into account the usage of French circles and, on the other hand, through joint collaboration of authors and editors, the individual articles shall experience manifold additions, especially regarding literature citations.*)

Thus the German work in its French edition will be made accessible to and appreciated by even wider circles.

We may see a further recognition of the work's implementation so far, which we welcome with particular joy, in the fact that recently the Royal Saxon Society of Sciences in \textit{Leipzig} has also made it possible to participate in the publication of the Encyclopedia. They have delegated \textit{O. Hölder} to the academic commission on their part.

Thus the publication now appears as a joint enterprise of the learned societies united in the Cartel of German Academies in Göttingen, Leipzig, Munich and Vienna, and it also demonstrates the significance of this union for the implementation of tasks that are only possible in united work; at the same time, however, the authority of the academies, which have made the enterprise their own, provides the guarantee that future development, completion of the whole, as well as later revisions are placed in the best hands and secured in their scientific foundation.

\centerline{\textbf{* * *}}

\vfill
\leftline{\rule{2in}{0.4pt}}
\vspace{0.2cm}
{\footnotesize 

*) The prospectus of the French edition characterizes the nature of the adaptation in the following way:

In the French edition, efforts have been made to reproduce the essential features of the articles from the German edition; however, in the adopted mode of presentation, French customs and habits have been extensively taken into account.

This French edition will offer a very particular character through the collaboration of German and French mathematicians. The author of each article in the German edition has, in fact, indicated the modifications they deemed appropriate to introduce in their article and, on the other hand, the French editing of each article has given rise to an exchange of views in which all interested parties have participated; additions due particularly to French collaborators will be placed between two asterisks. The importance of such collaboration, of which the French edition of the Encyclopedia will offer the first example, will escape no one.

}
