\chapter*{}

\vspace{-0.5cm}
\section*{Preface to the First Volume}

The present first volume of the Encyclopedia of Mathematical Sciences encompasses \textit{Arithmetic, Algebra, Number Theory, Probability Calculation} (with applications to adjustment and interpolation, statistics and life insurance), as well as some adjacent disciplines: \textit{Difference Calculation, Numerical Computation, Mathematical Games, and Mathematical Economics}.

These are approximately those parts of pure mathematics that are not specifically of an analytical or geometrical character.

This separation from \textit{Analysis} (Volume II) and \textit{Geometry} (Volume III) could naturally not be entirely rigid; in many cases, an overlap into these two large areas was inevitable, and it was equally impossible to maintain the concept of "pure" mathematics everywhere. This may be elaborated in more detail, following the main features and progression of Volume I.

In Arithmetic (Section A), the irrational and the concept of limits (A 3) simultaneously form the foundation of contemporary analysis; for the convergence and divergence of infinite series, products, fractions, and determinants, analytical and analytic functions therefore had to be drawn upon as evidence. The theory of simple and higher complex quantities (A 4) necessitated an examination of the metric properties of the simplest continuous transformation groups. Set theory (A 5) has become significant as a fundamental classification principle for functions in general, and recently also for the foundations of geometry and analysis situs.

In Algebra (Section B), some of the richest applications of invariant and group theory (B 2, B 3 c, d, B 3 f), especially the theory of finite linear groups, relate to significant geometric configurations; conversely, the doctrine of algebraic formations and transformations is equally

