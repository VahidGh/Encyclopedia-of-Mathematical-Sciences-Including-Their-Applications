\thispagestyle{fancy}

\vspace{0.5cm}

closely linked with the development of algebraic functions, as well as algebraic geometry.

In the section (C) on Number Theory, it is primarily the approximation methods of analytic number theory (C 3) that substantially originated from analysis and, conversely, have promoted it; a main application to geometry appears to be the impossibility of squaring the circle. The article (C 6) on complex multiplication could with equal justification be viewed as an integral component of the theory of elliptical functions.

Finally, reference should be made to the manifold relationships between probability and difference calculus, along with their applications (D, E), to the approximation of certain integrals, the intervention of the theory of general systems and methods of descriptive geometry in numerical computation (F).

Precisely these final sections (D, E, F, G) simultaneously teach the extent to which principles originally drawn from pure mathematics prove their power in solving the most diverse technical problems.

We now turn to the systematic division of sections into individual articles and can be briefer in doing so, as the detailed overall table of contents in any case permits an external orientation about the distribution of material.

The section (A) on \textit{Arithmetic} begins with the elements (A 1, \textit{H. Schubert}), the arithmetic basic operations and their applications to positive and negative, whole and fractional numbers; connected to this naturally is combinatorics (A 2, \textit{E. Netto}), whose most essential offshoot is the theory of determinants.

From the elements, one can proceed within arithmetic in four different directions.

Either one expands (A 3, \textit{A. Pringsheim}) the domain of rational numbers by incorporating the irrational, and simultaneously transfers the arithmetic basic operations to an unlimited number of objects. From this emerges the concept of the limit of a number sequence, and from this again through specification the theory of convergence and divergence of infinite series, products, continued fractions, and determinants. In this context, the theory of finite continued fractions could also be included.

Secondly, one can drop the restriction to the real and expand the domain of arithmetic quantities through the creation of
