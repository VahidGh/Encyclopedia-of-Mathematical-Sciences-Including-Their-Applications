\thispagestyle{fancy}

\vspace{0.5cm}

expand the domain of general and higher complex quantities (A 4, \textit{E. Study}); suitable classification principles enable the organic integration of particularly significant complex quantities, especially quaternions.

Thirdly, the natural number sequence can be continued beyond itself (A 5, \textit{A. Schoenflies}) and one arrives at the various modifications of sets and transfinite numbers.

Or finally, one builds (A 6, \textit{H. Burkhardt}), in connection with combinatorics, on the basis of the permutation process, the doctrine of substitutions of a number of elements. As the most far-reaching type of summarizing substitutions proves to be the "group", initially for a finite, subsequently for an unlimited series of elements or operations.

By borrowing from analysis the concept of a continuously variable quantity, one enters the domain of \textit{Algebra}, as treated in Section B. With the help of the first three or four arithmetic species, the entire or fractional rational functions of one and several variables arise (B 1 a, b, \textit{E. Netto}). The investigations belonging here group themselves around two main problems, first the formal elimination of unknowns from equation systems, which reaches a certain conclusion in the theory of modular systems, then the proof of existence for solutions of algebraic equations and equation systems.

The theory of entire functions experiences a sharper formulation on the basis of the concept of the "rationality domain", whereby the coefficients of the functions are themselves conceived as entire functions of a number of original variables, but with only integer coefficients (B 1 c, \textit{G. Landsberg}). Thereby it succeeds in subordinating the properties of algebraic formations, especially with respect to rational transformations, to the various operations of a fundamental process, the reduction of infinite function or form systems to a finite number. These developments therefore simultaneously serve as the algebraic foundation of higher number theory (Section C).

From this point, a branching into subspecialties of specific character occurs again in Section B. Apart from the article B 3 a (\textit{C. Runge}), which discusses more practical questions of how to enclose equation roots within suitable limits and approximate them using numerically usable algorithms, the group concept emerges as the predominant one. Among the rational transformations
