\thispagestyle{fancy}

\vspace{0.5cm}

of variables of entire functions (forms) are primarily the linear to be considered (B 2, \textit{W. Fr. Meyer}). If one subjects those variables to any linear group, the coefficients of the given forms are likewise subject to a certain linear group, and the task of linear invariant theory is to establish the invariants of this latter group, or more generally, of any linear group, to classify them appropriately, and to represent the limited series of these as entire or rational functions of a finite number among them. A special interest is claimed by the groups composed of a finite number of substitutions, due to their relationships to theory, analysis, and geometry, to which therefore a special article (B3f, \textit{A. Wiman}) is dedicated.

As the actual carrier of the entire section may be considered the \textit{Galois} theory of equation groups (B 3 c, d, \textit{O. Holder}), already touched upon in B 1 c, which, originally proceeding from the special question of the solvability of certain equations by root signs, has in its further development subordinated the theories of rational as well as arithmetic and geometric rationality domains (as well as the formal integration theories of differential equations). An introduction to this theory is formed by the doctrine (B 3 b, \textit{K. Th. Vahlen}) of one- and multi-valued algebraic functions of one or more quantity series, especially the roots of equations and equation systems.

\textit{Number theory}, or the explicit execution of the characteristics of individual arithmetic rationality domains, could be directly connected to the article B 1 c from today's standpoint. For historical reasons, however, the formation of a separate section (C) was recommended.

Following the presentation of the elementary divisibility laws of natural numbers (C 1, \textit{F. Bachmann}), the treatment of linear, bilinear, quadratic, and certain higher forms, equations, and congruences follows (C 2, \textit{K. Th. Vahlen}); the concepts of elementary divisors and the rank of a matrix, already emerging in algebra, serve here as fundamental principles.

The following article (C 3, \textit{P. Bachmann}) on analytic theory will once do justice to the scattered methods of additive composition of numbers, whose systematic treatment is still pending. On the other hand, it goes into the approximate determination of mean values of number-theoretic functions.
