\thispagestyle{fancy}

\vspace{0.5cm}

Finally, in connection with the periodic properties of algebraic numbers, discusses the recently emerged questions of the transcendence of specific irrationalities, such as \textit{e} and $\pi$.

The main goal of contemporary systematic number theory, the extension of the divisibility laws of natural numbers, as well as subsequently the reciprocity laws of power residues to algebraic number fields, that is, to the rational functions of algebraic numbers, especially quadratic ones, is pursued in C 4 a, b (\textit{D. Hilbert}). While the task in the case of a circular field was solved by introducing ideal numbers, in the general case the creations of field ideals or field forms take their place.

The special case of quadratic class fields occurring in the complex multiplication of elliptic functions finds its resolution in C 6 (\textit{H. Weber}).

Of the lower and higher arithmetic and algebra treated up to this point, it can be said that they constitute a closed unity.

This is not the same for the following sections, already mentioned above; they satisfy more of \textit{a negative definition, belonging neither to the preceding nor to the next two volumes}.

Section D is essentially dominated by probability calculation; although, for example, the adjustment calculation (D 2) can be theoretically constructed without the help of specific probability concepts, the relevant concepts and methods have gradually developed through probability calculation.

\textit{Probability calculation} (D 1, \textit{E. Czuber}), initially only an application of combinatorics to some hazard games, has, namely after the adoption of infinitesimal and geometric concepts, extended its scope so extensively that it lies at the basis of a considerable number of mathematical approximation methods, explicitly or implicitly. From an epistemological perspective, it has the service of resolving the concept of chance, or rather the circumstances accompanying it, to a certain degree into mathematical approaches and laws.

The most productive source of \textit{adjustment calculation} (D 2, \textit{J. Sehinger}) is the principle of the minimum sum of squares, which in a modified form, as the principle of least constraint, can also serve as the source of entire dynamics.

In \textit{interpolation calculation} (D 3, \textit{J. Bauschinger})
