\thispagestyle{fancy}

\vspace{0.5cm}

any arbitrary function determined by certain data through an entire rational function or also through a finite trigonometric series; both in the practical elaboration of the required algorithms and in the more theoretical establishment of remainder terms, difference calculus renders full service, which is independently treated in Article E (\textit{D. Seliwanoff}).

The tasks of \textit{statistics} (D 4 a, \textit{L. v. Bortkiewicz}) and \textit{life insurance} (D4b, \textit{G. Bohlmann}) speak for themselves.

The article F (\textit{R. Mehmke}), on \textit{numerical computation}, contains more than its title indicates. Starting from techniques and tables that serve to facilitate practical calculation with rational and other numbers, the material expands through the use of the most diverse types of computing apparatus and machines into a wide mathematical-technical discipline; also, it becomes a kind of geometric arithmetic through the inclusion of fruitful graphical methods.

The articles G 1 (\textit{W. Ahrens}), G 2 (\textit{L. Pareto}) on \textit{games} and \textit{economic theor} may be viewed as an appendix.

The article G 3 (\textit{A. Pringsheim}) on \textit{infinite processes with complex terms}, which represents a direct supplement to both A 3 and A 4, was originally intended for the second volume. With regard to its close relationship to these essays and in accordance with the arrangement in the French edition of the Encyclopedia, it has subsequently been assigned to the first volume as the concluding article.

To enable the reader a more convenient handling of the volume, it has been divided into two parts. For internal and external reasons, it was recommended to conclude the first part with the last article (B 3 f) of Section B (Algebra).

May the Encyclopedia, which presents the mathematical inventions of a century in historical development, also enliven the epistemological study of the fundamental question of what should actually be considered "new" in mathematics! Does the new lie in an expansion and deepening of a stock of a priori knowledge gained through inner intuition, or does it merely come down to a different grouping of existing empirical facts?

Then it shall still be permitted to set forth the guiding perspectives according to which the \textit{register} has been prepared. It was to be a word and subject register. In the \textit{word register}, only expressions are included
