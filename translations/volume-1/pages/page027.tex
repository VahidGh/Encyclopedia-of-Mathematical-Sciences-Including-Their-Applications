\thispagestyle{fancy}

\vspace{0.5cm}

in the present volume; if this inclusion has also extended to a considerable number of terms that are either outdated, or less in circulation, and may often have sprung from a momentary idea of the author, this was done to satisfy to a certain degree those who are interested in the philological side of word formations. The selection was less straightforward with the considerable treasure of technical expressions, as found primarily in the articles on statistics and life insurance, numerical computation, games, and economic theory. The editor has endeavored to extract expressions for such subjects that still retain a certain measure of mathematical thinking.

Regarding the subject \textit{register}, two difficulties should be pointed out at the outset, which could only be approximately resolved.

First is the question of which authors to cite. With some exceptions, only those authors who no longer belong to the present are considered here under the keywords, and even these mostly only when their name — often merely by chance or even improperly — has become a kind of common currency associated with a specific concept, sentence, or specific method. On the other hand, there was a temptation, when an author was mentioned, to strive for a certain completeness with respect to their outstanding achievements. The justified wish that some might have harbored, that this principle should have been extended to as many or even all authors, could not be satisfied already with regard to the available space. Where, conversely, the name of a researcher was mentioned within the register text, often only for greater clarity, it was not highlighted by print, as such emphasis was reserved for other purposes (see below).

The other difficulty lay in the \textit{word formations} themselves. The large number of contributors makes it understandable that the same word does not always denote the same concept, and conversely, that the same concept — not to mention sentences — is labeled with the most diverse names.

The editor has spared no effort to unite factually related citations at a single location, but is very well aware that many gaps may still remain in this regard, and conversely, that something superfluous may have been
