\thispagestyle{fancy}

\vspace{0.5cm}

in this respect so inhomogeneous that abstention was taken from it, in the hope that it will later be possible, which is also entirely desirable for other reasons, to allow supplementary volumes to follow the respective completed volumes of the Encyclopedia from time to time.

In conclusion, the editor gladly fulfills the duty to express his special thanks in various directions: first and foremost to the Academies of Göttingen, Leipzig, Munich, and Vienna, as well as the commission appointed by them; then to all the gentlemen contributors of the first volume, who have expended their notable time and effort; furthermore to colleague \textit{H. Burkhardt}, who subjected every article to multiple corrections and contributed an extensive series of critical and historical remarks; and not least to the \textit{Teubner} publishing house for its far-reaching accommodation during the arduous and lengthy printing process.

\vspace{1cm}
Königsberg i/Pr., April 1904.

\vspace{0.5cm}

\begin{minipage}{1.65\textwidth}
\begin{flushright}
\centering
\textbf{W. Franz Meyer}

as Editor.
\end{flushright}
\end{minipage}

