
\thispagestyle{fancy}

\vspace{0.5cm}

\setstretch{0.5}

\titlecontents{subsection}[0pt]
  {}
  {\numberline{}\makebox[1.5cm][l]{\thesubsection}}
  {}
  {\titlerule*[0.5pc]{.}\contentspage}

\fontsize{6}{0}\selectfont 
\bfseries 

\begin{enumerate}[itemsep=0pt]
    \setcounter{enumi}{50}
    \item Periodic Continued Fractions\dotfill 130
    \item Transformation of Infinite Continued Fractions\dotfill 133
    \item Transformation of an Infinite Series into an Equivalent Continued Fraction\dotfill 133
    \item Other Continued Fraction Developments of Infinite Series\dotfill 135
    \item Continued Fractions for Power Series and Power Series Quotients\dotfill 136
    \item Relationships between Infinite Continued Fractions and Products\dotfill 139
    \item Ascending Continued Fractions\dotfill 140
    \item Infinite Determinants. Historical\dotfill 141
    \item Main Properties of Infinite Determinants\dotfill 143
\end{enumerate}

\vspace{-0.1cm}
{\normalfont(Completed in September 1898.)}

\subsection*{\small4. Theory of Common and Higher Complex Quantities. \newline \normalfont{By E. Study in Greifswald (now Bonn)}}

\begin{enumerate}[itemsep=0pt]
    \item Imaginary Quantities in the 17th and 18th Centuries\dotfill 148
    \item Calculating with Quantity Pairs\dotfill 149
    \item Common Complex Quantities\dotfill 152
    \item Absolute Value, Amplitude, Logarithm\dotfill 135
    \item Representation of Complex Quantities by Points in a Plane\dotfill 155
    \item Representation of Certain Transformation Groups Using Ordinary Complex Quantities\dotfill 156
    \item General Concept of a System of Complex Quantities\dotfill 159
    \item Types, Shapes, Reducibility\dotfill 162
    \item Systems with Two, Three, and Four Units\dotfill 166
    \item Special Systems with $n^2$ Units. Bilinear Forms\dotfill 168
    \item Special Systems with Commutative Multiplication\dotfill 172
    \item Complex Quantities and Transformation Groups\dotfill 175
    \item Classification of Systems of Complex Quantities\dotfill 180
    \item Approaches to a Function Theory and Number Theory of Systems of Higher Complex Quantities\dotfill 182
\end{enumerate}

\vspace{-0.1cm}
{\normalfont(Completed in November 1898.)}

\subsection*{\small5. Set Theory. \normalfont{By A. Schönflies in Göttingen (now Königsberg)}}

\begin{enumerate}[itemsep=0pt]
    \item Accumulation Points of Point Sets and Their Derivatives\dotfill 185
    \item The Concept of Countability and the Continuum\dotfill 186
    \item Cantor's First Introduction of Transfinite Numbers\dotfill 187
    \item Transfinite Sets. Cardinality or Cardinal Number\dotfill 188
    \item Order Types\dotfill 190
    \item Well-Ordered Sets and Their Sections\dotfill 191
    \item Order Numbers and the Number Class $Z(N_{0})$\dotfill 192
    \item Sets of Higher Cardinality\dotfill 192
    \item General Calculation Laws of Order Numbers\dotfill 193
    \item Normal Form of Order Numbers and 8-Numbers\dotfill 194
    \item General Definitions and Formulas for Point Sets\dotfill 195
    \item General Theorems about Point Sets\dotfill 196
    \item Closed and Perfect Sets\dotfill 197
    \item Decomposition of a Set into Separated and Homogeneous Components\dotfill 198
    \item Content of Point Sets\dotfill 199
    \item The Continuum\dotfill 201
    \item Infinitesimal Calculus. The Infinite (\textit{U}) of Functions\dotfill 202
    \item Archimedes' Axiom and Continuity\dotfill 205
    \item The Most General Size Classes\dotfill 206
\end{enumerate}

\vspace{-0.1cm}
{\normalfont(Completed in November 1898.)}

\subsection*{\small6. Finite Discrete Groups. \normalfont{By H. Bürkhardt in Zürich}}

\begin{enumerate}[itemsep=0pt]
    \item Permutations and Substitutions\dotfill 209
    \item Order of a Substitution\dotfill 210
    \item Cycles\dotfill 210
    \item Analytical Representation of Substitutions\dotfill 211
    \item Substitution Groups\dotfill 211
\end{enumerate}

