\thispagestyle{fancy}

\vspace{0.5cm}

\setstretch{0.5}

\titlecontents{subsection}[0pt]
  {}
  {\numberline{}\makebox[1.5cm][l]{\thesubsection}}
  {}
  {\titlerule*[0.5pc]{.}\contentspage}

\fontsize{6}{0}\selectfont 
\bfseries 

\begin{enumerate}[itemsep=0pt]
    \setcounter{enumi}{6}
    \item Poisson's Method — Eliminant\dotfill 262
    \item Cayley's and Sylvester's Method\dotfill 262
    \item Kronecker's Method — Degree Number\dotfill 263
    \item Minding's Rule — Labatie's Theorem\dotfill 264
    \item Multiple and Infinite Roots of a System of Equations\dotfill 266
    \item Solution of Linear Equations — Special Elimination Problems\dotfill 268
    \item Properties of the Eliminant\dotfill 270
    \item Resultant and its Properties\dotfill 271
    \item Reduced Resultant\dotfill 273
    \item Reducibility and Divisibility of Systems of Equations\dotfill 273
    \item Discriminant of a System of Equations\dotfill 274
    \item Discriminant of an Equation\dotfill 274
    \item Independence of Functions\dotfill 275
    \item Independence of Equations\dotfill 276
    \item Functional Determinant\dotfill 276
    \item Hessian Determinant\dotfill 277
    \item Jacobi's Extension of an Euler Formula\dotfill 278
    \item Root Relations of a System of Equations — Interpolation\dotfill 279
    \item Characteristic of a System of Functions\dotfill 279
    \item Module or Divisor Systems\dotfill 280
    \item Further References\dotfill 280
\end{enumerate}

\vspace{-0.1cm}
{\normalfont(Completed in July 1899.)}

\subsection*{\small1c. Algebraic Structures. Arithmetic Theory of Algebraic Quantities. \newline \normalfont{By G. Landsberg in Heidelberg}}

\begin{enumerate}[itemsep=0pt]
    \item Task of the Arithmetic Theory of Algebraic Quantities\dotfill 284
    \item Fields or Domains of Rationality\dotfill 284
    \item Integral Quantities of a Rationality Domain, Irreducibility\dotfill 286
    \item Conjugate Fields, Discriminants\dotfill 288
    \item Relations to Galois Theory of Equations\dotfill 290
    \item Fundamental Systems\dotfill 292
    \item Types or Species\dotfill 294
    \item Decomposition of Integral Quantities into Prime Divisors or Prime Ideals\dotfill 294
    \item Representation of Prime Divisors through Association of Containing Types or through Association of Transcendental Functions\dotfill 296
    \item The Fundamental Equation\dotfill 298
    \item Implementation of Arithmetic Theory in Detail\dotfill 299
    \item Connection with Theory of Module Systems and Algebraic Structures\dotfill 301
    \item Elementary Properties of Module Systems\dotfill 301
    \item The Degree Concept. Prime Module Systems\dotfill 302
    \item Decomposition into Prime Module Systems. Discriminant of a Module System\dotfill 305
    \item Applications of Module Systems. Complex Numbers with Multiple Units\dotfill 306
    \item Dedekind's Theory of Modules\dotfill 307
    \item Hilbert's Theorems\dotfill 309
    \item Generalization of Divisibility and Equivalence Concepts\dotfill 312
    \item Noether's Fundamental Theorem\dotfill 313
    \item Module Systems of Second Degree, their Normal Forms\dotfill 314
    \item Representation of Algebraic Structures through Rational Parameters, Lüroth's Theorem\dotfill 316
    \item Transformation of Algebraic Structures\dotfill 318
\end{enumerate}

\vspace{-0.1cm}
{\normalfont(Completed in August 1899.)}

\subsection*{\small2. Invariant Theory. \normalfont{By W. Fr. Meyer in Königsberg}}

\begin{enumerate}[itemsep=0pt]
    \item Origins of the Theory\dotfill 322
    \item Development of the Invariant Concept\dotfill 323
    \item Equivalence of Quadratic and Bilinear Forms and Form Systems\dotfill 327
    \item Equivalence of Forms Higher than Second Order\dotfill 334
\end{enumerate}

