\chapter*{}

\vspace{-2.5cm}
{\centering\section*{I A 1. FOUNDATIONS OF ARITHMETIC}}

\begin{center}

(THE FOUR BASIC OPERATIONS; INTRODUCTION OF NEGATIVE\\
AND FRACTIONAL NUMBERS; THIRD-LEVEL OPERATIONS\\
FROM A FORMAL PERSPECTIVE)

\vspace{0.3cm}

{\small BY}

\vspace{0.2cm}

\textbf{H. SCHUBERT}\\[1pt]
{\small IN HAMBURG}

\centerline{\rule{0.5in}{0.2pt}}

\vspace{0.3cm}
\textbf{Table of Contents}
\vspace{0.1cm}

\end{center}

{\fontsize{10}{0}\selectfont

\begin{enumerate}[itemsep=0pt]
    \item Counting and Numbers
    \item Addition
    \item Subtraction
    \item Combination of Addition and Subtraction
    \item Zero
    \item Negative Numbers
    \item Multiplication
    \item Division
    \item Combination of Division with Addition, Subtraction and Multiplication
    \item Fractional Numbers
    \item The Three Third-Level Operations
\end{enumerate}
}

\vspace{-0.4cm}
\centerline{\rule{1in}{0.2pt}}
\vspace{0.3cm}

\textbf{1. Counting and Numbers.} \textit{Counting} things\textsuperscript{1)} means viewing them as similar\textsuperscript{2)}, comprehending them together, and to them individually other things 

\vfill
\leftline{\rule{2in}{0.4pt}}
\vspace{0.2cm}

{
\footnotesize{\fontsize{10}{12}\selectfont

1) That \textit{non-physical} things can also be counted was emphasized by \textit{G. F. Leibniz} against the Scholastics in "De arte combinatoria" (1666), as well as by \textit{J. Locke} in his work "An Essay concerning human understanding" (1690, Book II). In contrast, \textit{J. St. Mill} (Logic, Book III, 26) sees the fact stated in the definition of a number as \textit{physical}, similar to \textit{G. Frege} in "Foundations of Arithmetic" (Breslau 1884).

2) \textit{E. Schröder} in his "Textbook of Arithmetic" (Leipzig 1878) p. 4 emphasizes that the counting process must be preceded by both a combining of the things to be counted and the recognition of the similarity of the things to be counted, as does \textit{E. Mach} in his book "The Principles of Heat Theory" (Leipzig 1896) (No. 7).

}
}
