\thispagestyle{fancy}
\fancyhead[LO]{1. Counting and Numbers}

\vspace{0.5cm}

to assign\textsuperscript{3)}, which are also viewed as similar\textsuperscript{4)}. Each of the things to which other  \hfill things \hfill are  \hfill assigned  \hfill during  \hfill counting  \hfill is  \hfill called

\vspace{-0.1cm}
\leftline{\rule{2in}{0.4pt}}
\vspace{0.1cm}
{
\footnotesize

3) That counting is not possible without a more or less conscious \textit{assignment} or mapping was emphasized by \textit{K. Weierstrass} in the introduction to his lectures on the theory of analytical functions. (See \textit{E. Kossak} "The Elements of Arithmetic", Berlin, Progr. Friedr. Werder-Gymn., 1872). This is similarly emphasized by \textit{E. Schröder} in his textbook, \textit{L. Kronecker} in his essay "On the Concept of Number" (Philosophical Essays dedicated to Zeller, Leipzig 1887, Journ. f. Math. 101), \textit{R. Dedekind} in his work "What are Numbers and What Should They Be?" (Braunschweig 1887 and 1893), in which "assignment" is thoroughly analyzed through a series of definitions (chain) and theorems.

4) In this definition of number, all philosophers and mathematicians essentially agree. However, opinions differ on which psychological moments enable the formation of the number concept. Following \textit{I. Kant's} example, \textit{W. R. Hamilton} emphasizes \textit{time} perception as the foundation of the concept. For him, algebra is "Science of Order in Progression" or "Science of Pure Time". He first expresses this in his paper published in Dubl. Trans. 17, II (1835) "Theory of Conjugate Functions or Algebraic Couples with a Preliminary and Elementary Essay on Algebra as the Science of Pure Time". Later he repeats this view in the preface to his work "Lectures on Quaternions" (Dublin 1853). \textit{H. Helmholtz} takes the same position in his paper "Counting and Measuring" in the philosophical essays dedicated to \textit{Eduard Zeller} (Leipzig 1887); as does \textit{W. Brix} in his paper "The Mathematical Concept and its Development Forms" (Volumes V and VI of \textit{Wundt's} Philosophical Studies; also published as dissertation, Leipzig 1889). In contrast, \textit{J. F. Herbart} in his "Psychology as Science" (Königsberg 1824, Volume II) says that number has no more to do with time than many other types of ideas. \textit{J. J. Baumann} and \textit{F. A. Lange} believe that number aligns far better with \textit{spatial representation} than with time representation, specifically \textit{J. J. Baumann} in his work "The Theory of Space, Time and Mathematics in Modern Philosophy" (Berlin 1869) and \textit{F. A. Lange} in his "Logical Studies" of 1877. Similarly, \textit{F. G. Husserl} opposes efforts to base the concept of number on the idea of time, specifically in Volume I of his work "Philosophy of Arithmetic", Halle 1891.

\textit{Aristotle} is often presented as the first to define number through the concept of time. This is incorrect. \textit{Aristotle} conversely defines time through the concept of number in his "Physics" (Book IV, Chapter 11, p. 219 B or German translation by Prantl, Leipzig 1854, p. 207 ff.), where it states: "Time is the number of movement according to before and after" and further: "Time is the number of circular motion". \textit{Euclid's} often repeated statement (Elements, Book VII) "Number is a multitude of units" can hardly be considered a definition.

Analyses of the number concept can be found, besides in the already cited writings, particularly in the following recent papers and books:

\textit{W. Wundt}, Logic, Volume I;

\textit{G. Frege}, Foundations of Arithmetic, Breslau 1884;

\textit{R. Lipschitz}, Foundations of Analysis, Bonn 1877 (§ 1);

\textit{U. Dini}, Foundations for a Theory of Functions of a Variable Quantity, translated by J. Lüroth (Leipzig 1892);

}
