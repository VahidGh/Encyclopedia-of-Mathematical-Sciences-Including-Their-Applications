\thispagestyle{fancy}
\fancyhead[LO]{1. Counting and Numbers}

\vspace{0.5cm}

a unit\textsuperscript{5)}; each of the things that are assigned to other things during counting is called a one\textsuperscript{5)}. The result of counting is called a \textit{number}. Due to the similarity of the units among themselves and the ones among themselves, the number is independent of the order in which the ones are assigned to the units\textsuperscript{6)}.

When one reminds of a number through an added concept of how the units were viewed as similar, one expresses a \textit{denominate number}. By completely disregarding the nature of the counted things, one arrives from the concept of denominate number to the concept of \textit{undenominate number}. By number alone, an undenominate number is always to be understood.

To \textit{communicate} numbers, one can choose any similar things as ones (fingers, counting beads, chalk marks). Primitive peoples who have no writing use stones or shells as ones when they want to communicate numbers. If one assigns similar written characters to the things to be counted, one obtains the usual \textit{numerical symbols}\textsuperscript{7)}. Thus, in ancient times, the Romans represented the \hfill numbers \hfill from \hfill one \hfill to \hfill nine \hfill through \hfill a \hfill sequence \hfill of  \hfill strokes, the \hfill Aztecs

\vspace{-0.1cm}
\leftline{\rule{2in}{0.4pt}}
\vspace{0.2cm}
{
\footnotesize

\textit{G. Peano}, Arithmetices principia nova methodo exposita, Torino 1889;

\textit{K. Th. Michaelis}, On Kant's Concept of Number, Progr. Charlottenburg, Berlin, at Gärtner, 1884;

\textit{E. Knoch}, On the Concept of Number and Elementary Instruction in Arithmetic, Program of the Realprogymnasium in Jenkau, 1892;

\textit{G. F. Lipps}, Investigations on the Foundations of Mathematics in Wundt's Philosophical Studies (from Volume X onwards); the fourth chapter (in Volume XI) contains the logical development of the number concept.

Of these authors, \textit{G. Peano} and \textit{E. Knoch} build upon Dedekind's foundations in his already cited work.

The laws of arithmetic can be built upon the concept of number without any axiom, as \textit{K. Weierstrass} emphasized in his lectures, among others.

5) The distinction between units and ones in this sense comes from \textit{E. Schröder} (Textbook of Arithmetic and Algebra, Leipzig 1873, p. 5 or Outline of Arithmetic and Algebra, first issue, p. 1).

6) \textit{H. Helmholtz} and \textit{L. Kronecker}, in the already cited essays for Zeller's jubilee, conceive the assignment as if the numbers One, Two, Three, etc. are assigned to the units to be counted. This view is opposed by \textit{E. G. Husserl} in the appendix to the first part of his "Philosophy of Arithmetic" (Halle 1891).

7) On \textit{number communication} and \textit{number representation} through word or writing, one finds detailed information in the following writings:

}
