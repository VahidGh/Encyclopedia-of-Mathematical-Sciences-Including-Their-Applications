\thispagestyle{fancy}
\fancyhead[LO]{1. Counting and Numbers}

\vspace{0.5cm}

represented the numbers from one to nineteen through combinations of individual circles. Modern civilized peoples still have natural symbols only on dice, dominoes, and playing cards. If one assigns similar sounds to the things to be counted, one obtains the natural number sounds, as heard, for example, from the chimes of clocks. Instead of such natural number symbols and number sounds, one usually uses symbols and words that are methodically composed of a few elementary symbols and word stems\textsuperscript{7)}. The modern numeral writing, which is based on the principle of place value and the introduction of a symbol for nothing, was invented by Indian Brahma priests, became known to the Arabs around 800 and reached Christian Europe around 1200, where over the course of the following centuries the new numeral writing and new calculation gradually displaced calculation with Roman numerals\textsuperscript{7)}. 

The study of relationships between numbers is called \textit{arithmetic} ($\alpha\rho\iota\theta\mu\acute{o}\varsigma$ = number). \textit{Calculation} means methodically deriving sought numbers from given numbers. In arithmetic, it is customary to express any number through a \textit{letter}, whereby it must only be noted that within one and the same consideration

\vfill
\leftline{\rule{2in}{0.4pt}}
\vspace{0.2cm}
{
\footnotesize
\textit{M. Cantor}, Mathematical Contributions to Cultural Life of Peoples, Halle 1863;

\textit{G. Friedlein}, The Numerals and Elementary Calculation of Greeks and Romans and of Christian Occident from 7th to 13th Century, Erlangen 1869; Gerbert, The Geometry of Boethius and the Indian Numerals, Erlangen 1861.

\textit{H. Hankel}, On the History of Mathematics in Antiquity and Middle Ages, Leipzig 1874;

\textit{M. Cantor}, Lectures on History of Mathematics, Leipzig 1880, Volume I;

\textit{A. F. Pott}, The Quinary and Vigesimal Counting Methods Among Peoples of All Parts, Halle 1847;

\textit{A. F. Pott}, The Language Diversity in Europe, Demonstrated by Numerals, Halle 1868;

\textit{K. Fink}, Brief Outline of a History of Elementary Mathematics, Tübingen 1890;

\textit{A. von Humboldt}, On the Systems of Numerals Common Among Different Peoples and on the Origin of Place Value in Indian Numbers (J. f. Math., Volume 4);

\textit{P. Treutlein}, Progr. Gymn. Karlsruhe 1875.

\textit{M. Charles}, On the Passage of the First Book of Boethius' Geometry, Brux. 1836; Historical Overview..., Brux. 1837; Par. C. R. 4, 1836; 6, 1838; 8, 1839.

\textit{F. Unger}, The Methodology of Practical Arithmetic in Historical Development, Leipzig 1888;

\textit{H. G. Zeuthen}, History of Mathematics in Antiquity and Middle Ages, Copenhagen 1896;

\textit{H. Schubert}, Counting and Number, a Cultural-Historical Study, in Virchow-Holtzendorff's Collection of Popular Scientific Lectures, Hamburg 1887.

}
