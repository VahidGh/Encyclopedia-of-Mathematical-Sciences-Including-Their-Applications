\thispagestyle{fancy}
\fancyhead[LO]{1. Counting and Numbers}

\vspace{0.5cm}

the same letter must always represent one and the same number\textsuperscript{8)}.

Two numbers \textit{a} and \textit{b} are called \textit{equal}\textsuperscript{9)} when the units of \textit{a} and those of \textit{b} can be assigned to each other in such a way that all units of \textit{a} and \textit{b} participate in this assignment. Two numbers \textit{a} and \textit{b} are called \textit{unequal}\textsuperscript{9)} when such an assignment is not possible. Since during counting the units are considered as similar, it is irrelevant for determining whether \textit{a} and \textit{b} are equal or unequal which units of \textit{a} and \textit{b} are assigned to each other. When two numbers are unequal, one is called the \textit{greater}, the other the \textit{lesser}\textsuperscript{9)}. \textit{a} is called greater than \textit{b} when the units of \textit{a} and \textit{b} can be assigned to each other in such a way that while all units of \textit{b} participate in this assignment, not all units of \textit{a} do. The judgment that two numbers are equal or unequal is called an \textit{equation} or \textit{inequality}, respectively. For equal, greater, lesser, arithmetic uses the three symbols $=, >, <$,\textsuperscript{9)} which are placed between the compared numbers. When drawing a conclusion from multiple comparisons, this is indicated by a horizontal line. The most fundamental conclusions of arithmetic are:

\begin{center}
$\displaystyle \frac{a = b}{b = a}; \qquad \frac{a > b}{b < a}; \qquad \frac{a < b}{b > a}.$
\end{center}

These conclusions refer to only two compared numbers. The following conclusions refer to three numbers:

\vfill
\leftline{\rule{2in}{0.4pt}}
\vspace{0.2cm}
{
\footnotesize
8) The first seeds of arithmetic letter calculation can already be found among the Greeks (\textit{Nikomachos} around 100 AD, \textit{Diophantos} around 300 AD), even more among the Indians and Arabs (\textit{Alchwarizmî} around 800 AD, \textit{Alkalsâdî} around 1450). However, proper letter calculation using the symbols $=, >, <$; and operation symbols was only developed in the 16th century (\textit{Vieta} † 1603), primarily in Germany and Italy. The currently common equality sign first appears in \textit{M. Recorde} (1556). More details on this in:

\textit{L. Matthiessen}, Principles of Ancient and Modern Algebra of Literal Equations, 1878; 

\textit{P. Treutlein}, The German Coss, Zeitschr. f. Math., Volume 24; 

\textit{S. Günther}, History of Mathematical Education in Medieval Germany until 1525, Berlin 1887; Contributions to the Invention of Continued Fractions, Progr., Weissenburg 1872; Mixed Investigations on the History of Mathematical Sciences, Leipzig 1876. 

Only through \textit{L. Euler} (from 1707-1783) did the arithmetic symbolic language acquire its current more fixed form.

9) A more precise analysis of the concepts equal, more, less, greater and smaller can be found in \textit{E. G. Husserl's} Philosophy of Arithmetic, Volume I, Chapters 5 and 6 (Halle 1891), where additional philosophical literature can also be found.

}
