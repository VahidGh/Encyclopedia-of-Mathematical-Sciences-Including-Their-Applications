\thispagestyle{fancy}
\fancyhead[LO]{1. Counting and Numbers}

\vspace{0.5cm}

\begin{center}
$\displaystyle a = m \qquad \,\, a > m \qquad \,\,\,\, a < m \qquad \,\,\, a > m$

$\displaystyle \frac{b = m}{a = b}; \qquad \frac{b = m}{a > b}; \qquad \frac{b = m}{a < b}; \qquad \frac{m > b}{a > b}.$
\end{center}

\textbf{2. Addition}\textsuperscript{10)}. When one has two groups of units, and indeed in such a way that not only are all units of each group similar, but also each unit of one group is similar to each unit of the other group, one can do two things: either one can count each group individually and interpret each of the two counting results as a number, or one can extend the counting over both groups and interpret the counting result as a number. In the former case, one obtains two numbers, in the latter case only one number. One then says of this

\vspace{0.1cm}
\leftline{\rule{2in}{0.4pt}}
\vspace{0.2cm}
{
\footnotesize

10) The logically precise \textit{construction} of the four fundamental operations of arithmetic carried out here in the text was most thoroughly implemented by \textit{E. Schröder} in his textbook (Leipzig 1873, Volume I: The Seven Algebraic Operations). Besides him, the following have contributed to such a construction:

1) \textit{M. Ohm}, Attempt at a Completely Consistent System of Arithmetic (2 volumes, 2nd edition, Berlin 1829);

2) \textit{W. R. Hamilton}, Preface to the Lectures on Quaternions, Dublin 1853;

3) \textit{M. Cantor}, Foundation of Elementary Arithmetic, Heidelberg 1855;

4) \textit{H. Grassmann}, Textbook of Arithmetic, Berlin 1861;

5) \textit{H. Hankel}, Theory of Complex Number Systems, Leipzig 1867 (Section I, I, II);

6) \textit{J. Bertrand}, Treatise on Arithmetic, 4th edition, Paris 1867;

7) \textit{R. Baltzer}, The Elements of Mathematics, Volume I, last (7th) edition, Leipzig 1885;

8) \textit{O. Stolz}, Lectures on General Arithmetic, Leipzig 1885.

A combination of consistent construction with didactic considerations for beginners was first attempted by \textit{E. Schröder} in his Outline of Arithmetic and Algebra, Part I, Leipzig 1874, then more extensively by \textit{H. Schubert} in his textbooks (Collection of Arithmetic and Algebraic Questions and Problems, four editions, Potsdam 1883 to 1896, System of Arithmetic, Potsdam 1885, Arithmetic and Algebra in Göschen Collection (Leipzig 1896,1898).

In earlier centuries, there was even uncertainty about which operations should be considered as basic arithmetic operations, as in the middle of the 15th century with \textit{J. Regiomontanus}, \textit{G. v. Peurbach}, \textit{Lucius Pacioli} and in the \textit{Bamberg arithmetic book}. \textit{Peurbach's} Algorithm, for example, knows eight basic operations, namely Numeratio, Additio, Subtractio, Mediatio, Duplatio, Multiplicatio, Divisio, Progressio.

The operations and laws of arithmetic appear in connection with more general viewpoints in \textit{formal arithmetic}, in \textit{logic calculus}, and in \textit{conceptual notation}. Formal arithmetic studies the relationships of quantities without regard to these quantities being numbers. In particular, one should read on this subject on one hand \textit{H. Grassmann's} Theory of Extension, 1844 and 1878, on the other hand \textit{H. Hankel's} Theory of Complex Number Systems, Leipzig 1867. Regarding logic calculus and conceptual notation, cf. Vol. VI.

}
