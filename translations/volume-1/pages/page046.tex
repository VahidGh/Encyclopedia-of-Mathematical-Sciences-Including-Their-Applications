\thispagestyle{fancy}

\vspace{0.5cm}

summands, by conceiving in the operation of addition one as passive\textsuperscript{12)}, the other as active\textsuperscript{12)}. The passive summand could be called \textit{augend}\textsuperscript{12)}, the active \textit{auctor}\textsuperscript{12)} (increment\textsuperscript{12)}). This conceptually possible distinction is arithmetically unnecessary due to the commutation law.

From the concept of counting further follows:

\begin{center}
$(a + b) + c = a + (b + c)$
\end{center}

The expressed law is called the \textit{association law}\textsuperscript{11)} of addition.

From the uniqueness of addition alone follows the law about combining two equations through addition, whereby from \textit{a} = \textit{b} and \textit{c} = \textit{d} follows: \textit{a} + \textit{c} = \textit{b} + \textit{d}. However, how an equation and an inequality or two inequalities are to be combined can only be recognized through application of the association law. The combined effect of both basic laws of addition yields:

When any number of numbers are combined through addition in any order such that the sum of two numbers always appears again as a summand of a new addition, the number that represents the final result is always the same, regardless of the order in which the given numbers are combined by addition.

This truth justifies calling the final result the \textit{sum of all} given numbers, and thereby extending the concept of sum such that it may have not only two, but any number of summands.

\textbf{3. Subtraction}\textsuperscript{13)}. In addition, two numbers, the two summands \textit{a} and \textit{b}, are given, and from them emerges a third number, the sum \textit{s}. If one now conversely considers the sum \textit{s} and one summand as given, then the other summand emerges from this as a number which (according to No. 2) is completely determined. Finding this number from the sum \textit{s} and the given summand is called \hfill \textit{subtraction}\textsuperscript{13)} \hfill The \hfill number \hfill \textit{s}, \hfill which \hfill was \hfill previously \hfill sum

\vfill
\leftline{\rule{2in}{0.4pt}}
\vspace{0.2cm}
{
\footnotesize
12) The distinction of the two numbers connected by an arithmetic operation through the terms \textit{passive} and \textit{active} was first given by \textit{E. Schröder} in his "Outline of Arithmetic and Algebra" (Leipzig 1874). In addition, he calls the passive number augend, the active increment. \textit{H. Schubert} distinguishes between augend and auctor, for example in his "Arithmetic and Algebra" in "Göschen Collection" (Leipzig).

13) Addition, multiplication and exponentiation are usually called direct operations and their inversions indirect operations. In \textit{H. Hankel} (Theory of complex number systems, 

}
