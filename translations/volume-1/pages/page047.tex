\thispagestyle{fancy}
\fancyhead[LO]{2. Addition.   3. Subtraction}

\vspace{0.5cm}

is called \textit{minuend} in this just defined \textit{inverse}\textsuperscript{13)} calculation method. The given summand could be called "minutor" as the active number. However, the name \textit{subtrahend} is customary for this. The result of subtraction is called difference. The sign of subtraction is a horizontal line (minus), before which one places the minuend and after which one places the subtrahend. Thus it is:

\begin{center}
$(s - a) + a = s$
\end{center}

the \textit{definition formula} of subtraction. From the uniqueness of subtraction follows secondly:

\begin{center}
$(s + a) - a = s$
\end{center}

The \textit{transposition rule of first degree} is also based on the definition of subtraction, whereby a subtrahend (summand) on one side of an equation may be omitted there to appear on the other side as summand (subtrahend). Through transposition, an unknown summand or an unknown minuend can be \textit{isolated}, i.e., it can be arranged that it stands alone on one side of an equation.

An equation is called \textit{identical} if it remains correct regardless of what numbers one may substitute for the letters appearing in it. An identical equation is called a \textit{formula} if it serves to express a truth in arithmetic symbolic language. An equation is called a \textit{determining equation} if it becomes correct only when the letters appearing in it are replaced by specific (not by all) numbers. If in a determining equation all occurring numbers except one are known, one usually denotes the still unknown number with \textit{x}\textsuperscript{14)}, and then arises the task of \textit{solving the equation}, i.e., finding the number that must be substituted for \textit{x} for the equation to become correct. An equation in which \textit{x} is summand or minuend \hfill is \hfill solved \hfill by \hfill the \hfill transposition \hfill rule \hfill of \hfill first \hfill degree \hfill

\vfill
\leftline{\rule{2in}{0.4pt}}
\vspace{0.2cm}
{
\footnotesize
Leipzig 1867) the direct operations of arithmetic belong to the \textit{thetic}, the indirect to the lytic types of connections. From the thetic connection type \textit{a} × \textit{b} = \textit{c} follow the two lytic \textit{c} $\top$ \textit{a} = \textit{b} and \textit{c} $\perp$ \textit{b} = \textit{a}.

14) In \textit{Diophantos}, the unknown is denoted by a final sigma as the last letter of $\alpha\rho\iota\theta\mu\acute{o}\varsigma$. On the origin of the designation \textit{x} for an unknown number, one should read: \textit{Treutlein}, The German Coss, Zeitschr. f. Math. Vol. 24 and the multiply doubted view of \textit{P. A. de Lagarde}, Where does the mathematicians' x come from? (Gött. Nachr., 1882).

}
