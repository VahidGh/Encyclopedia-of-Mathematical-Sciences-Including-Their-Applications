\thispagestyle{fancy}

\vspace{0.5cm}

\textbf{4. Combination of Addition and Subtraction.} From the definition of subtraction follow, when applying the commutation law and the association law of addition, the four formulas:

\begin{center}
$a + (b - c) = (a + b) - c$

$a - (b + c) = (a - b) - c$

$a - (b - c) = (a - b) + c$

$a - b = (a - n) - (b - n)$
\end{center}

Since in each of these formulas there stands a difference on at least one of the two sides, the \textit{proof} of the same only concerns checking whether the other side fulfills the property prescribed by the definition of subtraction, in which checking only the two basic laws of addition or a formula may be applied that precedes the one to be proved here and so can already be considered as proved.

Since from \textit{p} = \textit{q} according to No.1 follows \textit{q} = \textit{p}, every arithmetic formula contains two truths, which one obtains depending on whether one interprets the formula from left to right or from right to left. When one reads the association law of addition and the first three of the four above formulas from left to right, they yield rules about adding and subtracting sums and differences. When one reads them from right to left, they yield rules about increasing and decreasing sums and differences. These formulas provide the proof that summands and subtrahends can be brought into any order, and the rules according to which from an equation and an inequality or from two inequalities through subtraction of the right and left sides a new inequality can be concluded.

The arithmetic symbolic language\textsuperscript{8)} has developed such that in the operations of addition and subtraction the parentheses\textsuperscript{15)} around the \textit{preceding} calculation type may be omitted, but must be placed around the following one, so that the association law of addition and the first three of the above formulas may also be written as:

\vfill
\leftline{\rule{2in}{0.4pt}}
\vspace{0.2cm}
{
\footnotesize
15) \textit{E. Schröder} first states in his textbook the following general rule about placing parentheses in arithmetic: An expression that is part of a new expression is enclosed in parentheses. Gradually it has become customary to omit these parentheses in two cases, first when of two \textit{same-level} operations the \textit{precedin} one should be executed first, second when of two \textit{different-level} operations the one of higher level should be executed first.

}
