\thispagestyle{fancy}
\fancyhead[LO]{4. Combination of Addition and Subtraction}

\vspace{0.5cm}

\begin{center}
$a + (b + c) = a + b + c$; \quad $a + (b - c) = a + b - c$; 

$a - (b + c) = a - b - c$; \quad $a - (b - c) = a - b + c$;
\end{center}

\textbf{5. Zero}\textsuperscript{16)}. Since according to the definition of subtraction the minuend is a sum whose one summand is the subtrahend, the connection of two equal numbers by the minus sign has no meaning. Such a connection has indeed the form of a difference but represents no number in the sense of No. 1. Now arithmetic follows a principle that one calls the \textit{principle of permanence}\textsuperscript{17)} or of exceptionlessness and that consists in four things:

\textit{first, in giving every sign connection that represents none of the numbers defined up to then such a meaning that the connection may be treated according to the same rules as if it represented one of the numbers defined up to then;}

\textit{second, in defining such a connection as a number in the extended sense of the word and thereby extending the concept of number;}

\textit{third, in proving that for numbers in the extended sense the same theorems hold as for numbers in the not yet extended sense;}

\textit{fourth, in defining what equal, greater and lesser means in the extended number domain\textsuperscript{19)}.}

Accordingly, the sign connection \textit{a} - \textit{a} is subjected to the two basic laws of addition and the definition formula of subtraction, whereby it is achieved that the formulas of No. 4 must also hold for the sign connection \textit{a} - \textit{a}. Through

\vfill
\leftline{\rule{2in}{0.4pt}}
\vspace{0.2cm}
{
\footnotesize
16) \textit{Zero} appears as the common symbol for all difference forms in which minuend and subtrahend are equal only since the 17th century. Originally, zero was only a vacat sign for a missing level number in the \textit{place-value} numeral writing invented by the Indians. (Cf. the literature indicated in note 7.) It is called "tziphra" in the arithmetic book of the monk \textit{Maximus Planudes} living in the 14th century (German by \textit{H. Waeschke}, Halle 1878), from which the English cypher and the French zero for zero originated. The German word Ziffer, also coming from tziphra, has gained a more general meaning in German. Other numeral writings, like the additive of the Romans or the multiplicative of the Chinese, have no sign for zero.

17) The principle of \textit{permanence}, which is given here in the text the form suitable for the extension of the number concept, was first expressed in most general form by \textit{H. Hankel} (§ 3 of the Theory of Complex Number Systems, Leipzig 1867), after \textit{G. Peacock} had already emphasized the necessity of a purely formal mathematics and in connection with it a principle from which through extension that of permanence emerges (\textit{G. Peacock} in Brit. Ass. III, London 1834, Symbolical Algebra, Cambridge 1845).

}
