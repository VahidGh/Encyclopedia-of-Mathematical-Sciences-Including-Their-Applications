\thispagestyle{fancy}

\vspace{0.5cm}

application of the formula \textit{a} - \textit{b} = (\textit{a} - \textit{n}) - (\textit{b} - \textit{n}) to \textit{a} - \textit{a}, one then recognizes that all difference forms in which the minuend equals the subtrahend are to be set equal to each other. This justifies introducing a common fixed symbol for all these equal sign connections. This is the symbol 0 (zero)\textsuperscript{16)}. Furthermore, one calls what this symbol expresses a \textit{"number"}, which one also calls zero. But since zero is not a result of counting (No. 1), the concept of number has experienced an extension through the inclusion of zero in the language of arithmetic. From the definition \textit{a} - \textit{a} = 0 follows how to proceed with zero in addition and subtraction, namely: \textit{p} + 0 = \textit{p}, 0 + \textit{p} = \textit{p}, \textit{p} - 0 = \textit{p}, 0 + 0 = 0, 0 - 0 = 0.

\textbf{6. Negative Numbers}\textsuperscript{18)}. When in \textit{a} - \textit{b} the minuend \textit{a} is smaller than the subtrahend \textit{b}, then \textit{a} - \textit{b} represents no number in the sense of No. 1. According to the principle of permanence\textsuperscript{17)} introduced in No. 5, the difference form \textit{a} - \textit{b} must then be subjected to the definition formula of subtraction \textit{a} - \textit{b} + \textit{b} = \textit{a}, from which follows that the formulas treated in No. 4 become applicable to \textit{a} - \textit{b} also in the case where \textit{a} $<$ \textit{b}. Hereby one recognizes that all difference forms can be set equal\textsuperscript{19)} to each other in which the subtrahend is greater than the minuend by the same amount. It is therefore natural to express all difference forms \textit{a} - \textit{b}, in which \textit{b} is greater than \textit{a} by \textit{p}, through \textit{p}. Finally, by calling such difference forms also \textit{"numbers"}, one extends

\vfill
\leftline{\rule{2in}{0.4pt}}
\vspace{0.05cm}
{
\footnotesize
18) Although in a logical construction of arithmetic the introduction of \textit{negative numbers} must precede the introduction of fractional numbers, historically negative numbers came into use much later than fractional numbers. The Greek arithmeticians calculated only with differences in which the minuend was greater than the subtrahend. The first traces of calculating with negative numbers are found with the Indian mathematician \textit{Bhāskara} (born 1114), who distinguishes between the negative and positive value of a square root. The Arabs also recognized negative roots of equations. \textit{L. Pacioli} at the end of the 15th century and \textit{Cardano}, whose Ars magna appeared in 1550, know something of negative numbers but attach no independent meaning to them. \textit{G. Cardano} calls them aestimationes falsae or fictae, \textit{Michael Stifel} (in his Arithmetica integra appearing in 1544) calls them numeri absurdi. Only \textit{T. Harriot} (around 1600) considers negative numbers for themselves and lets them form one side of an equation. The actual calculation with negative numbers, however, begins only with \textit{R. Descartes} († 1650), who assigned to one and the same letter sometimes a positive, sometimes a negative numerical value.

19) That the extension of the concepts equal, greater and lesser brought about by an extension of the number concept requires closer discussion is emphasized in newer textbooks.

}

