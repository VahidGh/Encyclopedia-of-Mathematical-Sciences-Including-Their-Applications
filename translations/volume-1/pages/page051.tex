\thispagestyle{fancy}
\fancyhead[LO]{5. Zero   6. Negative Numbers   7. Multiplication}

\vspace{0.5cm}

 the number concept and arrives at the introduction of negative numbers. Accordingly, the definition formula of the \textit{negative number} -\textit{p} (minus \textit{p}) reads:

\vspace{-0.4cm}
\begin{center}
$- p = b - (b + p)$
\end{center}
\vspace{-0.3cm}

In contrast to negative numbers, the results of counting defined in No. 1 are called \textit{positive} numbers. From the definition formula of the negative number -\textit{p} follows for \textit{p} = 0 that -\textit{p} = 0 - \textit{p}, and since also \textit{p} = 0 + \textit{p}, it is natural to set +\textit{p} for \textit{p}. The plus and minus signs placed before a number (in the sense of No. 1) are called \textit{signs}. Negative numbers thus have the sign minus, positive ones the sign plus. Of the two signs, each is called the \textit{inverse} of the other. Numbers provided with signs are called \textit{relative}. If one omits the sign from a relative number, there results a number in the sense of No. 1, which one calls the \textit{absolute value}\textsuperscript{20)} of the relative number. From these definitions follows how relative numbers are to be connected through addition and through subtraction. As result always appears a relative number or zero.

The introduction of relative numbers makes it possible to conceive any parentheses-free sequence of additions and subtractions as a \textit{"sum"} of purely relative numbers. One calls a sum conceived in this way \textit{algebraic} and the relative numbers that compose it its \textit{terms}. If an algebraic sum stands in parentheses before which a plus sign or minus sign stands, the same may be omitted if one retains or reverses all the signs of the terms contained in it.

Through the introduction of the number zero (No. 5) and negative numbers, the comparison conclusions indicated in No. 2 and No. 4 receive a more extended meaning when one applies greater and lesser also to the newly introduced numbers. One calls, regardless of whether \textit{a} and \textit{b} are zero, positive or negative, \textit{a} $>$ \textit{b} when \textit{a} - \textit{b} is positive, \textit{a} $<$ \textit{b} when \textit{a} - \textit{b} is negative\textsuperscript{19)}.

Finally, the newly introduced numbers also make such equations solvable that according to the original number concept had to be considered unsolvable. Thus the equation \textit{x} + 5 = 5 is unsolvable according to No. 1, but solvable according to No. 5. Thus further the equation \textit{x} + 5 = 3 is unsolvable according to No. 1, but solvable according to this number\textsuperscript{17)}.

\textbf{7. Multiplication}\textsuperscript{21)}. Since due to the basic laws of addition the order

\vspace{0cm}
\leftline{\rule{2in}{0.4pt}}
\vspace{0.1cm}
{
\footnotesize
20) The expression "absolute value" for the modulus of any complex number has become customary through \textit{K. Weierstrass'} lectures.

}
in which additions are performed, the result