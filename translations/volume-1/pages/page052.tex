\thispagestyle{fancy}

\vspace{0.5cm}

end of No. 2 the result of any number of successive additions could be conceived as a \textit{sum of many} summands. If the latter now all represent one and the same number \textit{a}, it is natural to set this number only once and add a sign which indicates \textit{how many} summands \textit{a} the sum should contain. One thereby arrives at a \textit{new} connection of two numbers, namely the number \textit{a}, which is thought of as summand and the number \textit{p}, which counts \textit{how often} this summand is thought. One calls this new connection \textit{multiplication} and designates it as an operation of \textit{second degree}, while one calls addition and its inverse, subtraction, operations of \textit{first degree}. To multiply a number \textit{a} (passive) with a number \textit{p} (active) thus means to calculate a sum of \textit{p} summands, each of which is \textit{a}. The number \textit{a}, \textit{which} appears as summand thereby, is called \textit{multiplicand}, the number \textit{p}, which counts \textit{how often} the summand is thought, is called \textit{multiplier}. The result is called product. Due to No. 5 and No. 6 the multiplicand can be positive, zero or negative. The multiplier however, which indicates \textit{how many} summands are meant, can only be a result of counting, thus only a number in the sense of No. 1. By conceiving a number \textit{a} also as a sum of a single summand, the multiplier may also be the number 1. The sign of multiplication is a point set between the multiplicand \textit{a} and the multiplier \textit{p}. The definition formula of multiplication accordingly reads:

\vspace{-0.3cm}
\begin{center}
$ \quad \quad \quad 1) \quad  2) \quad 3) \quad \quad \quad p)$

$a \cdot p = a + a + a + ... + a$
\end{center}

where the number set above each summand indicates which summand it is. Earlier one set instead of the point the sign ×.

From the uniqueness of addition follows the uniqueness of multiplication, and from that follows that equal multiplied with equal yields equal. From the definition formula of multiplication follow through the formulas of No. 3 and No. 4 the four \textit{distribution laws}\textsuperscript{11)}:

I. \quad $a \cdot p + a \cdot q = a \cdot (p + q)$;

IIa. $a \cdot p - a \cdot q = a \cdot (p - q)$, when $p > q$;

IIb. $a \cdot p - a \cdot q = 0$, when $p = q$;

IIc. $a \cdot p - a \cdot q = -[a \cdot (q - p)]$, when $q > p$;

III. \, $a \cdot p + b \cdot p = (a + b)p$;

IV. \, $a \cdot p - b \cdot p = (a - b)p$. 

[About omitting the parentheses on the left sides of these formulas read note 15.]
