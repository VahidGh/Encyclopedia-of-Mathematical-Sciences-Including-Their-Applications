\thispagestyle{fancy}
\fancyhead[LO]{7. Multiplication}

\vspace{0.5cm}

Formulas I and II show, read forward, how a common multiplicand is \textit{separated}, read backward, how to multiply \textit{with} a sum or difference. Formulas III and IV show, read forward, how a common multiplier is \textit{separated}, read backward, how a sum or difference is multiplied. From the distribution laws follows how an equation and an inequality or two inequalities are to be combined through multiplication, if the four compared numbers are positive.

How a product is to be treated whose multiplicand is zero or negative follows from No. 5 and No. 6. But when in a product the \textit{multiplier is zero or negative}, this initially represents a meaningless sign connection. According to the principle of permanence\textsuperscript{17)}, it is now to be given a meaning that permits calculating with it according to the same rules as if the multiplier were a difference that represents a number in the sense of No. 1. Therefore in formula II the restriction \textit{p} $>$ \textit{q} is to be lifted, to derive from it how to multiply \textit{with} zero and negative numbers. Thus it follows that \textit{a}·0 = 0 and \textit{a}·(-\textit{w}) = -(\textit{a}·\textit{w}). From this then also follows how relative numbers are to be connected through multiplication.

From the distribution laws also follows that for multiplication the \textit{commutation law}\textsuperscript{11)} and the \textit{association law}\textsuperscript{11)} are correct.

The commutation law of multiplication eliminates the necessity of distinguishing between multiplicand and multiplier for pure arithmetic. One therefore designates both with the common name \textit{factor} and writes them in any order. One calls the product a \textit{multiple} of each of its factors and each factor a \textit{divisor} of the product. Furthermore, one calls in a product each factor the \textit{coefficient} of the other.

With denominate numbers the distinction between multiplicand and multiplier emerges through the fact that the former can be denominate but the latter must be undenominate. Therefore with denominate multiplicand the commutation law is meaningless.

For the arithmetic of undenominate numbers follows from the combined effect of the commutation law and the association law that the order in which multiplications follow each other is irrelevant regarding the final result. This justifies extending the concept of product such that it may have not just two, but any \textit{number of factors}.
