\thispagestyle{fancy}

\vspace{0.5cm}

Through the fact that these factors can all represent the same number, the possibility of defining a direct operation of third degree - exponentiation\textsuperscript{21)} - is given.

An algebraic sum whose terms can also be products is called a \textit{polynomial}. One multiplies two polynomials by multiplying each term of one with each term of the other and combining the obtained products again into an algebraic sum. Each term becomes positive or negative depending on whether it arises through multiplication from terms with equal or unequal signs. The proof of this rule follows from formulas I to IV.

From the definitions and results developed so far, one can conclude that when any number of numbers that are zero or relative are connected in any way through addition, subtraction and multiplication, the final result must always be zero or relative, thus one of the numbers defined so far.

\vspace{0.5cm}

\textbf{8. Division}\textsuperscript{10)}. Division emerges from multiplication through inversion\textsuperscript{13)}, namely through considering the product and one factor as given, the other factor as sought. Thereby the number that was previously product receives the name \textit{dividend}, the given factor the name \textit{divisor}, the sought factor the name \textit{quotient}. The sign of division is a colon (read: divided by), before which one places the dividend and after which one places the divisor. Accordingly, the \textit{definition formula of division} reads:

\begin{center}
$(p:a) \cdot a = p$
\end{center}

Instead of \textit{p}:\textit{a} one also writes $\frac{p}{a}$, more rarely \textit{p}/\textit{a}. Like subtraction, division also has conceptually two inversions, since both the passive factor, the multiplicand, and the active factor, the multiplier, can be sought. If the dividend is a denominate number, then finding the multiplicand is called \textit{partition}, finding the multiplier \textit{measurement}. Due to the commutation law, however, with undenominate numbers it is unnecessary to distinguish between the two inversions of multiplication. For \textit{p}:\textit{a} to have meaning, \textit{p} must be able to be a product whose one factor is \textit{a}, i.e., \textit{p} must be a multiple of \textit{a}, or, what is the same, \textit{a} must be a divisor of \textit{p}. 

From the fact that 0·\textit{m} = 0 follows two things: \hfill

\vfill
\leftline{\rule{2in}{0.4pt}}
\vspace{0.2cm}
{
\footnotesize
21) Cf. here No. 11.

}
