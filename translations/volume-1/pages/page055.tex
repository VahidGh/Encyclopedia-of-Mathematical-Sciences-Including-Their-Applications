\thispagestyle{fancy}
\fancyhead[LO]{8. Division   9. Combination of Division with Addition, Subtraction and Multiplication}

\vspace{0.5cm}

1) Zero divided by zero is to be set equal to any arbitrary number. Therefore one calls the sign connection 0:0 \textit{ambiguous}.

2) Zero divided by any arbitrary number always yields the number zero.

But when the divisor is zero and the dividend is not zero but any relative number \textit{p}, then arises the question which number, multiplied by zero, leads to the relative number \textit{p}. Since none of the numbers defined so far has the required property, the principle of permanence\textsuperscript{17)} is to be applied. But the investigation of \textit{what} meaning is then to be assigned to \textit{p}:0 when \textit{p} is not zero belongs in another chapter of mathematics (cf. I A 3).

Since division leads uniquely to one of the already defined numbers only when the divisor is not zero, one may conclude a third equation from two equations through division only \textit{when the divisors are different from zero}. Many fallacies of elementary arithmetic as well as higher analysis are based on disregarding this restriction.

How relative numbers are divided follows from the corresponding rules for the multiplication of relative numbers.

From the definition formula of division also follows:

\vspace{-0.2cm}
\begin{center}
$(p \cdot a): a = p $, if \textit{a} is not zero
\end{center}
\vspace{-0.2cm}

This formula yields in conjunction with the definition of division the rule that multiplication and division with the same number cancel each other out, \textit{if this number is not zero}.

From the fact that the two equations \textit{x}·\textit{b} = \textit{p} and \textit{x} = \textit{p}:\textit{b} mutually condition each other, if \textit{b} is not zero, follows the \textit{transposition rule of second degree}. Through second-degree transposition one can either isolate an unknown factor or an unknown divisor and accomplish the solution of determining equations.

\vspace{0.1cm}
\textbf{9. Combination of Division with Addition, Subtraction and Multiplication.} With help of the definition formula of division (No. 8) one can recognize the correctness of the following formulas:

I. $(\textit{a} + \textit{b}):\textit{m} = \textit{a}:\textit{m} + \textit{b}:\textit{m}$, \quad 

II. $(\textit{a} - \textit{b}):\textit{m} = \textit{a}:\textit{m} - \textit{b}:\textit{m}$, \quad 

III. $\textit{a} \cdot (\textit{b}:\textit{c}) = \textit{a} \cdot \textit{b}:\textit{c}$,


IV. $(\textit{a}:\textit{b}) \cdot \textit{c} = \textit{a}:\textit{b}:\textit{c}$, \quad 

V. $\textit{a}:(\textit{b}:\textit{c}) = \textit{a}:\textit{b} \cdot \textit{c}$, \quad 

VI. $\textit{a}:\textit{b} = (\textit{a} \cdot \textit{m}):(\textit{b} \cdot \textit{m})$, \quad 

VII. $\textit{a}:\textit{b} = (\textit{a}:\textit{n}):(\textit{b}:\textit{n})$

