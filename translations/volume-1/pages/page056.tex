\thispagestyle{fancy}

\vspace{0.5cm}

[About omitting the parentheses on the right sides of F. I to V read note 15.]

Here the occurring divisors are naturally to be understood as divisors of the associated dividend. In particular, none of the divisors may be zero. Formulas III, IV, V, VII correspond in the second degree exactly to the four formulas established in No. 4 for the first degree.

The two distribution formulas III and IV in No. 7, as well as the formulas designated here with I and II teach, read in one direction, how a sum or difference is multiplied or divided, read in the other direction, how products with equal factor or quotients with equal divisor are added or subtracted. In the first case parentheses are dissolved, in the second set.

From formulas I and II also follows how an algebraic sum is divided by a number, and how conversely any algebraic sum of quotients with common divisor can be transformed into a quotient whose divisor is the common divisor of all terms. When in an algebraic sum of quotients the \textit{divisors are different}, one can transform these quotients through formula VI into other quotients that all have the same divisor (general divisor), and then apply the rule just mentioned.

The association law of multiplication and the above formulas III, IV, V teach, depending on whether one reads them in one direction or the other, both how to multiply or divide with products or quotients, and how products or quotients are multiplied or divided. In the first case parentheses are dissolved, in the second set. Furthermore, these forms show that factors and divisors can be brought into any order without the final result thereby changing.

When two quotients have equal positive divisor, that one represents the larger number which has the larger dividend. But when two quotients, whose dividend and divisor are positive, have equal dividend, that one which has the larger divisor represents the \textit{smaller} number. These rules follow from the established formulas and yield how an inequality and an equation or two inequalities are to be combined through division when the divisors are positive and divisors of the associated dividends.
