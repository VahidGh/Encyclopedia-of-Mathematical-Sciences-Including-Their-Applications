\thispagestyle{fancy}
\fancyhead[LO]{10. Fractional Numbers}

\vspace{0.5cm}

\textbf{10. Fractional Numbers}\textsuperscript{21)}. In §5 and §6 the principle of permanence\textsuperscript{17)} created from \textit{a} - \textit{b}, where \textit{a} is not greater than \textit{b}, zero and negative numbers. In the same way arise from \textit{a}:\textit{b}, where \textit{a} is not a multiple of \textit{b}, the \textit{fractional numbers}, namely through transferring the definition formula of division

\begin{center}
$(\textit{a}:\textit{b}) \cdot \textit{b} = \textit{a}$
\end{center}

to \textit{a}:\textit{b}, if \textit{a} is not a multiple of \textit{b}. Thus one also achieves the transfer of all definitions and formulas established so far to the quotient form \textit{a}:\textit{b} and the lifting of the restriction expressed in No. 8 and No. 9, "if the divisor is a divisor of the dividend". In particular, the equation \textit{b}·\textit{x} = \textit{a} now appears solvable even when \textit{b} is not a divisor of \textit{a}.

By calling the quotient form \textit{a}:\textit{b}, where \textit{b} is not a divisor of \textit{a}, a \textit{"number"}, one extends anew the concept of number, enlarges the field of investigation of arithmetic and perfects the means\textsuperscript{22)} with which it works. In contrast to the fractional numbers thus arising, all numbers defined so far (Nos. 1, 5, 6) are called \textit{whole numbers}. The dividend \textit{a} of a fraction \textit{a}:\textit{b} is called its \textit{numerator}, the divisor \textit{b} its \textit{denominator}. One designates a fraction by a horizontal line\textsuperscript{21)}, a whole number set above it, which is its numerator, and a whole number set below it, which is its denominator.

\vfill
\leftline{\rule{2in}{0.4pt}}
\vspace{0.2cm}
{
\footnotesize
21) Calculation with \textit{fractions} was already done in antiquity. Indeed, the oldest mathematical manual, the \textit{Rhind Papyrus} in the British Museum, already contains a peculiar fraction calculation (details in \textit{M. Cantor's} History of Mathematics, Volume I), in which each fraction is written as a sum of different \textit{unit fractions}. The \textit{Greeks} distinguished numerator and denominator in their letter-numeral writing through different stroking of the letters, but preferred unit fractions. The \textit{Romans} sought to represent fractions as multiples of $\frac{1}{12}, \frac{1}{24}$, etc. up to $\frac{1}{288}$, in connection with their coin division. The \textit{Indians} and \textit{Arabs} knew unit fractions and derived fractions, but preferred, just like the ancient \textit{Babylonians} and, following them, the \textit{Greek astronomers}, \textit{sexagesimal fractions} (cf. note 24). The \textit{fraction line} and today's writing of fractions comes from \textit{Leonardo of Pisa} (called \textit{Fibonacci}), whose liber abaci (around 1220) became the source for the arithmetic books of the next centuries.

22) That the introduction of negative and fractional numbers can be dispensed with by algebra, and that those numbers are only symbols that facilitate calculation, \textit{L. Kronecker} expounded in his treatise "On the Concept of Number" (J. f. Math. 101, 1887). According to \textit{Kronecker}, therefore, the extensions of the number concept serve only for what \textit{E. Mach} calls "economy of science". (Cf. \textit{E. Mach}, Mechanics, Leipzig 1883; Popular Scientific Lectures, Leipzig 1896; The Principles of Heat Theory, Leipzig 1896, p. 391 ff.)

}
