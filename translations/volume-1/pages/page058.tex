\thispagestyle{fancy}

\vspace{0.5cm}

To compare fractions, one brings them to a common denominator through application of formula VI in No. 9 and calls a fraction equal to another, greater or smaller\textsuperscript{19)}, than the other, when its numerator is \textit{equal} to the numerator of the other fraction, \textit{greater} or \textit{smaller} than its numerator. One calls a fraction $\frac{a}{b}$ greater or smaller than the whole number \textit{c}, depending on whether $\textit{a} > \textit{b} \cdot \textit{c}$ or $\textit{a} < \textit{b} \cdot \textit{c}$. A fraction that is smaller than 1 is called \textit{proper}, a fraction that is greater than 1 \textit{improper}. Through application of the considerations in No. 5 and No. 6 to fractions, one arrives at the concepts of negative fraction, positive fraction, relative fraction and absolute value\textsuperscript{20)} of a relative fraction. Each of the numbers defined so far is thus zero or positive-whole or negative-whole or positive-fractional or negative-fractional. One combines all numbers that have one of these characteristics, thus all numbers defined so far, through the word \textit{"rational"}\textsuperscript{23)}, in contrast to the later defined irrational numbers (cf. IA 3). The rules for how rational numbers are to be connected through addition, subtraction, multiplication and division follow from the formulas established in the earlier paragraphs. 

According to formula VII in No. 9, a fraction whose numerator and denominator have a common whole-number divisor can be set equal to that fraction which arises when one divides numerator and denominator by this divisor. This procedure is called \textit{reducing} the fraction. The first-degree analogue to reducing fractions is the reduction of the minuend and subtrahend of a difference by one and the same whole number. While however in this procedure any arbitrary whole number can be achieved as minuend and as subtrahend from any arbitrary difference of whole numbers, through reducing any arbitrary fraction not any arbitrary whole number can be achieved as numerator

\vfill
\leftline{\rule{2in}{0.4pt}}
\vspace{0.2cm}
{
\footnotesize
23) The distinction between rational and irrational quantities appears among the Greeks in geometric form already before \textit{Euclid} (around 300 BC), first probably with \textit{Pythagoras} (around 500), who recognized that the hypotenuse of an isosceles right triangle is unspeakable ($\alpha\rho\rho\eta\tau o\varsigma$) when the catheti are speakable. \textit{Plato} (429-348) recognized the irrationality in the diagonal of the square over five (Plato's Republic, VII 546). Even more extensively \textit{Euclid} treated the irrational ($\alpha\lambda o\gamma o\nu$) in the 10th book of his "Elements", and indeed in geometric form, distinguishing whether two lines are commensurable or incommensurable ($\alpha\sigma\upsilon\mu\mu\epsilon\tau\rho o\varsigma$). \textit{Archimedes} (287-212) in his calculation of the number $\pi$ enclosed the square root of three and of other numbers in very close rational bounds. About the irrational in modern times cf. here IA 3.

}
