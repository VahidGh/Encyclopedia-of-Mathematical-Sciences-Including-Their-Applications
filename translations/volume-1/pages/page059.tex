\thispagestyle{fancy}
\fancyhead[LO]{10. Fractional Numbers}

\vspace{0.5cm}

or denominator. One must be content to reduce until numerator and denominator no longer have a common divisor. If the numerator of a fraction is a divisor of the denominator, then through reduction a fraction can be achieved whose numerator is 1. Such fractions with numerator 1 are called \textit{unit fractions}\textsuperscript{21)}. Any arbitrary fraction can be conceived as the product of its numerator with its unit fraction, i.e., with that fraction which has the same denominator. When \textit{a} $>$ \textit{b} and \textit{a} and \textit{b} are whole numbers, then \textit{a} = \textit{m} · \textit{b} + \textit{v} can be set where \textit{m} is a quite definite whole number and \textit{v} $<$ \textit{b}, from which follows that every fraction is greater than a whole number by a proper fraction, or that every rational number can be enclosed\textsuperscript{19)} between two bounds that are whole numbers, differ by 1, and of which one is greater, the other smaller\textsuperscript{19)}, than the rational number.

Through continuation of the place-value principle, on which our numeral writing is based, to the right one arrives at \textit{decimal fractions}\textsuperscript{24)}, i.e. fractions of which only the numerator needs to be written because the denominator is ten or hundred or thousand etc. Which of these numbers is meant as denominator is indicated by the position of a comma\textsuperscript{24)}. (Cf. Numerical Calculation in IF.)

The introduction of relative numbers transforms every subtraction into an addition, namely through reversal of a sign. Similarly, the introduction of fractional numbers transforms every division into a multiplication. If one understands by \textit{reciprocal value} of a fraction the fraction whose numerator and denominator are denominator and numerator of the original fraction, then one recognizes that the result of division by a fraction agrees with the result of multiplication with its reciprocal value.

The arithmetic which comprises only the operations of first and second degree concludes, as follows from the above, with the following two final results:


\leftline{\rule{2in}{0.4pt}}
\vspace{0.1cm}
{
\footnotesize
24) \textit{Decimal fractions} arose during the 16th century. \textit{Johann Kepler} (1571-1630) introduced the decimal comma. The principle underlying decimal fraction notation was already used in antiquity with \textit{sexagesimal fractions}. In these one lets multiples of $\frac{1}{60}$ and then of $\frac{1}{3600}$ follow the wholes. That these are of \textit{Babylonian} origin has become undoubted through the discovery of a sexagesimal place-value numeral writing (with 59 different numerals, but without a sign for nothing) used by Babylonian astronomers. The Greek astronomers too (\textit{Ptolemy}, around 150 AD) calculated with sexagesimal fractions. For example, \textit{Ptolemy} set $\pi$ = 3 .. 8 .. 30 = 3 + $\frac{8}{60}$ + $\frac{30}{3600}$. Our sixty-division of the hour and degree, as well as the expressions \textit{minute} (pars \textit{minuta} prima) and \textit{second} (pars minuta \textit{secunda}) are remnants of the old sexagesimal fractions.

}
