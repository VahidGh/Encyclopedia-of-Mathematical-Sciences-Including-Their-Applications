\thispagestyle{fancy}

\vspace{0.5cm}

1) When general number symbols (letters) are connected in any way through operations of first and second degree, the result can always be represented as a quotient whose dividend and divisor is an algebraic sum of products;

2) When any number of rational numbers are connected in any way through operations of first and second degree, the result is always a rational number, provided division by zero does not occur.

New extensions of the number domain become necessary only when one connects the numbers defined so far through operations of third degree. (Cf. No. 11 as well as IA 3 and IA 4.)

\vspace{0.2cm}

\textbf{11. The Three Operations of Third Degree.} In No. 7 the definition of product is extended such that it may contain any number of factors. The case where these are all equal leads to the direct operation of third degree, \textit{exponentiation}\textsuperscript{25)}. To raise a number \textit{a} (passive) to the power of a number \textit{p} (active) thus means to form a product of \textit{p} factors, each of which is \textit{a}. The number \textit{a}, \textit{which} is set as factor of a product, is called \textit{base}, the number \textit{p}, which indicates \textit{how often} the other number \textit{a} is to be set as factor of a product, is called \textit{exponent}\textsuperscript{25)}. The result of exponentiation, which one writes \textit{a}\textsuperscript{\textit{p}} and reads "a to the power p", is called power. Insofar as one conceives \textit{a} as product of one factor, one sets \textit{a}\textsuperscript{1} = \textit{a}. Through the definition of exponentiation,

\vfill
\leftline{\rule{2in}{0.4pt}}
\vspace{0.2cm}
{
\footnotesize
25) Powers with exponents 1 to 6 were already designated in abbreviated form by \textit{Diophantus}. He calls the second power $\delta\upsilon\nu\alpha\mu\iota\varsigma$, a word to which through the Latin translation potentia the word "power" is to be traced back. In the 14th to 16th centuries there are already traces of calculation with powers and roots, as with \textit{Oresme} (†1382), \textit{Adam Riese} (†1559), \textit{Christoff Rudolf} (around 1530) and notably with \textit{Michael Stifel} in his Arithmetica integra (Nuremberg 1544). More details about this in \textit{M. Cantor's} History of Mathematics. But only the invention of logarithms at the beginning of the 17th century procured full citizenship in arithmetic for operations of third degree. The deeper understanding of their connection however belongs to an even later time. The inventors of logarithms are \textit{Jost Bürgi} (†1632) and \textit{John Napier} (†1617). \textit{Kepler} (†1630) also has great merits regarding the spread of knowledge of logarithms, \textit{Henry Briggs} (†1630) introduced base ten and published a collection of logarithms of this base. (Cf. Numerical Calculation, IF.) The word logarithm ($\lambda o\gamma o\nu \; \alpha\rho\iota\theta\mu o\varsigma$ - number of a ratio) is explained by the fact that one sought to relate two ratios by raising one to a power to obtain the other. Thus one called 8 to 27 the third ratio of 2 to 3. Also the expression "numerus rationem exponens" occurs for logarithm, from which perhaps the word "exponent" derives.

}
