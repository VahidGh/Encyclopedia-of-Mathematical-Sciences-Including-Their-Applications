\thispagestyle{fancy}
\fancyhead[LO]{11. The Three Operations of Third Degree}

\vspace{0.5cm}

the following laws of exponentiation arise from the laws of multiplication:

\begin{description}[
    leftmargin=1cm,
    style=multiline
]
    \item[ ] $\left.
    \begin{tabular}{l}
        I. \quad $a^p \cdot a^q = a^{p+q}$;\\

        IIa. $a^p : a^q = a^{p-q}$, if $p > q$ is;\\
        
        IIb. $a^p : a^q = 1$, if $p = q$ is;\\
        
        IIc. $a^p : a^q = 1:a^{q-p}$, if $p < q$ is;
    \end{tabular}
    \right\} \raisebox{-3ex}{\begin{tabular}[b]{l} (Distribution \\ formulas with \\ same base.) \end{tabular}}$

    \item[ ] $\left.
    \begin{tabular}{l}
        III. $a^q \cdot b^q = (a \cdot b)^q$; \\

        IV. $ a^q : b^q = (a : b)^q$; \\
    \end{tabular}
    \right\} \raisebox{-1.5ex}{\begin{tabular}[b]{l} (Distribution formulas with \\ same exponent.) \end{tabular}}$

    \vspace{0.2cm}

    \quad V. $(a^p)^q = a^{p \cdot q} = (a^q)^p$; \quad (Association formula.)
\end{description}

According to the definition of exponentiation, the base can be any number; the exponent however must be a result of counting, thus a positive whole number. For "positive-whole" one also says \textit{"natural"}; accordingly a power with such an exponent is called a \textit{"natural"} one.

Due to a geometric application, powers with exponent 2 are also called squares, with exponent 3 also cubes.

If the base is a sum, a difference, a product, a quotient or a power, it is to be enclosed in parentheses. On the other hand, the higher position of the exponent makes parentheses around it superfluous.

According to the definition of exponentiation, $a^0$ and $a^{-n}$, where -n is a negative whole number, are initially meaningless signs. Also products whose multiplier is zero or negative were, according to the original definition of multiplication, meaningless signs. Yet such signs received meaning, according to the principle of permanence, through the desire to be able to multiply with such differences just as with differences that represent a positive number. In the same way one proceeds with the power forms

\vspace{-0.3cm}
\begin{center}
$a^0 \quad and \quad a^{-n}.$
\end{center}
\vspace{-0.3cm}

One thus sets $a^0 = a^{p-p}$, lifts the restriction $p > q$ in formula IIa, and applies the same, read backwards. Then comes:

\vspace{-0.3cm}
\begin{center}
    $a^0 = a^{p-p} = a^p : a^p = 1$
\end{center}
\vspace{-0.2cm}

Similarly one sets $a^{-n} = a^{p-(p+n)}$, lifts the restriction $p > q$ in formula IIa, finds thereby $a^p : a^{p+n}$, now applies formula IIc and obtains $1:a^n$.

The extension of the concept of exponentiation to the case where the exponent is a fractional number can only be accomplished after the laws of radicalization, one of the two inversions of exponentiation, are established.
