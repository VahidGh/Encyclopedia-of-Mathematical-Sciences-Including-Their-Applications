\thispagestyle{fancy}

\vspace{0.5cm}

Since in exponentiation the commutation law does not hold, because in general $b^n$ is not equal to $n^b$, the two inversions of exponentiation must be \textit{distinguished} not only logically but also \textit{arithmetically}. The operation which in $b^n = a$ considers the base, thus the passive number, as sought, but $a$ and $n$ as given, is called \textit{radicalization}\textsuperscript{25)}; the operation which in $b^n = a$ considers the exponent, thus the active number, as sought, but $b$ and $a$ as given, is called \textit{logarithmization}\textsuperscript{25)}.

\textit{"$n$th root of $a$"}, written: $\sqrt[n]{a}$, \textsuperscript{25)} is thus the number which, raised to the $n$th power, yields $a$. Accordingly, $(\sqrt[n]{a})^n = a$ is the definition formula of radicalization. The number which was originally power is called \textit{radicand} in radicalization, the number which was power exponent is called \textit{root exponent} and the number which was base is called \textit{root}. Through the definition of radicalization arise from the laws of exponentiation the following laws of radicalization:

\begin{description}[
    leftmargin=1cm,
    style=multiline
]
    \item[ ] $\left.
    \begin{tabular}{l}
        I. $\sqrt[n]{a} \cdot \sqrt[n]{b} = \sqrt[n]{a \cdot b}$ ;\\

        II. $ \sqrt[n]{a} : \sqrt[n]{b} = \sqrt[n]{a : b}$ ;
    \end{tabular}
    \right\} \raisebox{0ex}{\begin{tabular}[b]{l} (Distributive formulas.) \end{tabular}}$

    \item[ ] $\left.
    \begin{tabular}{l}
        III. $\sqrt[p]{a}^q = (\sqrt[p]{a})^{q}$ ;\\

        IV. $\sqrt[p]{\sqrt[q]{a}} = \sqrt[pq]{a} = \sqrt[q]{\sqrt[p]{a}}$ ;
    \end{tabular}
    \right\} \raisebox{0ex}{\begin{tabular}[b]{l} (Associative formulas.) \end{tabular}}$

    \vspace{0.2cm}

    \quad V. $\sqrt[np]{a^{nq}} = \sqrt[p]{a}^q$ .
\end{description}

Through radicalization, powers with \textit{fractional exponents} can be defined. Since $\frac{p}{q} \cdot q = p$ is the definition formula of the fractional number $\frac{p}{q}$, and since $a^{\frac{p}{q} \cdot q}$ equals $(a^\frac{p}{q})^{q}$, under $a^{\frac{p}{q}}$ is to be understood a number which, raised to the $q$th power, yields $a^p$, and this is $\sqrt[q]{a^p}$.

Similarly one recognizes that $a^{-\frac{p}{q}} = 1 : \sqrt[q]{a^p}$. Furthermore formula V shows how $\sqrt[n]{a}$, where $n$ is a positive or negative fractional number, can be represented as a power whose base is $a$ and whose exponent is rational. Every root can thus be represented as a power whose base is the radicand of the root, just as every quotient can be represented as a product whose multiplicand is the dividend of the quotient.

When $a$ is any rational number and $n$ is a whole number, then $a^n$ represents a rational number. But when with rational $a$ the number $n$ is indeed rational but not whole-numbered, then there exists only a rational number that may
