\thispagestyle{fancy}
\fancyhead[LO]{11. The Three Operations of Third Degree}

\vspace{0.5cm}

 be set equal to $a^n$ when $a$ is the $q$th power of a rational number, where $q$ is the denominator of the number $n$. In all other cases $a^n$, where $n$ is not whole-numbered, represents a sign connection which still needs to be given meaning. (Cf. IA 3 and IA 4)

 \textit{"Logarithm of $a$ to base $b$"}\textsuperscript{25)}, written: $\log_b a$, is the exponent with which $b$ must be raised to yield $a$. Accordingly:

\begin{center}
    $b^{\log_b a} = a$
\end{center}

is the definition formula of \textit{logarithmization}. The number which was originally power is called \textit{logarithmand} in logarithmization, the number which was base is called \textit{logarithm base}, and the number which was exponent is called \textit{logarithm}. Through the definition of logarithmization arise from the laws of exponentiation the following laws of logarithmization:

\vspace{0.5cm}
I. $\log_b(p \cdot q) = \log_b p + \log_b q$ ;\\

II. $\log_b(p : q) = \log_b p - \log_b q$ ;\\

III. $\log_b(p^m) = m \cdot \log_b p$ ;\\

IV. $\log_b a = \frac{\log_c a}{\log_b a}$ .
\vspace{0.5cm}

When $b$ is any rational number, then $\log_b a$ represents a rational number only when $a$ equals a power whose base is $b$ and whose exponent is a rational number. [This is for example the case when $b = \frac{9}{4}$ and $a = \frac{8}{27}$, then $(\frac{9}{4})^{-\frac{3}{2}} = (\frac{4}{9})^{+\frac{3}{2}} = (\sqrt[2]{\frac{4}{9}})^3 = (\frac{2}{3})^3 = \frac{8}{27}$. Therefore $\log_{\frac{9}{4}} \frac{8}{27}$ equals the rational number $-\frac{3}{2}$.] In all other cases $\log_b a$ represents a sign connection which still needs to be given meaning. (Cf. IA 3 and IA 4.)

\vspace{0.5cm}
\centerline{\rule{1in}{0.4pt}}
\vspace{0.5cm}

In the following table of arithmetic operations, 16 is always considered as passive, 2 as active number.
