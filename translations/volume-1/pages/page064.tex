\thispagestyle{fancy}

\vspace{0.5cm}

{\centering
\textbf{Table of 7 Operations}

\vspace{0.3cm}

{\fontsize{10}{12}\selectfont
\renewcommand{\arraystretch}{1.5}
\begin{tabular}{|c|c|c|c|c|}
    \hline
    Operation & Example & \makecell{The passive \\ number, here 16, \\ is called:} & \makecell{The active \\ number, here 2, \\ is called:} & \makecell{Result \\ is called:} \\
    \hline
    Addition & 16 + 2 = 18 & \makecell{Augend \\ (Summand)} & \makecell{Addend \\ (Summand)} & Sum \\
    \hline
    Subtraction & 16 - 2 = 14 & Minuend & Subtrahend & Difference \\
    \hline
    Multiplication & 16 · 2 = 32 & \makecell{Multiplicand \\ (Factor)} & \makecell{Multiplier \\ (Factor)} & Product \\
    \hline
    Division & 16 : 2 = 8 & Dividend & Divisor & Quotient \\
    \hline
    Exponentiation & 16² = 256 & Base & Exponent & Power \\
    \hline
    Radicalization & $\sqrt[2]{16}$ = 4 & Radicand & \makecell{Root \\ Exponent} & Root \\
    \hline
    Logarithmization & $\log_2 16$ = 4 & Logarithmand & \makecell{Logarithm \\ Base} & Logarithm \\
    \hline
\end{tabular}
}
}

\vspace{0.3cm}
How from each of the three direct operations addition, multiplication, exponentiation their two inversions follow, shows the following table:
\vspace{0.3cm}

{\centering
{\fontsize{10}{12}\selectfont
\renewcommand{\arraystretch}{1.5}
\begin{tabular}{|c|c|c|c|}
\hline
Degree: & Direct Operation: & Indirect Operation: & Sought is: \\
\hline
\multirow{2}{*}{I} & \multirow{2}{*}{\makecell{Addition: \\ 5 + 3 = 8}} & Subtraction: 8 - 3 = 5 & Augend \\
\cline{3-4}
& & Subtraction: 8 - 5 = 3 & Addend \\
\hline
\multirow{2}{*}{II} & \multirow{2}{*}{\makecell{Multiplication: \\ 5 · 3 = 15}} & Division: 15 : 3 = 5 & Multiplicand \\
\cline{3-4}
& & Division: 15 : 5 = 3 & Multiplier \\
\hline
\multirow{2}{*}{III} & \multirow{2}{*}{\makecell{Exponentiation: \\ 5³ = 125}} & Radicalization: $\sqrt[3]{125}$ = 5 & Base \\
\cline{3-4}
& & Logarithmization: $\log_5 125$ = 3 & Exponent \\
\hline
\end{tabular}
}
}

\vspace{0.3cm}

In the same way as multiplication arises from addition, exponentiation from multiplication, one could also derive from exponentiation as the direct operation of third degree a direct \textit{operation of fourth degree}\textsuperscript{26)}, from this one of fifth degree and so on derive.

\vfill
\leftline{\rule{2in}{0.4pt}}
\vspace{0.2cm}
{
\footnotesize
26) Of treatises that relate to operations of fourth or higher degree, mentioned here are: those by \textit{H. Gerlach} in Zeitschr. f. math. nat. Unterr. Vol. 13, Issue 6, by \textit{F. Wöpcke} in J. f. Mat. 42, by \textit{E. Schulze}
}
