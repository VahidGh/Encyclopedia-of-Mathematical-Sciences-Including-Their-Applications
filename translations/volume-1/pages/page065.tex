\thispagestyle{fancy}
\fancyhead[LO]{11. The Three Operations of Third Degree}

\vspace{0.5cm}

Yet already the definition of a direct operation of fourth degree, while logically justified, is unimportant because already at the third degree the commutation law loses its validity. To arrive at a direct operation of fourth degree, one has to consider $a^a$ as exponent of $a$, consider the power thus created again as exponent of $a$ and continue so until $a$ is set $b$ times. If one calls the result then $(a; b)$, then $(a; b)$ represents the result of the direct operation of fourth degree. For this holds e.g.: $(a; b)^{(a; c)} = (a; c+1)^{(a; b-1)}$.


\vspace{0.2cm}
\leftline{\rule{2in}{0.4pt}}
\vspace{0.2cm}
{
\footnotesize
in Arch. f. Math. (2) III (1886). \textit{G. Eisenstein} investigated through series expansions the function $x^{y^\frac{1}{y}}$ as inversion of $y = (x;\infty)$ in J. f. Math. Vol. 28. The textbooks by \textit{Hankel, Grassmann, H. Scheffler, E. Schröder, O. Schloemilch, Schubert} mention the direct operation of fourth degree without going into it in more detail.

}

\vspace{2cm}
\centerline{\rule{1in}{0.4pt}}
