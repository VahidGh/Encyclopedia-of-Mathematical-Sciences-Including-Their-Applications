\thispagestyle{fancy}
\fancyhead[LO]{Combinatorics; Historical Appreciation.   Combinatorial Operations; Definitions.}

\vspace{0.5cm}

\textbf{1. Combinatorics; Historical Appreciation.} Combinatorics has developed neither in its elementary nor in its higher analytical domains as was hoped for in an exuberant manner at the beginning of the century by representatives of the \textit{"combinatorial school"}. Beginnings of combinatorics can be traced far back; as a branch of science it may be considered only from \textit{Bl. Pascal}\textsuperscript{1)}, \textit{G. W. Leibnitz}\textsuperscript{2)}, \textit{J. Wallis}\textsuperscript{3)}, but especially from \textit{Jac. Bernoulli I.} and \textit{A. de Moivre}\textsuperscript{4)} onwards. The basic features of the elementary parts have passed into every textbook; the analytical applications recede very much. Thus the more comprehensive monographs all come from earlier times\textsuperscript{5)}, and more deeply penetrating treatises exist only in small number\textsuperscript{6)}.

\vspace{0.2cm}

\textbf{2. Combinatorial Operations. Definitions.} Of the infinitely many possible combinatorial operations, three have gained principal validity as equal (despite logical concerns): permutations (P.), combinations\textsuperscript{7)} (C.) and variations (V.). We call any arrangement of $n$ elements a complexion (Cp.) of the same. — P. of $n$ elements are called the Cp. which deliver all given elements in all possible sequences. If the elements are different from each other, then there are $n!$, if among them $a$ equal ones of one kind, $b$ equal ones of another kind etc. occur, then there are $n!:(a!\;b!\;...)$.

C. of $n$ elements to the $k$th class are all Cp. of $k$ each of those $n$ elements without consideration of the arrangement; if each element may be taken only once, then they are C. without repetition (w/o r.), otherwise with repetition (w/ r.). There are in $k$th class

\vspace{-0.1cm}
\begin{center}
    C. w/o r. \quad $ \frac{n!}{k!(n-k)!}$, \quad C. w/ r. \quad $\frac{(n+k-1)!}{k!(n-1)!}$ .
\end{center}
\vspace{-0.3cm}

\vspace{0.1cm}
\leftline{\rule{2in}{0.4pt}}
\vspace{0.1cm}
{
\footnotesize
1) \textit{Pascal}, Traité du triangle arithmétique. Paris 1665, written 1664 (Op. posth.).

2) \textit{Leibnitz}, Dissertatio de arte Combinatoria. Lipsiae 1668. Opp. II, T. I, p. 339.

3) \textit{Wallis}, Treatise of algebra. Lond. 1673 and 1685.

4) \textit{Bernoulli}, Ars conjectandi. Basil.1713 (Op. posth.). \textit{Moivre}, Probabilities. Lond.1718.

5) \textit{K. F. Hindenburg}, Nov. Syst. Permutationum etc. Lips. 1781. — \textit{J. Weingärtner}, Lehrb. d. combinator. Analysis. Leipz. 1800. — \textit{Knr. Stahl}, Grundrifs d. Kombin.-Lehre. Jena 1800. — \textit{Bernh. Thibaut}, Grundr. d. allgem. Arithm. od. Analysis. Götting. 1809. — \textit{Chr. Kramp}, Elem. d'Arithm. Cologne 1808. — \textit{Fr. W. Spehr}, Lehrbegr. d. rein. Kombin.-Lehre. Braunschw. 1824. — \textit{A. v. Ettingshausen}, D. kombinat. Analysis. Wien 1826. — \textit{L. Öttinger}, Lehre v. d. Kombinat. Freiburg 1837.

6) \textit{Hessel}, Arch. f. Math. 7 (1845), p. 395. — \textit{Öttinger}, ib. 15 (1850), p. 241.

7) \textit{Hindenburg} also writes "Komplexionen"; these break down into proper combinations, conternations, conquaternations etc.

}
