\thispagestyle{fancy}

\vspace{0.5cm}

V. arise when one permutes the elements in the C. There are in $k$th class

\begin{center}
    V. w/o r. \; $ \frac{n!}{(n-k)!}$, \quad \quad V. w/ r. \; $n^k$ .
\end{center}

\textbf{3. Inversion; Transposition.} Since the elements are valid only insofar as they are identical or different, their designation can be made arbitrarily e.g. through digits 1, 2, 3, ... or through letters $a$, $b$, $c$, ... . Thereby a new agent enters the consideration, which can now be used in different directions. A Cp. is called \textit{well-ordered} if always the higher digit (the alphabetically later letter) stands behind the lower (the earlier). Any deviation from this is called \textit{inversion}\textsuperscript{8)}. For counting the number of inversions in extensive Cp., \textit{P. Gordan} gives a rule\textsuperscript{9)}. A Cp. of different elements belongs to the \textit{first} or \textit{second} class (is \textit{even} or \textit{odd}), depending on whether it contains an even or odd number of inversions\textsuperscript{10)}. Through a \textit{transposition}, i.e. rearrangement of two elements, the class is changed\textsuperscript{11)}.

\vspace{0.3cm}

\textbf{4. Permutations with Restricted Position Occupation.} P. with \textit{restricted position} occupation are those where either a prescribed number of elements maintain their positions, or where certain positions may only be occupied by certain elements.

\textit{L. Euler}\textsuperscript{12)} introduces a function $f(n)$ which gives the number of P. where each element changes its original position. Connected with this is an $F(n,m)$ which indicates in how many P. of $n$ elements exactly $m$ keep their position\textsuperscript{13)}. It is:

\vfill
\leftline{\rule{2in}{0.4pt}}
\vspace{0.2cm}
{
\footnotesize
8) \textit{G. Cramer}, Introduct. à l'Analyse des lignes courbes (1750); Genève. Appendice p. 658. — \textit{T. P. Kirkman}, Cambr. a. Dubl. J. 2 (1847), p. 191.

9) Vorles. üb. Invarianten-Theor., herausgeg. v. \textit{G. Kerschensteiner} (1885), Leipz. I, p. 2.

10) \textit{E. Bezout}, Mém. Paris (1764), p. 292. — \textit{A. L. Cauchy}, J. d. l'Éc. pol. cah. 10 (1815), p. 41. — \textit{K. G. Jacobi}, Werke 3, p. 359 = J. f. M. 22 (1841), p. 285.

11) \textit{P. S. Laplace}, Mém. Paris (1772), p. 294.

12) Mém. Pétersb. 3 (1809), p. 57. — \textit{Öttinger}, Lehre v. d. Kombin. Freiburg 1837. — \textit{M. Cantor}, Z. f. Math. 2 (1857), p. 65. — \textit{R. Baltzer}, Leipz. Ber. (1873), p. 534. — \textit{S. Kantor}, Z. f. Math. 15 (1870), p. 361. — \textit{A. Cayley}, Edinb. Proceed. 9 (1878), p. 338 u. 388.

13) \textit{M. Cantor}, Z. f. Math. 2 (1857), p. 410. — \textit{J. J. Weyrauch}, J. f. M. 74 (1872), p. 273

}
