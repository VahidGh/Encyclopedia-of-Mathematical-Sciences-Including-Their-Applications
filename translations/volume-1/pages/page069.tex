\thispagestyle{fancy}
\fancyhead[LO]{Related Permutations.   Sequences.}

\vspace{0.5cm}

\begin{center}
    $f(n) = nf(n-1) + (-1)^n = (n-1)[f(n-1)+f(n-2)]$ ; 
    
    $f(0)=1$, $f(1)=0$ .
    
    $F(n,m) = \binom{n}{m}f(n-m)$ .
\end{center}

Another restriction of position occupation is that certain positions may only be occupied by certain elements\textsuperscript{14)}, e.g. the even positions only by even numbers\textsuperscript{15)}; or, if equal elements occur, that not two such follow each other\textsuperscript{16)}.

Restrictions of position occupation also lie therein that the P. themselves enter into different arrangements, e.g. such that $n$ P. of $n$ elements should be placed under each other so that in no column do equal elements occur\textsuperscript{17)} (Latin square).

\vspace{0.1cm}

\textbf{5. Related Permutations.} \textit{Related permutations} according to \textit{H. A. Rothe}\textsuperscript{18)} are two P. when the position order and the position element (as number) of one are exchanged against those of the other; it will be to determine how many P. are related to themselves\textsuperscript{19)}. \textit{P. A. Mac-Mahon} gives extensions of these concepts\textsuperscript{19)}.

\vspace{0.1cm}

\textbf{6. Sequences.} \textit{D. André} has introduced the concept of \textit{sequence} for P. and investigated it in a whole series of treatises\textsuperscript{20)}. Consecutive number elements of a P. form a sequence if each following is larger (smaller) than the preceding. Every P. breaks down into sequences. The number of occurring sequences determines the "type" of the P. It is investigated how many P. belong to a certain type. It shows that the number of P. with even sequence number equals that of P. with odd sequence number. Geometric representations are given, etc.

\vspace{-0.1cm}
\leftline{\rule{2in}{0.4pt}}
\vspace{0.1cm}
{
\footnotesize
14) \textit{C. W. Baur}, Z. f. Math. 2 (1857), p. 267.

15) \textit{A. Laisant}, C. R. 112 (1891), p. 1047.

16) \textit{O. Terquem}, J. d. Math. 4 (1839), p. 177. — Further restrictions in position occupation investigated by \textit{Th. Muir}, Edinb. Proceed. 10 (1881), p. 187. \textit{A. Holtze}, Arch. f. Math. (2), 11 (1892), p. 284.

17) \textit{A. Cayley}, Messenger (2), 19 (1890), p. 135. \textit{M. Frolov}, J. m. spec. (3), 4 (1890), p. 8 u. 25. \textit{J. Bourget}, J. de Math. (3), 8 (1882), p. 413. \textit{P. Seelhof}, Arch. f. Math. (2), 1 (1884), p. 97.

18) Hindenb. Arch. f. M. (1795).

19) \textit{P. A. Mac Mahon}, Messeng. (2), 24 (1894), p. 69.

20) C. R. 97 (1883), p. 1356; 115 (1892), p. 872; 118 (1894), p. 575. [\textit{G. Darboux} Rapport; C. R. 118 (1894), p. 1026]. Soc. m. d. Fr. 21 (1893), p. 131; Ann. Éc. Norm. (3), 1 (1884), p. 121.

}
