\thispagestyle{fancy}

\vspace{0.5cm}

\textbf{7. Application to Questions of Arithmetic.} Connected with the P.-theorems are the questions of in how many ways one can arithmetically carry out sums or products of given elements, which are different or partly equal to each other, with or without rearrangement of these elements\textsuperscript{21)}.

\vspace{0.2cm}

\textbf{8. Combinations to Specific Sum or Specific Product.} Also with C. and V. the individual Cp. can be subject to restrictions. The most well-known and important consists in that the C. and V. of natural numbers are considered whose elements w/ r. or w/o r. have a \textit{specific sum}, which is called the \textit{weight} of the Cp. Their significance appears in invariant theory. \textit{L. Euler} was the first who treated this question (Introd. in Anal. Lausanne 1748, § 299 ff.; Comm. Acad. Petr. 3 [1753], p. 159), who gave development coefficients of certain products for the number of these C. and thereby arrived at relations between C. w/ r. and C. w/o r. These questions were later pursued further in many ways\textsuperscript{22)}, and their answering particularly promoted by \textit{Cayley}\textsuperscript{23)} and \textit{J. J. Sylvester}\textsuperscript{24)} through establishment of tables and geometric representations. \textit{Mac-Mahon} has carried these investigations further, which were then also extended to the decomposition of number pairs. We must content ourselves with these remarks, as the further applications to symmetric functions and invariant theory are no longer of combinatorial nature.

In similar manner \textit{Möbius} has treated the C. where the elements of the Cp. have a \textit{specific product}\textsuperscript{25)}. They are ordered according to their classes, and the numbers of associated C. brought into connection with each other. Such relations also appear for the case that conditions are imposed on the Cp., e.g. that with the prescribed product $a^\alpha b^\beta$ in each Cp. each element has at least one factor $b$.

\textit{Ettingshausen} has furthermore gone into treating each Cp. as a product

\vspace{-0.2cm}
\leftline{\rule{2in}{0.4pt}}
\vspace{0.1cm}
{
\footnotesize
21) \textit{E. Ch. Catalan}, J. d. Math. 6 (1874), p. 74. \textit{E. Schröder}, Z. f. Math. 15 (1870), p. 361.

22) \textit{M. Stern}, J. f. M. 21 (1840), p. 91 u. p. 177, ibid. 95 (1883), p. 102. \textit{C. Wasmund}, Arch. f. Math. 21 (1853), p. 228, ibid. 34 (1860), p. 440.

23) Lond. Transact. 145 (1855), p. 127, ibid. 148 (1858), p. 47. Amer. J. 6 (1883), p. 63.

24) Quart. J. 1 (1855—57), p. 81 u. p. 141. Amer. J. 5 (1882), p. 251. C. R. 96 (1883). Vgl. auch \textit{Mac Mahon}, Lond. Trans. 184 (1894), p. 835, sowie den Bericht über Combin. Analysis: Lond. M. S. Pr. 28 (1897), p. 5.

25) J. f. M. 9 (1832); 105.

}
