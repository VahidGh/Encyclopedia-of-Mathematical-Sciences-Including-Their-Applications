\thispagestyle{fancy}
\fancyhead[LO]{9. Combinations with Restricted Position Occupation.   10. Triple Systems.}

\vspace{0.5cm}

to consider, and to sum all such products belonging to a C.-class\textsuperscript{26)}. Further the classes are divided according to specific moduli and numerical relations between them are determined\textsuperscript{27)}.

And not only to the C. themselves do such investigations relate, but also to Cp. that are derived in various ways from the ordinary C. For example, to the first element of each Cp. 0 is added, to the second 1, ... to the $n$th $(n-1)$. Thus products arise between whose sums again remarkable relations can be specified\textsuperscript{28)}. Cf. also \textit{Th. B. Sprague}, Edinb. Proc. 37 (1893), p. 399.

\vspace{0.1cm}

\textbf{9. Combinations with Restricted Position Occupation.} The path of investigation which relates to \textit{restricted position occupation} branches here. First, similar to P., requirements are made that certain elements occur in a prescribed number of times\textsuperscript{29)}, or that a maximum number for their occurrence is given\textsuperscript{30)}.

\vspace{0.1cm}

\textbf{10. Triple Systems.} Another direction has proved particularly important for geometry, probability calculation, for algebra. Independent of each other, \textit{T. P. Kirkman}\textsuperscript{31)} and \textit{J. Steiner}\textsuperscript{32)} posed almost identical tasks; the First his "schoolgirl problem": Fifteen girls are taken out 35 times in rows of 3, so that not 2 go together twice; the Last the following: From $N$ elements C. of the 3rd class (triples) should be selected so that each pair occurs once and only once; further C. of the 4th class (quadruples) so that in them each triple that did not occur among the previous ones occurs once and only once etc. \textit{Cayley}\textsuperscript{33)} and \textit{R. R. Anstice}\textsuperscript{34)} treated individual cases of the \textit{"triple systems"}. A general rule for the formation of such systems, which require $N=6n+1$, $6n+3$, was given by \textit{M. Reiss}\textsuperscript{35)}.

\vspace{-0.1cm}
\leftline{\rule{2in}{0.4pt}}
\vspace{0.1cm}
{
\footnotesize
26) Die kombinatorische Analysis. Wien (1826).

27) \textit{A. A. Cournot}, Bull. sci. m. (1829). \textit{Ch. Ramus}, J. f. M. 11 (1834), p. 353.

28) \textit{H. F. Scherk}, J. f. M. 3 (1828), p. 96; J. f. M. 4 (1829), p. 226.

29) \textit{Ad. Weiss}, J. f. M. 34 (1847), p. 255.

30) \textit{Öttinger}, Arch. f. M. 15 (1850), p. 241. \textit{Baur}, Z. f. M. 2 (1857), p. 267. \textit{Scherk}, Math. Abhandl. Berlin (1825), p. 67. \textit{Andre}, Ann. Éc. norm. (2), 5 (1876), p. 155.

31) Cambr. a. Dubl. m. J. 7 (1852), p. 527 u. 8 (1853), p. 38; vgl. \textit{T. Clausen}, Arch. f. M. 21 (1853), p. 93.

32) J. f. M. 45 (1853), p. 181 — Werke II, p. 435.

33) Phil. Mag. (3), 37 (1850), p. 50. — Ibid. (4), 25 (1862), p. 59.

34) Cambr. a. Dubl. m. J. 7 (1852), p. 279 u. 8 (1853), p. 149.

35) J. f. M. 56 (1859), p. 326.

}
