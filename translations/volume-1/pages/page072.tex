\thispagestyle{fancy}

\vspace{0.5cm}

In recent times, analytical representations were derived for prime numbers $6n+1$, further for numbers $6n+3$, if this is three times a prime number of the form $6n+5$, etc. Finally, analytical formation rules were given from which the construction follows for every possible $N$\textsuperscript{36)}. For $N=13$ two triple systems are known\textsuperscript{37)}. The further parts of \textit{Steiner}'s task have not yet been tackled. — \textit{Cayley} draws attention to similar tasks\textsuperscript{38)}. Cf. IA6 No. 13 Note 67.

\vspace{0.3cm}

\textbf{11. Extension of the Concept of Variation.} The concept of V. has been extended in the direction that $m$ rows of $n$ elements each are given, and as V. $m$th class are then designated the Cp. which contain one element from each of the $m$ rows. If the same position index of the elements may occur only once, then they are V. w/o r., otherwise V. w/ r.

\vspace{0.3cm}

\textbf{12. Formulas.} Between the various numbers discussed so far for P., C. and V. there exists an extraordinarily large number of connecting formulas. Here it must suffice to point to the main writers who have dealt with the derivation or compilation\textsuperscript{39)}.

\vspace{0.3cm}

\textbf{13. Binomial Coefficients.} We have already mentioned that the proofs of the \textit{binomial} and the \textit{polynomial} theorem for whole positive exponents $n$

\begin{center}
    $(z_1+z_2+...+z_Q)^n = \sum \frac{n!}{x_1!x_2!...x_Q!} z_1^{x_1}z_2^{x_2}...z_Q^{x_Q}$
    
    $(x_1+x_2+...+x_Q=n)$
\end{center}

are applications of combinatorial formulas. The binomial formula is found first in \textit{H. Briggs}\textsuperscript{40)}, then in \textit{J. Newton}\textsuperscript{41)};

\vfill
\leftline{\rule{2in}{0.4pt}}
\vspace{0.2cm}
{
\footnotesize
36) \textit{E. Netto}, Substit.-Theorie § 192ff. Leipz. (1882). Math. Ann. 42 (1892), p. 143. \textit{E. H. Moore}, Math. Ann. 43 (1893), p. 271; N. Y. Bull. (2), 4 (1897), p. 11. \textit{L. Heffter}, Math. Ann. 49 (1897), p. 101. \textit{J. de Vries}, Rend. Palermo 8 (1894).

37) \textit{K. Zulauf}, Dissert. Giessen (1897).

38) Phil. Mag. 30 (1865), p. 370.

39) \textit{Hindenburg}, Nov. Syst. Permutationum, Combin. etc. primae lineae. Lips. (1781). — D. polynom. Lehrs., d. wichtigste Theorem d. ganzen Analysis, neu bearb. v. \textit{J. N. Tetens}, \textit{G. S. Klügel}, \textit{A. Krauss}, \textit{J. F. Pfaff} u. \textit{Hindenburg}, herausgeg. v. \textit{Hindenburg}. Leipz. 1796. \textit{Hindenburg}, Infinitonomii dignitatum historia, leges etc. Vgl. auch \textit{J. A. Grunert}, Arch. f. M. 1 (1841), p. 67; \textit{Brianchon}, J. d. l'Éc. Polyt. t. 15 (1837), p. 159.

40) Arithmetica Logarithmica. London (1620).

41) Briefe an \textit{Oldenburg} (1676) 13. Juni u. 24. Oktober.

}
