\thispagestyle{fancy}
\fancyhead[LO]{13. Binomial Coefficients.   14. Applications.}

\vspace{0.5cm}

The coefficients of the binomial expansion ($n=2$), the \textit{binomial coefficients} in their arrangement etc. as "arithmetic triangle"

\vspace{-0.3cm}
\begin{center}
    $1$\\[-1pt]
    
    $1 \quad 1$\\[-1pt]
    
    $1 \quad 2 \quad 1$\\[-1pt]
    
    $1 \quad 3 \quad 3 \quad 1$\\[-1pt]
    
    $. \quad . \quad . \quad . \quad .$\\[-1pt]
\end{center}
\vspace{-0.1cm}

already appear in \textit{Bl. Pascal}\textsuperscript{42)}.

As extensions of the binomial theorem are to be mentioned, first the expansion of $a(a+b)(a+2b)...(a+nb)$ according to powers of $a$;\textsuperscript{43)} further the expansion

\vspace{-0.5cm}
\begin{center}
    $(x+a)^n = x^n + c_1(x+t_1)^{n-1} + c_2(x+t_1+t_2)^{n-2} + ...$ , 
\end{center}
\vspace{-0.2cm}

where the $t_\alpha$ are arbitrary quantities\textsuperscript{44)}. 

As extension of the binomial coefficients, expressions of the form 

\vspace{-0.1cm}
\begin{center}
    $[n(n+k)(n+2k)...(n+(p-1)k)]:p!$ 
\end{center}
\vspace{-0.1cm}

have been introduced\textsuperscript{45)}, whose numerators as faculties have been thoroughly investigated\textsuperscript{46)}. The analytical treatment does not belong here. 

Between the binomial coefficients there exists an innumerable number of relations, whose classification has been initiated by \textit{J. G. Hagen}\textsuperscript{47)}. Cf. also the "figurate numbers" of the ancients.

\vspace{0.1cm}

\textbf{14. Applications.} As already mentioned in No. 1, most applications of analytical nature of combinatorics offer only historical interest anymore. We limit ourselves to indicating the most important branches which combinatorics had undertaken to support. In first place belongs here probability calculation, in whose elementary parts combinatorial questions occur continuously, and from which conversely combinatorics has received many stimulations. 

\vspace{-0.1cm}
\leftline{\rule{2in}{0.4pt}}
\vspace{0.1cm}
{
\footnotesize
42) Traité du triangle arithmet. Paris (1665) posth.; and earlier in \textit{M. Stifel}, Arithm. integra. Norimb. (1544), p. 44.

43) \textit{Pascal} "productum continuorum".

44) \textit{N. H. Abel}, J. f. M. 1 (1826), p. 159 gives a special case; generally \textit{A. v. Burg}, J. f. M. 1 (1826), p. 367. — \textit{Cayley}, Phil. Mag. 6 (1853), p. 185 = Werke II, 102.

45) \textit{Bl. Pascal}, see above.

46) \textit{L. Euler}, Calc. diff. II. c. 16 u. 17. Berl. (1755). \textit{Öttinger}, J. f. M. 33 (1846), p. 1, 117, 226, 329; further 35 (1847), p. 13 u. 38 (1849), p. 162, 216; finally 44 (1852), p. 26 u. 147, where historical information is also listed.

47) Synopsis. Berlin (1891), p. 64ff. Cf. also \textit{G. Eisenstein}, Brief an \textit{M. A. Stern}, Z. f. Math. 40 (1895), p. 198 of the hist. section.

}
