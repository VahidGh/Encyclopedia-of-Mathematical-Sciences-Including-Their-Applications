\thispagestyle{fancy}

\vspace{0.5cm}

It furthermore connects to the theory of series, formally determines the products, powers, quotients of series; the result of substituting a series for the variable $z$ in a series that progresses according to powers of $z$; the formal inversion of series; the rationalization of such in which irrationalities enter; the general terms of recurring series; the logarithms of series and series of logarithms etc. Similarly it gives the form for the higher differentials of more complicated functions etc. For its purposes it had devised a complete notation system, which now is certainly entirely outdated\textsuperscript{48)}.

The entire theory of finite discrete groups (IA 6) can be directly connected to combinatorics.

Yet a second application, directed at solving linear equations, has developed in a surprising way. It has become the \textit{theory of determinants}.

\vspace{0.3cm}

\textbf{15. Determinants. Definition of the Concept.} Let $n^2$ quantities $a_{ik}$ ($i,k=1,2,...n$) be given; form all $n!$ products $a_{1i_1}a_{2i_2}...a_{ni_n}$ in which $i_1,i_2,...i_n$ means a P. of $1,2,...n$ and give each the sign + or -, depending on whether this P. belongs to the first or second class. The sum of these $n!$ summands is the determinant of $n$th degree\textsuperscript{49)}. \textit{A. L. Cauchy}\textsuperscript{50)} defines it also such that he develops the alternating product $\prod(a_i-a_k)$, ($i=1,2,...n$; $k=i+1,...n$), and writes the exponents as second lower indices. \textit{E. Schering}\textsuperscript{51)} gives a geometric and an analytical explanation, \textit{Kronecker} laid a function-theoretical one as foundation in his lectures.

The most common notations are\textsuperscript{52)}

\begin{center}
$\begin{vmatrix} 
a_{11} & a_{12} & ... & a_{1n}\\
... & ... & ... & ... \\
a_{n1} & a_{n2} & ... & a_{nn}
\end{vmatrix}$ = $\sum \pm a_{11} a_{22} ... a_{nn}= \|a_{h1} a_{h2} ... a_{hn}\| = \|a_{hk}\|$ ;

$(h, k=1,2,...n)$ .
\end{center}

\vfill
\leftline{\rule{2in}{0.4pt}}
\vspace{0.2cm}
{
\footnotesize
48) Cf. \textit{Hindenburg}, Nov. Syst. etc. Leipz. (1781).

49) \textit{Jacobi}, J. f. M. 22 (1841), p. 285 = Werke III, p. 355.

50) Analyse algébrique. Paris (1821).

51) Gött. Abh. 22 (1879), p. 102.

52) The third notation is often used by \textit{L. Kronecker}; the last first by \textit{St. Smith}, Brit. Ass. Rep. (1862) p. 504. As newly introduced \textit{L. Kronecker} then has it, J. f. M. 68 (1868), p. 273. On further notations cf. \textit{Cayley}, Phil. Mag. 21 (1861), p. 180. \textit{Nanson}, Lond. phil. Mag. (5) 44 (1897), p. 396. \textit{W. Schrader}, Determinanten. Halle 1887.

}
