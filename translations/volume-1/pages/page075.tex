\thispagestyle{fancy}
\fancyhead[LO]{15. Determinants. Definition of the Concept. 16. Definitions.}

\vspace{0.5cm}

Historically it is to be noted that determinants were invented by \textit{Leibnitz}\textsuperscript{53)} and later independently by \textit{Cramer}\textsuperscript{54)} and were initially used for solving a system of linear equations. The first detailed theoretical expositions come from \textit{J. Binet} and \textit{Cauchy}; the theory of D. was generally introduced by \textit{Jacobi}\textsuperscript{55)}. Extensive literature references can be found in \textit{Muir}\textsuperscript{56)} continued from the beginning of the theory until 1885; Historical information also in \textit{S. Günther}, Determinantentheorie, Erlangen (1875), in \textit{Baltzer}, Determinanten, Leipzig (1881) and in \textit{G. Salmon}, Modern higher algebra, Note I in \textit{Baltzer}.

\vspace{0.1cm}

\textbf{16. Definitions.} The quantities $a_{ik}$ are called the \textit{elements} (El.) of the D.; the first (second) index gives the ordinal number of the row (column). The quantities $a_{ii}$ form the \textit{main diagonal}; the $a_{i,n+1-i}$ form the \textit{secondary diagonal}. The term $a_{11}a_{22}...a_{nn}$ of D. is called its \textit{principal term}. If one selects $m$ values of the first and $m$ of the second index from $1,2,...n$, the corresponding El. form a \textit{subdeterminant} (Subd.) of $m$th degree\textsuperscript{57)}. If the El. of its main diagonal are also El. of that of D., then the Subd. is called a \textit{principal subdeterminant}. If the product of the principal terms of two Subd. is a term of D., then the Subd. are called \textit{adjunct}, or also \textit{complementary} Subd.

If $a_{ik}=a_{ki}$, then D. is called a \textit{symmetric} D.

If $a_{ik}=a_{i+k-2}$, then D. is called a \textit{recurrent} (one-sided, \textit{orthosymmetric}) D. It is symmetric\textsuperscript{58)}.

If $a_{ik}=a_{i+1,k+1}$, where the indices are reduced mod $n$, then D. is called a \textit{circulant}\textsuperscript{59)}, (also \textit{negative-orthosymmetric} D.).

If $a_{ik}+a_{ki}=0$, $a_{ii}=0$, then D. is called a \textit{half-symmetric}. If $a_{ik}+a_{ki}=0$, for $i \neq k$, then D. is called a \textit{skew}\textsuperscript{60)}.

\vspace{-0.3cm}
\leftline{\rule{2in}{0.4pt}}
\vspace{0.1cm}
{
\footnotesize
53) Lettres à l'Hospital (1693). — Acta Erudit. Leipz. (1700), p. 206.

54) Introd. à l'anal. des courbes algebr. (1750). Genève. Appendice p. 656.

55) J. de l'Éc. Polytechn. Cah. 16 (1812), p. 280 u. Cah. 17 (1812), p. 29. — \textit{Jacobi}, J. f. M. 22 (1841), p. 285 = Werke III, p. 355.

56) Quart. J. 18 (1882), p. 110; ibid. 21 (1886), p. 299. Edinb. Proc. 13 (1886), p. 547. In Phil. Mag. (5), 18 (1884), p. 416 \textit{Muir} draws attention to \textit{Ferd. Schweins} as a forgotten discoverer, "Theorie der Differenzen u. Differentiale" (1825). Heidelberg. Cap. IV, p. 317.

57) Also called Unterdeterminante, Partialdeterminante, Minor.

58) \textit{H. Hankel}, Dissert. Leipz. (1861) Göttingen. — "Recurrierend" according to \textit{G. Frobenius}; Berl. Ber. (1894), p. 253.

59) \textit{Th. Muir}, Quart. J. (1882), p. 166. \textit{Hankel} l. c.

60) \textit{Jacobi}, J.f.M.2(1857), 354; ibid.29(1845), p.236. \textit{Cayley}, J.f.M.38(1849), p.93; ibid.32(1846), p.119; ibid.50(1855), p.299. \textit{Cayley} designates the half-symmetric D. as "skew-symmetric".

}
