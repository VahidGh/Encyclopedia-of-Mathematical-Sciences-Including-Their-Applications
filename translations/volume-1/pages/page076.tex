\thispagestyle{fancy}

\vspace{0.5cm}

If $a_{ik}=a_{n+1-i,n+1-k}$, then D. is called a \textit{centrosymmetric}.

\vspace{0.2cm}

\textbf{17. Number Problems Regarding the Terms.} The number of terms of a D. of $n$th degree is $n!$. Further questions connect to this: how many of the terms contain a prescribed number of El. of the main diagonal\textsuperscript{61)}? How many terms has a D. whose main diagonal contains $k$ El. 0 \textsuperscript{62)}? How many different terms are there in symmetric, how many in half-symmetric D.\textsuperscript{63)}?

\vspace{0.2cm}

\textbf{18. Elementary Properties.} The following properties of elementary nature show immediately: one can, without changing the value of D., make every $\alpha$th R. into the $\alpha$th C.\textsuperscript{64)}. When transposing two parallel rows, the sign of D. changes; consequently a D. with two identical parallel rows equals zero\textsuperscript{65)}. The D. can be represented as a linear, homogeneous function of the El. of each row\textsuperscript{66)}. From this follows that one can pull out a common factor of all El. of a row before D. The degree of a D. can be increased by suitable \textit{bordering}, i.e. addition of new R. and C. If two D. agree in $(n-1)$ rows, then they can be summed to a D. with the same $(n-1)$ rows. If $a_{ik}=b_{ik}+c_{ik}$ ($k=1,...n$), then conversely the D. breaks down into individual summands. The linear, homogeneous representation delivers the partial derivative of D. with respect to $a_{ik}$. If we denote it with $a'_{ki}$, then follows $\sum_{\lambda} a_{i\lambda}a'_{\lambda k} = \bar{c}_{ik}D$ (i.e. $=D$ if $i=k$, otherwise $=0$)\textsuperscript{67)}. Like the $a'$, so can also the higher Subd. be represented as partial derivatives of higher order\textsuperscript{68)}.

The D. does not change its value when to a row a parallel row is added or from it subtracted\textsuperscript{69)}.

\vfill
\leftline{\rule{2in}{0.4pt}}
\vspace{0.2cm}
{
\footnotesize
61) \textit{Baltzer}, Determin. 4. Aufl. Leipz. 1875, p. 39. Leipz. Ber. (1873), p. 534. \textit{C. J. Monro}, Messeng. (2), 2 (1872), p. 38.

62) \textit{N. v. Szütz}, Math. Ann. 33 (1889), p. 477.

63) \textit{J. J. Weyrauch}, J. f. M. 74 (1872), p. 273. \textit{Cayley}, Monthly Not. of Astron. Soc. 34 (1873—74), p. 303 u. p.335. \textit{G. Salmon}, Modern Algebra. Dublin (1885), p. 45.

64) \textit{J. C. Becker}, Z. f. M. 16 (1871), p. 326. \textit{Gordan}, Vorles. üb. Invar.-Th. (1885), p. 21. — The D. becomes "turned".

65) \textit{Ch. A. Vandermonde}, Par. Acad. (1772), 2° part., p. 518, 522.

66) \textit{Cramer}, l.c. \textit{J. L. Lagrange}, Berl. Mem. (1773), p. 149, 153.

67) $\epsilon_{ik}$ introduced by \textit{Kronecker}, J. f. Math. 68 (1868), p. 273. If one sets $a'_{ik}/D=\alpha_{ik}$, then \textit{Kronecker} calls the systems $a_{ik}$, $\alpha_{ik}$ reciprocal systems.

68) \textit{Jacobi}, J. f. Math. 22 (1841), p. 285, §10 = Werke III, p. 365.

69) \textit{Jacobi}, J. f. Math. 22 (1841), p. 371 = Werke III, p. 452.

}
