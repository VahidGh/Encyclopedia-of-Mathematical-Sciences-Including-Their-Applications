\thispagestyle{fancy}
\fancyhead[LO]{19. Laplace's and Other Decomposition Theorems. 20. Developments.}

\vspace{0.5cm}

With help of the stated theorems, the calculation of D. with numerical El. can often be shortened, just as that of D. whose El. follow analytical laws. There exists an almost innumerable number of results; highlighting even just the most important would exceed the framework of this presentation\textsuperscript{70)}.

\vspace{0.2cm}

\textbf{19. Laplace's and Other Decomposition Theorems.} From \textit{P. S. Laplace}\textsuperscript{71)} comes an important theorem about the development of D. according to products of adjunct Subd. From the $m$ first R. (C.) all possible Subd. of $m$th degree are formed, from the following R. (C.) all adjunct ones of $(n-m)$th degree. To the product of two adjunct ones such a sign is given that the product of their principal terms is a term of D. The sum of these products equals D. If one takes arbitrary $m$ and $(n-m)$ R. (C.), then the sum $=0$ if even just one common row occurs\textsuperscript{72)}. \textit{Jacobi} draws from this a series of conclusions about D. with zero elements\textsuperscript{73)}.

Very obvious is the extension of the theorem in the direction that the products consist of more than two factors\textsuperscript{74)}.

Another extension uses the bordering of D. and indicates how from each result delivered by \textit{Laplace}'s formula a new one about bordered D. can be derived\textsuperscript{75)}. To this extension another concerning adj. Subd. stands alongside\textsuperscript{76)}.

\vspace{0.2cm}

\textbf{20. Developments.} Of further developments would still be to mention that of a D. where the diagonal terms $a_{ii}=b_{ii}+z$ are called; the development happens according to powers of $z$\textsuperscript{77)}. Also the development of a single-row bordered D. of $(n+1)$th degree according to the El. of the border is important\textsuperscript{78)}.

\vspace{-0.1cm}
\leftline{\rule{2in}{0.4pt}}
\vspace{0.1cm}
{
\footnotesize
70) Cf. the examples in \textit{Baltzer}, \textit{S. Günther}, \textit{R. F. Scott}, \textit{Salmon} etc.

71) Recherches sur le calcul integral et sur le systeme du monde. Paris Ac. d. Sc. (1772) 2° part., p. 267. — \textit{Cauchy}, l. c. p. 100. — \textit{Jacobi}, l. c. Nr. 5.

72) \textit{Cauchy}, l. c.

73) \textit{Jacobi}, l. c. Nr. 5.

74) \textit{Vandermonde}, l. c. p. 524. — \textit{Laplace}, l. c. p. 294. — \textit{Jacobi}, l. c. Nr. 8.

75) \textit{Netto}, J. f. Math. 114 (1895), p. 345.

76) \textit{E. Pascal}, Rend. Acc. d. Linc. (5) 5, (1896), p. 188. The theorem established there follows by the way from the previous one by means of a general theorem by \textit{Th. Muir}, Edinb. Transact. 30 (1882), p. 1, through which one can transition from a formula about Subd. to another about adjunct Subd.

77) \textit{Laplace}, Mécan. celeste, 1, liv. 2, Nr. 56. Paris (1799). — \textit{Jacobi}, J. f. Math. 12 (1834), p. 15 = Werke III, p. 208.

78) \textit{Cauchy}, l. c. p. 69.

}
From \textit{O. Hesse} comes