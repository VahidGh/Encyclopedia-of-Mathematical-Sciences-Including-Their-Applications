\thispagestyle{fancy}

\vspace{0.5cm}

From \textit{O. Hesse} comes a theorem about decomposition of the bordered D. if the unbordered vanishes\textsuperscript{79)}.

\vspace{0.2cm}

\textbf{21. Composition and Product.} The product of a D. of $m$th into a D. of $n$th degree can be easily represented as D. of $(m+n)$th degree by pushing together in diagonal direction (textit{Laplace}'s theorem). \textit{J. Ph. M. Binet} and \textit{A. L. Cauchy} have represented the product of two D. of $n$th degree again as D. of $n$th degree\textsuperscript{80)}. Simultaneously they have given the following extension: From two systems $a_{ik}$, $b_{ik}$ a third $c_{ik}$ is formed, textit{composed},

\vspace{-0.1cm}
\begin{center}
    $a_{ik}$ $(i=1,...m; k=1,...n)$
    $b_{ik}$ $(i=1,...n; k=1,...m)$
    
    $c_{ik} = \sum a_{i\lambda}b_{\lambda k}$ $(i=1,...m; k=1,...m; \lambda=1,...n)$;
\end{center}
\vspace{-0.1cm}

then is $|c_{ik}|=0$ for $m>n$; further $|c_{ik}| = |a_{ik}||b_{ik}|$ for $m=n$; and finally $|c_{ik}| = \sum_t |a_{it}||b_{it}|$ for $m<n$, where $t$ runs through all possible $r$-combinations of $m$th class from $1,2,...n$. The middle case gives the multiplication rule\textsuperscript{81)}; the different arrangement of El. in R. and C. delivers four different forms for the product\textsuperscript{82)}. To this representation connect analytically and number-theoretically important formulas\textsuperscript{83)}.

\vspace{0.2cm}

\textbf{22. Other Kind of Composition.} \textit{Kronecker}\textsuperscript{84)} has drawn attention to another kind of composition: $a_{ik}$ $(i,k=1,...m)$ and $b_{gh}$ $(g,h=1,...n)$ are composed to $c_{pq} = a_{ik}b_{gh}$ $(p=(i-1)n+g; q=(k-1)n+h; i,k=1,...m; g,h=1,...n)$. Then is 

\vspace{-0.7cm}
\begin{center}
$|c_{pq}| = |a_{ik}|^n \cdot |b_{gh}|^m$ .
\end{center}
\vspace{-0.1cm}

\textbf{23. Compound Determinants.} Detailed interest has turned to the question of \textit{compound} D. (compound det.), i.e. to such whose elements are themselves D. formed according to certain laws. Most obvious is the investigation of the D. formed from the El. $a'_{ik}$, i.e. the adjuncts of $a_{ik}$. \textit{Cauchy}\textsuperscript{85)} has for $|a'_{ik}|$ $(i,k=1,...n)$ indicated the given value;

\vspace{-0.1cm}
\leftline{\rule{2in}{0.4pt}}
\vspace{0.1cm}
{
\footnotesize
79) J. f. Math. 69 (1868), p. 319.

80) J. de l'Éc. polyt. Cah. 16 (1812), p. 280; Cah. 17 (1812), p. 29.

81) Further proofs among others: \textit{J. König}, Math. Ann. 14 (1879), p. 507. \textit{M. Falk}, Brit. Ass. Rep. (1878), p. 473. \textit{A. V. Jamet}, Nouv. Corresp. M. 3 (1877), p. 247.

82) \textit{Cauchy}, l. c. p. 83.

83) \textit{Ch. Hermite}, J. f. Math. 40 (1850), p. 297. \textit{K. F. Gauss} Werke 3, p. 384. \textit{Baltzer}, Leipz. Ber. (1873), p. 352. \textit{S. Gundelfinger}, Z. f. Math. 18 (1873), p. 312.

84) Vorlesungen. \textit{K. Hensel}, Acta mat.14 (1890—91), p.317. \textit{Netto}, Acta mat.17 (1894), p.200. \textit{B. Igel}, Monatsh. f. Math.3 (1892), p.55. \textit{G. v. Escherich}, ib.3 (1892), p.68.

85) l. c. p. 82.

}
