\thispagestyle{fancy}
\fancyhead[LO]{24. Rank of the Determinant.}

\vspace{0.5cm}

\textit{Jacobi}\textsuperscript{86)} more generally for $|a'_{ik}|$ $(i,k=1,2,...m; m<n)$. In the first case a power of D. appears, in the second one such, multiplied with a Subd. $|a_{ik}|$.

These theorems have been extended by \textit{Franke}\textsuperscript{87)}; instead of the $a'_{ik}$, the Subd. of $m$th degree $p_{ik}$ $(i,k=1,2,...\mu)$ are considered, where $\mu=\binom{n}{m}$, and the numbering extends to all $\mu$ Subd. of $m$th degree of D. Further shall $p'_{ik}$ be adjunct to $p_{ik}$, i.e. $p'_{ik}$ is a Subd. of $(n-m)$th degree of $|a_{ik}|$, and the product of the principal terms of $p_{ik}$ and $p'_{ik}$ is a term of $|a_{ik}|$. It results then 

\vspace{-0.1cm}
\begin{center}
    $|p_{ik}| = D^{\binom{n-1}{m-1}}$, \quad $|p'_{ik}| = D^{\binom{n-1}{m}}$ ,
\end{center}
\vspace{-0.1cm}

and here too one can represent the Subd. of $|p'_{ik}|$ in similar way as with \textit{Jacobi} the Subd. of $|a'_{ik}|$.\textsuperscript{88)}

Even more general is \textit{Sylvester}'s theorem\textsuperscript{89)}, which we can briefly characterize as referring to bordering of the D. $|p_{ik}|$.

Other works concern themselves with composing D. from rows of two given D., and considering these new D. as elements of a D.\textsuperscript{90)}.

\vspace{0.2cm}

\textbf{24. Rank of the Determinant.} According to \textit{Kronecker} one designates as rank $r$ of a D. the largest number of the property that not all Subd. of $r$th degree vanish\textsuperscript{91)}. Through exchange and through linear combinations of the rows $r$ is not changed. If D. is of rank $r$, then its El. can be composed from two systems $a_{ik}$ $(i=1,...n; k=1,...r)$ and $b_{ik}$ $(i=1,...r; k=1,...n)$\textsuperscript{92)}. Of importance is this concept for many questions of algebra, especially solution of linear equations (IB1b).

\vfill
\leftline{\rule{2in}{0.4pt}}
\vspace{0.2cm}
{
\footnotesize
86) l. c. §11. — \textit{C. W. Borchardt}, Brief an Baltzer (1853).

87) J. f. Math. 61 (1863), p. 350.

88) \textit{C. W. Borchardt}, J. f. Math. 61 (1863), p. 353, 355, draws attention that the theorem is a special case of the one given earlier by \textit{Sylvester}; \textit{Kronecker}, Berl. Ber. (1882), p. 822 proves its identity with the above one by \textit{Jacobi}. — Cf. \textit{Picquet}, C. R. 86 (1878), p. 310; J. de l'Éc. Pol. cah. 45 (1878), p. 201.

89) Phil. Mag. (4), 1 (1851), p. 415. Cf. \textit{Frobenius}, J. f. Math. 86 (1879), p. 54); Berl. Ber. (1894), p. 242. — \textit{Netto}, Acta mat. 17 (1894), p. 201; J. f. Math. 114 (1895), p. 345. \textit{R. F. Scott}, Lond. Proceed. 14 (1883), p. 91. \textit{C. A. v. Velzer}, Amer. J. 6 (1883), p. 164. \textit{Em. Barbier}, C. R. 96 (1883), p. 1845; ib. 97 (1883), p. 82. \textit{E. J. Nanson}, Lond. phil. Mag. (5) 44 (1897), p. 396.

90) \textit{Picquet}, l. c. \textit{G. Zehfuss}, Z. f. Math. 7 (1862), p. 496.

91) Berl. Ber. (1884), p. 1071.

92) \textit{Kronecker}, J. f. Math. 72 (1870), p.152. \textit{Baltzer}, Determinanten, 4. Aufl. Leipz. (1875), p. 53.

}
