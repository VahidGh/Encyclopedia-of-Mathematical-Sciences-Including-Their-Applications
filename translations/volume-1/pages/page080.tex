\thispagestyle{fancy}

\vspace{0.5cm}

\textbf{25.} Here may still be mentioned a theorem by \textit{Mac-Mahon} relating to general D. (Phil. Trans. 185 (1894), p.146). Between a D. and all the Subd. whose main diagonals fall into the main diagonal of D., there exist $2^n-n^2+n-2$ relations. Cf. also \textit{Muir}, Phil. Mag. (1894), p. 537; Edinb. Proceed. 20 (1895), p. 300. \textit{Cayley}, ibid. p. 306. \textit{Nanson}, ibid. (1897), p. 362.

\vspace{0.2cm}

\textbf{26. Symmetric Determinants.} For \textit{symmetric} D., i.e. such D. whose El. are symmetric in relation to the main diagonal, the $a'_{ik}$ also form a symmetric D. — Every power of a symmetric D., and every even power of any D. is symmetric\textsuperscript{93)}. The product of a symmetric D. into the square of any D. can be represented as symmetric D.\textsuperscript{94)} If $r$ is the rank of a symm. D., then it has a non-vanishing principal Subd. of degree $r$.\textsuperscript{95)} \textit{H. G. Grassmann} had first indicated\textsuperscript{96)}, that between the Subd. of symmetric D. linear relations exist; the same theorem was later rediscovered by \textit{Kronecker}\textsuperscript{97)}, and \textit{C. Runge} has shown\textsuperscript{98)}, that the relations given by him are the only existing ones. These have the following character:

\vspace{-0.1cm}
\begin{center}
    $|a_{gh}| = \sum |a_{ik}|$ \quad $(g=1,...m; h=m+1,...2m; i=1,...m-1,r;$
    
    \quad \quad \quad \quad $k=m+1,...r-1,m,r+1,...,2m)$ .
    
\end{center}
\vspace{-0.1cm}

If one borders a symmetric, vanishing D. in symmetric way, then the resulting D. considered as function of the bordering elements is a square\textsuperscript{99)}, as easily follows from Nr.19. If one enters $a_{ii}+z$ instead of the $a_{ii}$ and sets the resulting symmetric D. equal to zero, then this equation has in $x$ only real roots. The resulting equation is called the "secular equation"\textsuperscript{100)}. (Cf. Nr. 31.)

\vfill
\leftline{\rule{2in}{0.4pt}}
\vspace{0.2cm}
{
\footnotesize
93) \textit{H. Seeliger}, Z. f. Math. 20 (1875), p. 468 - the El. of any power of a sym. D.

94) \textit{O. Hesse}, J. f. Math. 49 (1853), p. 246. — Cf. about an extension \textit{Muir}, Amer. J. 4 (1881), p. 351.

95) \textit{S. Gundelfinger}, J. f. Math. 91 (1881), p. 235; cf. \textit{Hesse}, analyt. Geom. d. Raumes, 3. Aufl. Leipz. (1881), p. 460. \textit{Frobenius}, Berl. Ber. (1894), p. 245.

96) Ausdehnungslehre, Berlin (1862), p. 131. Cf. \textit{Mehmke}, Math. Ann. 26 (1885), p. 209. The way how \textit{Grassmann} uses instead of D. certain "combinatorial product formations" is most simply recognized from the "Overview" (Arch. f. Math. 6 [1845], p. 337). More details are found in the "Ausdehnungslehre" §37, §51ff., §63 ff. The D. appears thereby as a product $\prod(a_{i_1}e_1+a_{i_2}e_2+...)$ of "extensive quantities", where $e_x^2=0$, $e_x e_{\lambda}=-e_{\lambda}e_x$ is.

97) Berl. Ber. (1882), p. 821. Cf. \textit{Darboux}, J. d. Mat. (2) 19 (1874), p. 347.

98) J. f. Math. 93 (1882), p. 319.

99) \textit{Cauchy}, l. c., p. 69.

100) \textit{J. L. Lagrange}, Mém. de Berlin (1773), p. 108 for $n=3$; generally

}
