\thispagestyle{fancy}
\fancyhead[LO]{27. 28. Recurrent and Half-symmetric Determinants.}

\vspace{0.5cm}

\textbf{27. Recurrent Determinants. Circulants.} The symmetry appears in \textit{recurrent} D. $a_{ik}=a_{i+k-2}$ in even stronger measure. \textit{Hankel} (l. c.), who designates them as orthosymmetric, represents them as $|\Delta_k|$, where the $\Delta_k$ are the initial terms of the difference series of $a_{i+k}$. These D. appear frequently in algebra\textsuperscript{101)}; their rank becomes thereby of importance.

A special case of this form those recurrent D. of $n$th degree, where $a_{n+i}=a_i$\textsuperscript{102)}; and with these closely connected are the \textit{circulants} (cf. Nr. 16) ($a_{ik}=a_{i+1,k+1}$), which are symmetric in relation to the secondary diagonal, which can be transformed into those through exchange of R. A circulant can be decomposed into the product of the $n$ factors 

\vspace{-0.1cm}
\begin{center}
    $ a_1 + \omega^\alpha a_2 + \omega^{2\alpha} a_3 + ... + \omega^{(n-1)\alpha} a_n$ ,\quad $(\alpha=1,2,...n)$, 
\end{center}
\vspace{-0.1cm}
    
where $\omega$ means a primitive $n$th root of unity; from this follows immediately that a circulant of $2n$th degree can be represented as one of $n$th degree\textsuperscript{103)}. A circulant of $2n$ degree can further be expressed as product of one of $n$th degree and a similarly formed one\textsuperscript{104)}.

\vspace{0.2cm}

\textbf{28. Half-symmetric Determinants.} For \textit{half-symmetric} D.\textsuperscript{105)} ($a_{ik}=-a_{ki}$, $a_{ii}=0$) the following theorems hold: $A_{ik}=A_{ki}$; $D=0$ for odd $n$; on the other hand $a'_{ik}=-a'_{ki}; \frac{\partial D}{\partial a_{ik}}=0; D=0$; is a square for even $n$. Every term of $\sqrt{D}$ is a product of $\frac{1}{2}n$ El. $a_{ik}$, whose indices are all different from each other, as e.g. the term appearing in $\sqrt{D}$ shows $a_{12} a_{34} ... a_{n-1,n}$. $\sqrt{D}$ is set by \textit{Cayley} (l. c.) $=\pm(1,2,...n)$ and designated as "Pfaffian".

\vspace{-0.1cm}
\leftline{\rule{2in}{0.4pt}}
\vspace{0.1cm}
{
\footnotesize
\textit{Cauchy}, Exerc. d. Math. 4 (1829), p. 140. \textit{E. Kummer}, J. f. Math. 26 (1843), p. 268. \textit{G. Bauer}, J. f. Math. 71 (1870), p. 46. \textit{Sylvester}, Phil. Mag. 2 (1852), p. 138. \textit{Borchardt}, J. de Math. 12 (1847), p. 50; J. f. Math. 30 (1846), p. 38.

101) \textit{Jacobi}, J. f. Math. 15 (1836), p. 101. \textit{Kronecker}, Berl. Ber. (1881), Juni; J. f. Math. 99 (1886), p. 346. \textit{Frobenius}, Berl. Ber. (1894), p. 241.

102) "persymmetric D." according to \textit{Muir}, Quart. J. 18 (1882), p. 264.

103) \textit{J. W. L. Glaisher}, Quart. J. 15 (1878), p. 347; ibid. 16 (1878), p. 31. Cf. also IA 6, Nr. 28, 24.

104) \textit{R. F. Scott}, Quart. J. 17 (1880), p. 129.

105) \textit{Lagrange} and \textit{S. D. Poisson} are probably, according to \textit{Jacobi}, first encountered such D. Cf. \textit{Jacobi}, J. f. Math. 2 (1827), p. 354. — \textit{Cayley}, J. f. Math. 38 (1849), p. 93, calls them "skew-symmetric". He proves first that D is a square for even n. J. f. Math. 32 (1846), p. 119; ibid. 50 (1855), p. 299.

106) \textit{Brioschi}, J. f. Math. 52 (1856), p. 133. \textit{Cayley}, l. c. Cf. an extension by \textit{Muir}, Phil. Mag. (5) 12 (1881), p. 391.

}
