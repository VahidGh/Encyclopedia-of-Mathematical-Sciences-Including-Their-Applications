\thispagestyle{fancy}

\vspace{0.5cm}

The square of every D. of even degree can be transformed into a half-symmetric D.\textsuperscript{106)}, so that D. itself appears as Pfaffian. \textit{Cayley} has further shown (l. c.), that when one borders a half-symmetric D. of odd degree arbitrarily by $a_{\alpha k}$, $a_{i\beta}$, the resulting D. breaks down into a product $\pm(\alpha,2,...n)\cdot(\beta,2,...n)$ of two Pfaffians. For $\alpha=\beta=1$ the previous theorem follows this.

\vspace{0.1cm}

\textbf{29. Skew Determinants.} If one drops the condition $a_{ii}=0$, one arrives at the \textit{skew} D., whose treatment likewise goes back to \textit{Cayley} (l. c.). The development of the skew D. according to the terms of the main diagonal (Nr. 20) delivers aggregates of half-symm. D. Thus if every $a_{ii}=z$, then in the development of D. according to powers of $z$ only the terms with exponents $n$, $n-2$, $n-4$,... appear.

\vspace{0.1cm}

\textbf{30. Centrosymmetric and Other Determinants.} Finally let the \textit{centrosymmetric} D. ($a_{ik}=a_{n+1-i,n+1-k}$) be briefly mentioned. Every such of even degree $2m$ can be represented as product of two D. of $m$th degree. Since now circulants (Nr. 27) can be made into centrosymmetric D. through rearrangement of the rows, the theorem mentioned (Nr. 27) follows this easily.

\vspace{0.1cm}

\textbf{31. Further Determinant Formations.} Besides the mentioned special formations many others have been investigated; thus for example the centroskew D. connect to the last discussed ones; further the \textit{Vandermonde} or \textit{power determinants} are to be mentioned, where $a_{ik}=a_i^{v_k}$, where the $v_k$ mean arbitrary numbers. The continued fraction determinants\textsuperscript{107)}, the \textit{continuants} (\textit{Sylvester}), deliver the representation of numerators and denominators of the approximation values of a continued fraction\textsuperscript{108)}. \textit{Hermite} considers Par. C. R. 41 (1855), p. 181, J. f. Math. 52 (1856), p. 40 Det., in which $a_{ik}$ and $a_{ki}$ are complex conjugates. Extension of the secular equation.

To function theory connect formations like: 1) the \textit{Wronskian} D.; 2) the \textit{Jacobian} (functional) D.; 3) the \textit{Hessian} D. 

For 1) the $a_{1i}$ are functions of $x$; the $a_{xi}$ their $(x-1)$th derivatives\textsuperscript{109)}.

\vspace{-0.1cm}
\leftline{\rule{2in}{0.4pt}}
\vspace{0.1cm}
{
\footnotesize
107) \textit{Painvin}, J. d. Math. (2) 3 (1858), p. 41. \textit{J. Sylvester}, Am. J. 1 (1878), p. 344.

108) \textit{Sylvester}, Phil. Mag. 5 (1859), p. 458; 6 (1853), p. 297. \textit{W. Spottiswoode}, J. f. Math. 51 (1856), p. 209. \textit{E. Heine}, ibid. 57 (1860), p. 231. \textit{S. Günther}, Erlangen (1873) u. Math. Ann. 7 (1874), p. 267. — Cf. II A 3.

109) \textit{C. J. Malmsten}, J. f. Math. 39 (1850), p. 91. \textit{Hesse}, ibid. 54 (1857), p. 249. \textit{E. B. Christoffel}, ibid. 55 (1858), p. 281. \textit{Frobenius}, ibid. 76 (1873), p. 236. \textit{M. Pasch}, ibid. 80 (1875), p. 177.

}
