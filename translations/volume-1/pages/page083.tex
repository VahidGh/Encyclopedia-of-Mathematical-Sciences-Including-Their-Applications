\thispagestyle{fancy}
\fancyhead[LO]{32. Determinants of Higher Rank. 33. Infinite Determ. 34. Matrices.}

\vspace{0.5cm}

For 2) $a_i$ are functions of $n$ variables $x_1,...x_n$, and $a_{\chi i} = \frac{\partial a_i}{\partial x_\chi}$\textsuperscript{110)}.

For 3) $a$ is a function of $x_1,...x_n$, and $a_{\chi i} = \frac{\partial^2 a}{\partial x_\chi \partial x_i}$\textsuperscript{111)}.

To algebra connect formations like resultants and discriminants. We refer about this to IB1a and b.

\vspace{0.1cm}

\textbf{32. Determinants of Higher Rank.} Determinants of higher ($\nu$th) rank are formed from $n^\nu$ quantities $a_{h_1,...h_\nu}$ in such a way that one exchanges the indices of equal position among themselves; then products of $n$ of these quantities are formed, where never two factors at equal position have equal index, and finally according to the earlier sign rule the $\pm$ sign is prefixed. All these aggregates form the D. Of these hold a series of properties of ordinary Det.; others are to be modified; Det. of \textit{even} and such of \textit{odd} rank behave in some respects differently\textsuperscript{112)}. Here too a treatment in \textit{Grassmann}'s sense is possible (\textit{G. v. Escherich} l. c.), cf. note 96.

\vspace{0.1cm}

\textbf{33. Infinite Determinants.} If one considers $a_{ik}$ $(i,k=1,2,...\infty)$, one can understand $D_n=|a_{ik}|$ $(i,k=1,2,...n)$ as function of $n$. If $n$ grows, one arrives at the concept of \textit{infinite} Det. Above all here the existence question is to be raised\textsuperscript{113)}. These formations are important for differential equations. Cf. IA3 Nr. 58, 59.

\vspace{0.1cm}

\textbf{34. Matrices.} A system of $m \cdot n$ quantities $a_{ik}$ $(i=1,2,...m; k=1,2,...n)$ is called a \textit{matrix}. To these structures connects a series of fundamental questions, whose treatment in IA 4 (bilinear forms) is given.

\vfill
\leftline{\rule{2in}{0.4pt}}
\vspace{0.2cm}
{
\footnotesize
110) \textit{Jacobi}, J. f. Math. 12 (1834), p. 38 = Werke III, p. 233; J. f. Math. 22 (1841), p. 319 = Werke III, p. 393. \textit{Sylvester}, Phil. Trans. (1854), p. 72. \textit{Cayley}, J. f. Math. 52 (1856), p. 276. \textit{Clebsch}, ibid. 69 (1868), p. 355. \textit{Kronecker}, ibid. 72 (1870), p. 155 etc.

111) \textit{Hesse}, J. f. Math. 28 (1844), p. 83; ibid. 42 (1851), p. 117; ibid. 56 (1859), p. 263. \textit{Sylvester}, Cambr. a. Dubl. M. J. 6 (1851), p. 186.

112) Cubic D. were first treated by \textit{A. de Gasparis} (1861). Following were: \textit{Dahlander}, Oefvers. of K. Akad. Stockh. (1863). \textit{G. Armenante}, Giorn. di Battagl. 6 (1868), p. 175. \textit{E. Padova}, ibid. p. 182. \textit{G. Zehfuss}, Frankf. (1868). \textit{G. Garbieri}, Giorn. d. Batt. 15 (1877), p. 89. \textit{H. W. L. Tanner}, Proceed. Lond. M.S. 10 (1879), p. 167. \textit{R. F. Scott}, ib. 11 (1880), p. 17. \textit{G. v. Escherich}, Wien. Denkschr. 43 (1882), p. 1. \textit{L. Gegenbauer}, ib. 43 (1882), p. 17; 46 (1883), p. 291; 50 (1885), p. 145; 55 (1889), p. 39. Wien. Ber. 101 (1892), p. 425.

113) \textit{G.W. Hill}, Acta Math. 8 (1886), p. 1, essentially reprint of a monograph Cambridge U.S.A. (1877). \textit{H. Poincaré}, Bull. Soc. d. Fr. 13 (1884—85), p. 19; 15 (1885—86), p. 77. \textit{Helge von Koch}, Acta math. 15 (1891), p. 53; ibid. 16 (1892—1893), p. 217.

}
