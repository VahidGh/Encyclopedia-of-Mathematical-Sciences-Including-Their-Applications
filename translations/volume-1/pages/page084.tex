\thispagestyle{fancy}

\vspace{0.5cm}

The concept of rank as well as composition of matrices is to be established. From a matrix Det. can be formed in different ways. Their connection, as well as their invariant properties are to be investigated. Here belongs the case of \textit{corresponding matrices}: $a_{ik}$ $(i=1,...m; k=1,...\alpha)$ and $b_{jl}$ $(j=1...\beta; l=1,2,...m)$, where $\alpha+\beta=m$, and the $\alpha\cdot\beta$ relations exist $\sum_{(q)} a_{qk}b_{jq}=c_{kj}=0$, where proportionality of corresponding determinants occurs\textsuperscript{114)}.

\vspace{0.2cm}

\textbf{35. Monographs.} As textbooks about determinants we list, passing over those intended only for school use, as the main ones:

\textit{Brioschi}, La teoria dei determinanti. Pavia (1854). German, Berlin (1856).

\textit{Spottiswoode}, Elementary Theorems relating to Determinants, J. f. Math. 51 (1856), p. 209—271 and 328—381.

\textit{Baltzer}, Theorie u. Anwendung der Determinanten. Leipzig (1857). Fifth Ed. (1881).

\textit{Salmon}, Lessons introductory to the modern higher algebra. Dublin (1859). German Leipz. (1877) by \textit{Fiedler}.

\textit{Hesse}, Die Determinanten, elementar behandelt. Leipz. (1872).

\textit{Günther}, Lehrbuch der Determinantentheorie. Erlangen (1875). Second Ed. (1877).

\textit{Scott}, A treatise on the theory of determinants. Cambridge (1880).

\textit{P. Mansion}, Eléments de la théorie des déterminants. Paris 4th ed. (1883).

\textit{L. Leboulleux}, Traité élémentaire des déterminants. Genève (1884).

\textit{A. Sickenberger}, Die Determinanten in genetischer Behandlung. München (1885).

\textit{Gordan}, Vorlesungen über Invariantentheorie. I. Determinanten. Leipz. (1885).

\textit{Pascal}, I determinanti. Milano (1897).

\vspace{0.2cm}
\leftline{\rule{2in}{0.4pt}}
\vspace{0.2cm}
{
\footnotesize
114) The concept of matrix was introduced by \textit{A. Cayley}, J. f. Math. 50 (1855), p. 282. \textit{Cayley} wants to keep the theory of matrices separate from that of determinants.

}

\vspace{2cm}
\centerline{\rule{2in}{0.4pt}}