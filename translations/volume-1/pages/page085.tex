\chapter*{}

\vspace{-4cm}
{\centering\section*{I A 3. IRRATIONAL NUMBERS AND CONVERGENCE OF INFINITE PROCESSES}}

\begin{center}

{\small BY}

\vspace{0.2cm}

\textbf{ALFRED PRINGSHEIM}\\[1pt]
{\small IN MUNICH}

\centerline{\rule{0.5in}{0.2pt}}

\vspace{0.2cm}
\textbf{Table of Contents}
\vspace{0.2cm}

\textbf{\small{Part One. Irrational Numbers and Concept of Limit}}

\textbf{\small{I. Irrational Numbers}}
\end{center}
\vspace{-0.2cm}

{\fontsize{10}{0}\selectfont
\begin{enumerate}[itemsep=-1pt]
    \item[1.] \textit{Euclid}'s Ratios and Incommensurable Quantities
    \item[2.] \textit{Michael Stifel}'s Arithmetica Integra
    \item[3.] The Concept of Irrational Numbers in Analytic Geometry
    \item[4.] The \textit{Cantor-Dedekind} Axiom and Arithmetic Theories of Irrational Numbers
    \item[5.] The Theories of \textit{Weierstrass} and \textit{Cantor}
    \item[6.] The Theory of \textit{Dedekind}
    \item[7.] \textit{Du Bois-Reymond}'s Fight Against Arithmetic Theories
    \item[8.] The Complete Arithmetization in \textit{Kronecker}'s Sense
    \item[9, 10.] Various Forms of Representation of Irrational Numbers and Irrationality of Certain Forms of Representation
\end{enumerate}
\begin{center}
    \textbf{\small{II. Concept of Limit}}
\end{center}
\begin{enumerate}[itemsep=-1pt]
    \item[11.] The Geometric Origin of the Concept of Limit
    \item[12.] The Arithmetization of the Concept of Limit
    \item[13.] The Criterion for the Existence of Limits
    \item[14.] The Infinitely Large and Infinitely Small
    \item[15.] Upper and Lower Limit
    \item[16.] Upper and Lower Bound
    \item[17.] Calculation with Limits. The Number $e=\lim(1+\frac{1}{\nu})^\nu$
    \item[18.] So-called Indeterminate Expressions
    \item[19.] Gradation of Becoming Infinite and Zero
    \item[20.] Limits of Doubly-Infinite Number Sequences
\end{enumerate}
\vspace{0.2cm}
\begin{center}
    \textbf{\small{Part Two. Infinite Series, Products, Continued Fractions and Determinants}}
    \vspace{0.2cm}

    \textbf{\small{III. Infinite Series}}
\end{center}
\begin{enumerate}[itemsep=-1pt]
    \item[21.] Convergence and Divergence
    \item[22, 23.] The Convergence Criteria of \textit{Gauss} and \textit{Cauchy}
    \item[24.] \textit{Kummer}'s General Criteria
    \item[25.] The Theories of \textit{Dini}, \textit{Du Bois-Reymond} and \textit{Pringsheim}
    \item[26, 27.] The Criteria of First and Second Kind
    \item[28.] Other Forms of Criteria
\end{enumerate}
}

