{\fontsize{10}{0}\selectfont
\begin{enumerate}[itemsep=-1pt]
    \item[29.] Scope of Criteria of First and Second Kind
    \item[30.] The Boundary Regions of Divergence and Convergence
    \item[31.] Conditional and Unconditional Convergence
    \item[32.] Value Changes of Conditionally Convergent Series
    \item[33.] Criteria for Possibly Only Conditional Convergence
    \item[34.] Addition and Multiplication of Infinite Series
    \item[35.] Double Series
    \item[36.] Multiple Series
    \item[37.] Transformation of Series
    \item[38.] \textit{Euler-MacLaurin} Sum Formula. Semi-convergent Series
    \item[39.] Divergent Series
    \item[40.] Divergent Power Series
\end{enumerate}
% \vspace{0.1cm}
\begin{center}
    \textbf{\small{IV. Infinite Products, Continued Fractions and Determinants}}
\end{center}
\begin{enumerate}[itemsep=-1pt]
    \item[41.] Infinite Products: Historical
    \item[42.] Convergence and Divergence
    \item[43.] Transformation of Infinite Products into Series
    \item[44.] Factorials and Faculties
    \item[45.] General Formal Properties of Continued Fractions
    \item[46.] Recursive and Independent Calculation of Approximation Fractions
    \item[47.] Approximation Fraction Properties of Special Continued Fractions
    \item[48.] Convergence and Divergence of Infinite Continued Fractions. General Divergence Criterion
    \item[49.] Continued Fractions with Positive Terms
    \item[50.] Convergent Continued Fractions with Terms of Arbitrary Sign
    \item[51.] Periodic Continued Fractions
    \item[52.] Transformation of Infinite Continued Fractions
    \item[53.] Transformation of an Infinite Series into an Equivalent Continued Fraction
    \item[54.] Other Continued Fraction Developments of Infinite Series
    \item[55.] Continued Fractions for Power Series and Power Series Quotients
    \item[56.] Relations between Infinite Continued Fractions and Products
    \item[57.] Ascending Continued Fractions
    \item[58.] Infinite Determinants: Historical
    \item[59.] Main Properties of Infinite Determinants
\end{enumerate}
}

\vspace{-0.5cm}
\begin{center}
\centerline{\rule{1in}{0.2pt}}

\textbf{Literature}

\textbf{\small{Textbooks}}
\end{center}

{\fontsize{10}{0}\selectfont
\textit{Leonhard Euler}, Introductio in analysin infinitorum. I. Lausannae 1748. German by \textit{Michelsen} (Berlin 1788) and by \textit{H. Maser} (Berlin 1885).

\textit{Augustin Cauchy}, Cours d'analyse de l'école polytechnique. I. Analyse algébrique. Paris 1821. German by \textit{C. Itzigsohn}, Berlin 1885.

\textit{M. A. Stern}, Lehrbuch der algebraischen Analysis. Leipzig 1860.

\textit{Eugene Catalan}, Traité élémentaire des séries. Paris 1860.

\textit{Oskar Schlömilch}, Handbuch der algebraischen Analysis. Jena 1868 (4th Ed.).

\textit{Charles Méray}, Nouveau Précis d'analyse infinitésimale. Paris 1872. Nouvelles leçons sur l'analyse infinitésimale. I. Paris 1894.

\textit{Karl Hattendorff}, Algebraische Analysis. Hannover 1877.

}
