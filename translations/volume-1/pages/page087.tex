\thispagestyle{fancy}
\fancyhead[LO]{1. Euclid's Ratios and Incommensurable Quantities}

\vspace{0.5cm}

{\fontsize{10}{0}\selectfont
\textit{Rudolf Lipschitz}, Lehrbuch der Analysis. I: Grundlagen der Analysis. Bonn 1877.

\textit{Moritz Pasch}, Einleitung in die Differential- und Integral-Rechnung. Leipzig 1882.

\textit{Max Simon}, Elemente der Arithmetik als Vorbereitung auf die Functionentheorie. Strassburg 1884.

\textit{Otto Stolz}, Vorlesungen über allgemeine Arithmetik. 2 Vols. Leipzig 1885-86.

\textit{Jules Tannery}, Introduction à la théorie des fonctions d'une variable. Paris 1886.

\textit{Camille Jordan}, Cours d'analyse de l'école polytechnique. 2nd ed. I. Paris 1893.

\textit{Ernesto Cesàro}, Corso di analisi algebrica. Torino 1894.

\textit{Otto Biermann}, Elemente der höheren Mathematik. Leipzig 1895.

\textit{Alfred Pringsheim}, Vorlesungen über die element. Theorie der unendl. Reihen und der analyt. Functionen. I. Zahlenlehre. (Forthcoming from B. G. Teubner, Leipzig.)

Regarding irrational numbers compare also: \textit{P. Bachmann}, Vorl. über die Natur der Irrationalzahlen, Leipzig 1892; regarding infinite series: \textit{J. Bertrand}, Traité de calc. différentiel, Paris 1864\textsuperscript{1)}.

\vspace{-0.3cm}
\begin{center}
\textbf{Monographs}
\end{center}

\textit{Siegm. Günther}, Beiträge zur Erfindungsgeschichte der Kettenbrüche. School Program, Weissenburg 1872.

\textit{Paul du Bois-Reymond}, Die allgemeine Functionentheorie. I (only). Tübingen 1882.

\textit{R. Reiff}, Geschichte der unendlichen Reihen. Tübingen 1889.

\textit{Giulio Vivanti}, Il concetto d'infinitesimo e la sua applicazione alla matematica. Mantova 1894.

}

\vspace{-0.1cm}
\centerline{\rule{1in}{0.2pt}}

\begin{center}
{\fontsize{13}{13}\selectfont\textbf{Part One. Irrational Numbers and Concept of Limit}}

{\fontsize{12}{12}\selectfont\textbf{I. Irrational Numbers}}
\end{center}

{\fontsize{11}{11}\selectfont\textbf{1. Euclid's Ratios and Incommensurable Quantities.}} The \textit{irrational numbers}, whose fundamental introduction forms one of the most essential foundations of \textit{general arithmetic}, nevertheless first grew out of \textit{geometric} needs: they originally appear as expression for the \textit{ratio of incommensurable} (i.e. not measurable by any common measure) \textit{pairs of line segments} (e.g. the diagonal and side of a square\textsuperscript{2)}). In this sense, Book 5 of \textit{Euclid}, which develops the \textit{general} theory of \textit{"ratios"}, as well as Book 10 dealing with \textit{incommensurable} quantities, can be viewed as the literary starting point for the theory of \textit{irrational numbers}. Nevertheless, \textit{Euclid} naturally treats only very specific quantities constructible with \textit{compass} and \textit{ruler} (thus, arithmetically speaking, representable by square roots

\vfill
\leftline{\rule{2in}{0.4pt}}
\vspace{0.2cm}
{
\footnotesize
1) The very extensive sections on \textit{series} in \textit{S. F. Lacroix's} large Traité de calc. diff. et intégr. (3 Vols., 2nd ed., Paris 1810-1819) contain little useful material about \textit{elementary} series theory.

2) That the diagonal and side of a square are incommensurable is said to have been recognized already by \textit{Pythagoras}; see \textit{M. Cantor}, Gesch. der Math. 1 (Lpz. 1880), p. 130, 154.

}
