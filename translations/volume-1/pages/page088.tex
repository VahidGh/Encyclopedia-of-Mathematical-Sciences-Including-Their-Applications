\thispagestyle{fancy}

\vspace{0.5cm}

as \textit{incommensurable line segments}\textsuperscript{3)}; the notion that the ratio of two such \textit{special} or even two \textit{entirely arbitrary} incommensurable line segments defines a \textit{specific} (irrational) \textit{number} remained foreign to him, as to all mathematicians of antiquity\textsuperscript{4)}.

\vspace{0.5cm}

\textbf{2. Michael Stifel's Arithmetica Integra.} But even for the arithmeticians and algebraists of the Middle Ages and Renaissance, the \textit{irrationalities} inherited from geometry were still not \textit{"real"}, but at best \textit{improper} or \textit{fictitious numbers}\textsuperscript{5)}, which were merely tolerated as a necessary evil. The first decisive step toward a more correct assessment of irrational numbers is probably owed to \textit{Michael Stifel}, who in Book 2 of his \textit{Arithmetica integra}\textsuperscript{6)} deals extensively with \textit{"Numeris irrationalibus"}\textsuperscript{7)} following \textit{Euclid}'s Book 10. Although he initially still adheres to the view abstracted from \textit{Euclid} that \textit{irrational} numbers \textit{are not} \textit{"real"} numbers\textsuperscript{8)}, this ultimately contains, as the

\vfill
\leftline{\rule{2in}{0.4pt}}
\vspace{0.2cm}
{
\footnotesize
3) More details about this (besides in \textit{Euclid}): \textit{Klügel}, Math. W.-B. 2, p. 949. \textit{M. Cantor} l.c. p. 230. \textit{Schlömilch}, Ztschr. f. M. 34 (1889), Hist.-lit. Abth. p. 201.

4) \textit{Euclid} says (Elem. X, 7) quite explicitly: Incommensurable quantities \textit{do not} relate to each other like numbers. \textit{Jean Marie Constant Duhamel} (Des méthodes dans les sciences de raisonnement, Paris 1865-70) attempted (l.c. 2, p. 72-75) to make Euclidean ratio theory useful for founding the general concept of irrational numbers. But he ultimately spoils his initially correct method through unnecessary introduction of an unclear geometric limit concept. In contrast, \textit{O. Stolz} (Allg. Arithm. 1, p. 35ff.), alongside a reproduction of Euclidean ratio theory adapted to today's presentation style, gives the necessary indications of how the latter could be developed into an unobjectionable theory of real numbers (particularly including irrational ones). Cf. also: \textit{O. Stolz}, Grössen und Zahlen (Rectoral address of March 2, 1891, Lpz. 1891), p. 16; further: Nr. 13, footnote 84.

5) \textit{"Numeri ficti"}, usually designated as \textit{"Numeri surdi"}: this name attributed to \textit{Leonardo of Pisa} (Liber abaci, 1202) persisted into the 18th century, in England (\textit{"Surds"}) up to the present.

6) Nürnberg 1544. Fol. 103-223.

7) \textit{Stifel} uses the designation \textit{"radices surdae"} in a narrower sense l.c. Fol. 134.

8) The contrary statement in \textit{C. J. Gerhardt} (Gesch. der Math. in Deutschl., Munich 1877, p. 69) seems incorrect to me. The relevant passage in \textit{Stifel} (l.c. Fol. 103) reads quite unambiguously: "Non autem potest dici numerus verus, qui talis est, ut praecisione careat et ad numeros veros nullam cognitam habet proportionem. Sicut igitur infinitus numerus non est numerus: \textit{sic irrationalis numerus non est verus numerus} atque lateat sub quadam infinitatis nebula."

}
