\thispagestyle{fancy}
\fancyhead[LO]{3. The Concept of Irrational Numbers in Analytic Geometry}

\vspace{0.5cm}

relevant context teaches, only a different mode of \textit{expression} from today's, which basically says nothing other than that \textit{irrational} numbers are simply \textit{not rational} ones. On the other hand, \textit{Stifel} documents his understanding, essentially \textit{agreeing} with modern views, through the statement that \textit{every irrational} number, just like \textit{every rational} one, has a \textit{uniquely determined place} in the ordered number sequence\textsuperscript{9)}. With this, indeed, the most essential aspect that characterizes irrationalities as \textit{numbers} appears sharply emphasized for the first time. Of course, here under irrational numbers only certain simple \textit{root quantities} are to be understood - a restriction that explains itself partly from the then still existing sole dominion of \textit{Euclidean} methods in \textit{geometry}, partly also from the circumstance that finding the nth root of a whole number lying between $g^n$ and $(g + 1)^n$ ($g$ whole number) was the \textit{only} task whose \textit{insolubility} by a \textit{rational} number one could really \textit{prove} at that time\textsuperscript{10)}.

\vspace{0.5cm}

\textbf{3. The Concept of Irrational Numbers in Analytic Geometry.} Only the gradually occurring break with the geometry of the ancients, particularly the development of analytic-geometric method beginning with the appearance of \textit{Descartes' Géométrie} (1637), then the invention of infinitesimal calculus by \textit{Leibniz} and \textit{Newton} (1684; 1687) created the need to further develop the equivalence between \textit{line segments} and \textit{numbers} and correspondingly complete the concept of irrational numbers. While \textit{Descartes} had already designated arbitrary \textit{line segment ratios} with \textit{simple letters} and calculated \textit{with them like numbers}, the statement that \textit{every ratio} of two quantities corresponds to a \textit{number} appears at the beginning of \textit{Newton's Arithmetica universalis} (1707) directly as \textit{definition} of number\textsuperscript{11)}. And even more specifically tied to the geometric concept of \textit{measurable quantity}, \textit{Chr. Wolf}, whose textbooks,

\vfill
\leftline{\rule{2in}{0.4pt}}
\vspace{0.2cm}
{
\footnotesize
9) L.c. Fol. 103b, line 3 from bottom: "Item licet infiniti numeri \textit{fracti} cadant inter quoslibet duos numeros immediatos, quemadmodum etiam infiniti numeri \textit{irrationales} cadunt inter duos numeros integros immediatos. \textit{Ex ordinibus tamen utrorumque facile est videre, ut nullus eorum ex suo ordine in alterum possit transmigrare.}"

10) \textit{Stifel} l.c. Fol. 103b.

11) "\textit{Numerum} non tam multitudinem unitatum quam abstractam quantitatis cujusvis ad aliam ejusdem generis quae pro unitate habetur rationem intelligimus." Of course, as \textit{Stolz} aptly remarks (Allg. Arithm. 1, p. 94), this definition appears in \textit{N}. only as a kind of showpiece: for a real development of irrational number theory based on Euclidean ratio

}
