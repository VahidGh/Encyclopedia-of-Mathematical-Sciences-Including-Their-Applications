\thispagestyle{fancy}

\vspace{0.5cm}

extremely widespread in the first half of the 17th century, despite their lack of originality and sharper criticism can still be considered as expression of the views then accepted by the great majority, defines: "\textit{Number} is that which relates to unity as a straight line to a certain other straight line"\textsuperscript{12)}. The \textit{number} thus appears as expression for the result of \textit{measuring} one \textit{line segment} by another which plays the role of the \textit{unit segment} - a view that became the sole dominant one up into recent times and is still strictly maintained by individual mathematicians today. To every \textit{line segment} (or also - by means of a simple and known modification - to every \textit{point} on a line) there now corresponds a specific \textit{number}, namely either a \textit{rational} or an \textit{irrational} one, i.e. initially an unboundedly continuable algorithm in rational numbers (infinite continued fraction; infinite decimal fraction) to be obtained according to suitable rules (\textit{Euclidean} procedure for finding the greatest common measure\textsuperscript{13)} or unbounded subdivision of the measuring unit segment) to be obtained as an \textit{unboundedly continuable algorithm in rational numbers} (infinite continued fraction; infinite decimal fraction); the \textit{justification} for considering such an \textit{unbounded system of rational numbers} as a \textit{single specific number} is then seen exclusively in the fact that it is found as \textit{arithmetic equivalent} of a \textit{given line segment} using the \textit{same measurement methods} which yield a \textit{specific rational} number for \textit{other} line segments. From this it \textit{does not follow}, however, that one is conversely also justified in considering any \textit{arbitrarily given arithmetic} structure of the indicated kind as an \textit{irrational number} in the sense just defined, i.e. in regarding the existence of a \textit{line segment} generating that structure under suitable measurement as \textit{evident}\textsuperscript{14)}. This for the consistent

\vfill
\leftline{\rule{2in}{0.4pt}}
\vspace{0.2cm}
{
\footnotesize
theory it is by no means utilized. (Cf. also: \textit{Stolz}, Zur Geometrie der Alten, Math. Ann. 22 [1888], p. 516.)

12) Elementa Matheseos universae. 1, Halae 1710: Elementa Arithmeticae, Art. 10. (I quote from the second edition of 1730 available to me.)

13) Eucl. Elem. X, 2, 3. \textit{A. M. Legendre}, Geometrie, Livre III, Probl. 19.

14) The above cited \textit{Chr. Wolf} knows only the following to say about this (l.c. Art. 296): "In geometria et analysi demonstrabitur, tales radices, quae actu dari non possunt, esse ad unitatem ut rectam lineam ad rectam aliam, consequenter numeros eosque irrationales, cum ex hypothesi rationales non sint." That ultimately comes down again to considering as \textit{numbers} only those \textit{arithmetically defined} irrationalities that are \textit{geometrically constructible}. In doing so, \textit{W}. handles the concept of \textit{geometric constructibility} carelessly in such a way that he e.g. uses parabolas

}
