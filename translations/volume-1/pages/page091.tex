\thispagestyle{fancy}
\fancyhead[LO]{4. The Cantor-Dedekind Axiom and Arithmetic Theories}

\vspace{0.5cm}

the concept of irrational numbers fundamental point was until recent times either passed over in silence, or dismissed with the help of alleged geometric evidences, or obscured rather than clarified through metaphysical phrases about continuity, concept of limit, and infinitesimals.

\vspace{0.5cm}

\textbf{4. The Cantor-Dedekind Axiom and the Arithmetic Theories of Irrational Numbers.} \textit{G. Cantor} was probably the first to sharply emphasize that the assumption that to \textit{every} \textit{arithmetic structure} defined in the manner of an irrational number there must correspond a specific \textit{line segment} appears neither \textit{self-evident nor provable}, but rather involves an essential, purely \textit{geometric axiom}\textsuperscript{15)}. And almost simultaneously \textit{R. Dedekind} showed that the \textit{axiom} in question (or, more precisely, one \textit{equivalent} to it) first gives \textit{tangible content} \textit{to that} property which had previously been designated as \textit{continuity} of the straight line \textit{without any adequate definition}\textsuperscript{16)}. To make the foundations of general \textit{arithmetic} completely independent of such a \textit{geometric} axiom, each of the two named authors developed his own \textit{purely arithmetic} theory of irrational numbers\textsuperscript{17)}.

Another, likewise \textit{purely arithmetic} method of introduction had already been used for some time by \textit{K. Weierstrass} in his lectures on analytic functions\textsuperscript{18)}. 

\vfill
\leftline{\rule{2in}{0.4pt}}
\vspace{0.2cm}
{
\footnotesize
and then uses these for the alleged construction of $\sqrt[m]{x}$ (l.c. Elementa Analyseos, Art. 630).

15) Math. Ann. 5 (1872), p. 128.

16) \textit{Stetigkeit und irrationale Zahlen}. Braunschweig 1872. The axiom in question appears there in the following formulation: "If all points of the line fall into two classes such that \textit{every} point of the \textit{first} class \textit{lies} to the left of \textit{every} point of the \textit{second} class, then there exists \textit{one and only one} point which produces this division..."

17) L.c. The \textit{Cantor} theory was published approximately at the same time, as by its author himself, also by \textit{E. Heine} (with explicit reference to verbal communications from \textit{Cantor}) in a somewhat more detailed manner: J. f. Math. 74, p. 174 ff. On the other hand, \textit{Ch. Méray} \textit{independently} of \textit{Cantor} likewise discovered the foundations of this theory and published them approximately simultaneously with \textit{Cantor} and \textit{Heine} in his: Nouveau Précis d'Analyse infinitésimale, Paris 1872.

18) The basic principles of \textit{W}.'s theory were first briefly communicated by \textit{H. Kossak} in a program treatise of the Werder Gymnasium, Berlin 1872 (p. 18 ff.). More details can be found in \textit{S. Pincherle}, Giorn. di mat. 18 (1880), p. 185 ff.-- \textit{O. Biermann}, Theorie der analytischen Functionen, Leipzig 1887, p. 19 ff.

}
