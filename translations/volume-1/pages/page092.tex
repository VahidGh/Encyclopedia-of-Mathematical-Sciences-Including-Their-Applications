\thispagestyle{fancy}

\vspace{0.5cm}

\textit{Cantor} himself has subjected all \textit{three} forms of definition to a critical comparison in volume 21 (1883) of Math. Ann. (p. 565 ff.) and on this occasion somewhat modified his first presentation (probably following that given by \textit{Heine}), in such a way that the separation of the \textit{irrational number} to be defined from any \textit{notion of limit} is expressed even more sharply.

\vspace{0.3cm}

\textbf{5. The Theories of Weierstrass and Cantor.} The \textit{Weierstrass} theory and the somewhat more convenient to handle \textit{Cantor} theory, which one can appropriately describe with \textit{Heine}\textsuperscript{19)} as a fortunate further development of the former, both connect to a specific \textit{formal representation} of irrational numbers, as whose simplest and most familiar to everyone type appears that through \textit{infinite decimal fractions}\textsuperscript{20)}. While \textit{W}. retains from this the principle of \textit{sum formation} as exclusive \textit{generative element}, \textit{C}. takes from that model the more general concept of the so-called \textit{fundamental sequence}, i.e., a sequence of rational numbers $a_v$ of the nature that $|a_{v+\varrho}-a_v|$ for a \textit{sufficiently large} chosen value of $v$ and \textit{any} value of $\varrho$ becomes \textit{arbitrarily small}. It is then \textit{essential} that the \textit{general real} number to be defined (which according to circumstances can be a \textit{rational} or \textit{irrational} one) \textit{is not} obtained as \textit{sum} of an \textit{"infinite"} number of elements or as \textit{"infinitely distant"} term of a sequence through some nebulous \textit{limit process}. It appears rather as a finished, \textit{newly created object}, or, even more concretely according to \textit{Heine}\textsuperscript{21)}, as a \textit{new number symbol}, whose \textit{properties} are uniquely determined from those of the defining rational elements, to which a \textit{uniquely determined place} within the domain of rational numbers is assigned, and with which one can \textit{calculate} according to definite rules\textsuperscript{22)}.

\vspace{-0.1cm}
\leftline{\rule{2in}{0.4pt}}
\vspace{0.1cm}
{
\footnotesize
19) L.c. p. 173.

20) A detailed presentation of the \textit{Cantor} theory, which appropriately takes the theory of \textit{systematic fractions} (generalization of decimal fractions) as starting point, can be found in \textit{Stolz}, Allg. Arithm. 1, p. 97 ff.; another presentation, likewise not inappropriate for the beginner, in which \textit{two monotonic} number sequences (see No. 13 of this article) serve for the definition of irrational numbers and their arithmetic operations, is given by \textit{P. Bachmann}, Vorl. über die Natur der Irrationalzahlen (Lpz. 1892), p. 6 ff.

21) L.c. p. 173: "I place myself at the purely formal standpoint for the definition (of numbers), \textit{in that I call certain tangible signs numbers}, so that the existence of these numbers is thus not in question." Differently \textit{Cantor}: Math. Ann. 21 (1883), p. 553.

22) Cf. \textit{Pincherle} l.c. Art. 18, 28. \textit{Biermann} l.c. p. 24. \textit{Heine} l.c. p. 177. \textit{Cantor}, Math. Ann. 5, p. 125; 21, p. 568.

}
