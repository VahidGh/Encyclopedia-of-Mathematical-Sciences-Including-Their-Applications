\thispagestyle{fancy}
\fancyhead[LO]{The Theory of Weierstrass, Cantor Dedekind}

\vspace{0.5cm}

The \textit{general real numbers} defined in this way are naturally \textit{not} to be viewed initially as signs \textit{for specific quantities} (countable or measurable magnitudes), and the concepts \textit{"greater"} and \textit{"smaller"} defined for them accordingly do not designate \textit{quantity differences}, but merely \textit{successions}. In particular, the concept of \textit{rational} numbers also undergoes an \textit{extension} in the sense that they appear as \textit{signs} to which primarily only \textit{a specific succession} belongs\textsuperscript{23)}, and which \textit{can} represent specific \textit{quantities}, but \textit{need} not. If this decisive point is overlooked\textsuperscript{24)}, objections become understandable, such as those wrongly raised by \textit{E. Hittgens} against the theories of \textit{Weierstrass} and \textit{Cantor}\textsuperscript{25)}. That, moreover, the \textit{Weierstrass-Cantor} numbers (including the \textit{irrational} ones) can be used for representing specific \textit{quantities}, e.g. \textit{line segments}, has been explicitly shown by the authors of the theories in question\textsuperscript{26)}: to \textit{every line segment} corresponds (after fixing an arbitrary unit segment) \textit{one} and \textit{only one} specific \textit{number}. The \textit{converse} naturally holds again only for \textit{rational} and \textit{special irrational} numbers; for \textit{arbitrary irrational numbers} only \textit{if} one accepts the geometric \textit{axiom} mentioned in Art. 4\textsuperscript{27)}.

\vspace{0.5cm}

\textbf{6. The Theory of Dedekind.} \textit{Dedekind} defines the irrational number, without direct use of any arithmetic formalism, with the help of the concept of \textit{"cut"} introduced by him\textsuperscript{28)}; by this he understands a division of all rational numbers into two classes of individuals $(a_1)$ and $(a_2)$, such that throughout $a_1 < a_2$.

\vfill
\leftline{\rule{2in}{0.4pt}}
\vspace{0.2cm}
{
\footnotesize
23) One can, starting from this concept of \textit{unique succession}, arrive at a \textit{completely unified} construction of number theory if one introduces from the outset the \textit{natural} numbers \textit{not}, as usual, on the basis of the concept of cardinal number as \textit{cardinal numbers}, but rather (following \textit{H. Helmholtz} and \textit{L. Kronecker}) as \textit{ordinal numbers}. Cf. my essay Münch. Ber. 27 (1897), p. 325.

24) See e.g. \textit{R. Lipschitz}, Grundl. d. Anal., Section I, and cf. my lecture: Über den Zahl- und Grenzbegriff im Unterricht. Jahresb. d. D. M.-V. 6 (1848), p. 78.

25) Math. Ann. 33 (1889), p. 155; likewise 35, p. 451. Reply by \textit{Cantor}: ibid. 33, p. 476. Cf. also \textit{Pringsheim}, Münch. Sitzber. 27, p. 322, footnote.

26) \textit{Pincherle} l.c. Art. 19. \textit{Cantor}, Math. Ann. 5, p. 127.

27) In \textit{Pincherle}, whose presentation of \textit{W}.'s theory certainly cannot be regarded as an authentic one, curiously that \textit{axiom} (in \textit{Dedekind}'s form) is again considered as a self-evident fact (l.c. Art. 20).

28) L.c. 4.

}