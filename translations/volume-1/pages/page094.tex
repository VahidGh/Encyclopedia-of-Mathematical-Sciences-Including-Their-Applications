\thispagestyle{fancy}

\vspace{0.5cm}

If then among the numbers $a_1$ there is a \textit{greatest} or among the numbers $a_2$ a \textit{smallest}, then the respective (\textit{rational}) number is precisely the one which produces the \textit{cut} in question. In the other case, a \textit{newly created} individual $\alpha$, an \textit{irrational} number, is assigned to it and regarded as producing this \textit{cut}. On the basis of this definition, the relationships of these new numbers $\alpha$ among themselves and to the rational numbers $a$, as well as the elementary arithmetic operations, can be uniquely determined, as \textit{D}. himself has essentially carried out. A more detailed presentation in more \textit{geometric} garb has been given by \textit{M. Pasch} in his "Einleitung in die Differential- und Integral-Rechnung"\textsuperscript{29)} and later added some modifications\textsuperscript{30)} which make the in truth still essential \textit{arithmetic} foundation of that theory appear more clearly.

Precisely because \textit{Dedekind}'s method of introducing irrational numbers does not connect to any arithmetic algorithm, it gains the advantage of a very special brevity and conciseness. For the same reason, however, it also appears noticeably more abstract and adapts less conveniently to calculation than the \textit{Cantor} theory. Not inappropriately, therefore, \textit{J. Tannery} in his "Introduction à la Théorie des Fonctions"\textsuperscript{31)} has chosen a presentation which, starting from the \textit{Dedekind} definition, subsequently gains connection to \textit{Cantor}'s theory through inclusion of \textit{Cantor}'s \textit{fundamental sequences}.

\vspace{0.5cm}

\textbf{7. Du Bois-Reymond's Fight Against the Arithmetic Theories.} The separation of the concept of \textit{number} from that of \textit{measurable} magnitude, as it is established by the arithmetic theories of irrational numbers, with determination particularly by \textit{P. Du Bois-Reymond}

\vfill
\leftline{\rule{2in}{0.4pt}}
\vspace{0.2cm}
{
\footnotesize
29) Leipzig 1882, §13.

30) Math. Ann. 40 (1892), p. 149.

31) Paris 1886, Chap. 1. Incidentally, \textit{Tannery} makes an error when he (p. IX) attributes the actual basic idea of the \textit{Dedekind} theory to \textit{J. Bertrand} (Traité d'Arithmétique), as \textit{Dedekind} has rightly emphasized in the preface to his work: "Was sind und was sollen die Zahlen?" (p. XIV). \textit{Bertrand} actually uses the two classes designated in the text as $(a_1), (a_2)$ only, just like the older mathematicians, for the \textit{approximate representation} of the \textit{irrational number}; he does not at all connect its \textit{definition} to the concept of the \textit{cut}, which is foreign to him, but thoroughly to that of \textit{measurable magnitude} (see l.c. 11th ed., 1895, Art. 270, 313), and he tacitly usurps for the foundation of addition and multiplication of irrational numbers (Art. 314, 315) the \textit{axiom} of Art. 4.

}