\thispagestyle{fancy}
\fancyhead[LO]{7. Du Bois-Reymond's Fight Against Arithmetic Theories}

\vspace{0.5cm}

has been opposed\textsuperscript{32)}. In his "Allgemeine Funktionentheorie" (Tübingen 1882) he rejects it as formalistic, degrading analysis to a mere play of symbols\textsuperscript{33)}, and emphasizes for historical and philosophical reasons the inseparable connection of \textit{number} with \textit{measurable} or, as he calls it, \textit{"linear" magnitude}. In doing so, he reduces the \textit{requirement} contained in the \textit{axiom} of Art. 4 to that of the \textit{decimal fraction limit}, i.e., the \textit{existence} of a specific \textit{line segment} which (in the sense discussed more closely above in No. 3) corresponds to an arbitrarily given \textit{infinite decimal fraction}\textsuperscript{34)}. He does not regard this statement without further ado as an \textit{axiom}, but investigates to what extent it can be justified by considerations of an essentially psychological nature. The \textit{epistemological} value of this discussion\textsuperscript{35)} will be reported at a later point\textsuperscript{36)}. For the \textit{mathematician}, hardly anything else comes out of it in the end than that he must accept the requirement in question as an \textit{axiom} if he wants to base the theory of \textit{irrational numbers} on that of \textit{measurable magnitudes}. This is the standpoint which \textit{G. Ascoli} has recently emphasized as the only one appearing natural to him in contrast to the arithmetic irrational number theories\textsuperscript{37)}. Nevertheless, today the vast majority of scientific mathematicians might have

\vfill
\leftline{\rule{2in}{0.4pt}}
\vspace{0.2cm}
{
\footnotesize
32) \textit{Herm. Hankel} (Theorie der complexen Zahlensysteme) wrote as early as 1867, thus at a time when at most the \textit{Weierstrass} theory could be known to him through verbal communication, the following (l.c. p. 46): "Any attempt to treat irrational numbers formally and without the concept of magnitude must lead to highly abstruse and cumbersome artifices which, even if they could be carried out with perfect rigor, which we have just reason to doubt, would not have a higher scientific value." It appears extremely curious that precisely the creator of a \textit{purely formal theory of rational numbers} has shown so little understanding for the corresponding further development of the concept of number.

33) L.c. Art. 18.

34) This requirement is indeed sufficient, since any arbitrarily arithmetically defined irrational number can be represented as a systematic fraction with arbitrary base, see No. 9, footnote 48.

35) L.c. p. 116ff.

36) VI A2 a, 3 a.

37) Rend. Ist. Lomb. (2) 28 (1895), p. 1060. \textit{A}. formulates that axiom as follows: "If of the segments $\overline{a_1b_1}, \overline{a_2b_2}, \overline{a_3b_3},...$ each lies completely in the interior of the preceding one and $\lim_{n=\infty}\overline{a_n b_n}=0$, then there always \textit{exists one} and \textit{only one} point that lies in the interior of all these segments."

}