\thispagestyle{fancy}

\vspace{0.5cm}

joined one of the purely \textit{arithmetic} forms of definition of irrational numbers and thus agree to a \textit{separation} of \textit{pure number theory} from the \textit{actual theory of magnitude}\textsuperscript{38)}. The introduction of that \textit{axiom} becomes necessary with this view only when it is a matter of transferring the results validly existing within pure \textit{arithmetic} without its participation into \textit{spatial intuition}\textsuperscript{39)}.

\vspace{0.5cm}

\textbf{8. Perfect Arithmetization in Kronecker's Sense.} While the adherents of the just described \textit{"arithmetizing"} direction content themselves with basing the \textit{definition} of \textit{irrational numbers} and the arithmetic operations to be performed with them on the theory of \textit{rational}, thus ultimately of \textit{whole} numbers, \textit{Kronecker} has set forth a substantially higher degree of \textit{"arithmetization"} of the entire number theory (arithmetic, analysis, algebra) as a desirable and, in his opinion, also attainable goal\textsuperscript{40)}. According to this, the arithmetic disciplines should \textit{"strip off again all modifications and extensions of the concept of number} (except that of the natural number)"; in particular, therefore, \textit{irrational numbers} should be definitively banished from them. That it will ever come to that does not seem very likely to me\textsuperscript{41)}. For if one observes what \textit{Kronecker} proposes l.c. for the elimination of \textit{negative} and \textit{fractional}\textsuperscript{42)}, as well as \textit{algebraic} numbers, one gets the impression that the \textit{perfect arithmetization} of those disciplines in question would amount to \textit{dissolving} their well-tested mode of expression and symbolic language, which \textit{summarizes} extremely \textit{complicated} relations between \textit{natural} numbers in the shortest and completely precise manner, into a most extensive and cumbersome formalism.

\vfill
\leftline{\rule{2in}{0.4pt}}
\vspace{0.2cm}
{
\footnotesize
38) Cf. also \textit{Helmholtz}, Ges. Abh. 3, p. 359.

39) Cf. \textit{F. Klein}, Math. Ann. 37 (1850), p. 572.

40) J. f. Math. 101, p. 338. The catchword \textit{"Arithmetization"}, which has meanwhile become a technical term, was probably first used by \textit{K}.

41) Cf. my above-cited essay: Münch. Sitzber. 27, p. 323, footnote. Further: \textit{E. B. Christoffel}, Ann. di Mat. (2) 15 (1887), p. 253. (The content of this essay is, incidentally, essentially number-theoretical in nature.)

42) The method used by \textit{Kronecker}, based on the arithmetic concept of \textit{congruence}, is, incidentally, exactly the same as that already developed by \textit{Cauchy} for the elimination of \textit{imaginary} numbers: Exerc. d'anal. et de phys. math. 4 (1847), p. 87. Cf. I A 2, No. 3.

}