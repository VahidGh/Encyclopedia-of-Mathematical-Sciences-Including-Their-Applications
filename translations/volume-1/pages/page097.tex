\thispagestyle{fancy}
\fancyhead[LO]{9. Different Representation Forms of Irrational Numbers}

\vspace{0.5cm}

\textbf{9. Different Representation Forms of Irrational Numbers and Irrationality of Certain Representation Forms.} The simplest type of \textit{number sequences} for \textit{representing irrational numbers} are the infinite, i.e., unboundedly continuable \textit{systematic fractions}\textsuperscript{43)}. Already in \textit{Theon of Alexandria}\textsuperscript{44)} one finds a method for the approximate calculation of square roots using \textit{sexagesimal} fractions. The latter remained exclusively in use even in the Middle Ages and were only gradually displaced by \textit{decimal} fractions since the 16th century\textsuperscript{45)}. Instead of the \textit{decimal} fractions now generally common in practice, the \textit{dyadic} ones\textsuperscript{46)}, due to their extraordinary formal simplicity and special geometric intuitiveness, prove to be preferentially suitable for the purposes of analytical proof.

The \textit{non-periodic} infinite decimal fractions may be considered as the first arithmetic representation forms whose \textit{irrationality} one has explicitly recognized (on the basis of the unique representability of every \textit{rational} fraction by an always \textit{periodic} infinite decimal fraction\textsuperscript{47)}). That conversely \textit{every} irrational number is uniquely representable by an infinite decimal fraction (or systematic fraction with arbitrary base) was generally proven by \textit{Stolz}\textsuperscript{48)}.

A second fundamental representation form of irrational numbers, namely through infinite \textit{continued fractions}\textsuperscript{49)} likewise connects to the problem of square root extraction. The calculation of a square root using an unboundedly continuable \textit{regular continued fraction}\textsuperscript{50)} was first taught (admittedly only with \textit{numerical} examples) by \textit{Pietro}

\vfill
\leftline{\rule{2in}{0.4pt}}
\vspace{0.2cm}
{
\footnotesize
43) A detailed theory of these in \textit{Stolz}, Allg. Arithm. 1, p. 97 ff.

44) Around 360 A.D. \textit{M. Cantor}, 1, p. 420.

45) Cf. \textit{M. Cantor}, 2, p. 252, 565-569. \textit{Siegm. Günther}, Verm. Unters. zur Gesch. der math. Wissensch. (Leipzig 1876), p. 97 ff.

46) \textit{Leibniz} among others has particularly drawn attention to the advantages of the dyadic system: Mém. Par. 1703. (Opera omnia, Ed. Dutens, 3, p. 390.)

47) \textit{Joh. Wallisii} de Algebra Tractatus (1693), Cap. 80.

48) L.c. p. 119.

49) In truth, this representation form would have been directly indicated by the geometric origin of the irrational number as ratio of \textit{incommensurable} line segments and by the Euclidean method for establishing commensurability or incommensurability (Elem. X 2, 3). Historical development, however, has taken a different course.

50) I.e., one whose partial numerators are all = 1, whose partial denominators are natural numbers.

}
