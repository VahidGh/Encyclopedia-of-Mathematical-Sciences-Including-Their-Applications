\thispagestyle{fancy}

\vspace{0.5cm}

\textit{Cataldi}\textsuperscript{51)} who accordingly is to be regarded as the \textit{inventor of continued fractions}\textsuperscript{52)}. The \textit{purely numerical} procedure discovered by \textit{Cataldi} appears in the form of a \textit{general analytical method} in \textit{Leonhard Euler}, to whom we owe the first coherent theory of continued fractions. Already in his first treatise\textsuperscript{53)} on this subject he shows among other things the following: Every \textit{rational} fraction can be represented by a \textit{finite}, every \textit{irrational} by an \textit{infinite regular} continued fraction. In particular, the development of a \textit{square root} always yields a \textit{periodic} regular continued fraction; conversely, every convergent continued fraction of this type satisfies a \textit{quadratic} equation with integer coefficients\textsuperscript{54)}. Then the numbers $e$, $\frac{e^2 - 1}{2}$ and others are represented by continued fractions\textsuperscript{55)} at first, of course, \textit{purely numerically} (i.e., by setting approximately: $e = 2.71828182845904$). The law found on this path by mere \textit{induction} for the formation of the \textit{infinite} continued fractions is, however, then also \textit{analytically} actually proven: with this, \textit{Euler} has indeed established the \textit{irrationality} of $e$ and $e^2$ for the first time\textsuperscript{56)}.

With the help of more general continued fraction developments, \textit{Joh. Heinr. Lambert}\textsuperscript{57)} then succeeded in proving the \textit{irrationality} of $e^x$, $\tan x$ and thus also of $\log x$, $\arctan x$ for \textit{any} rational $x$, in particular\textsuperscript{58)} therefore that of $\pi$ (= $4 \arctan 1$)\textsuperscript{56)}. An abbreviation of these proofs and at the same time a generally useful aid for recognizing

\vfill
\leftline{\rule{2in}{0.4pt}}
\vspace{0.2cm}
{
\footnotesize
51) Trattato del modo brevissimo di trovare la radice quadra delli numeri. Bologna 1613.

52) Also the current \textit{notation} of continued fractions (both the usual and the more compact one, cf. footnote 338) is already found in \textit{C}., with the only difference that he writes \& instead of + (or .\& instead of $\dot{+}$) (e.g., l.c. p. 70). The assumption that already the Greeks, especially \textit{Archimedes} and \textit{Theon of Smyrna} (around 130 A.D.), had known in principle the calculation of square roots using continued fractions is based solely on conjectures. Cf. \textit{M. Cantor}, 1, p. 272, 369.

53) De fractionibus continuis. Comment. Petrop. 9 (1737), p. 98.

54) This theorem forms, as is well known, the basis of important investigations in the theory of quadratic forms (\textit{Euler, Lagrange, Legendre, Dirichlet}, see I C 2) and the numerical solution of algebraic equations (\textit{Lagrange}, see I B 3 a).

55) The notations $e$ and $\pi$ come from \textit{Euler}, cf. \textit{F. Rudio}, Archimedes, Huygens, Lambert, Legendre. Leipzig 1892, p. 53.

56) Cf. my note in the Münch. Sitzber. 1898, p. 325.

57) Hist. de l'Acad. de Berlin, Année 1761 (printed 1768), p. 265.

58) L.c. p. 297.

}