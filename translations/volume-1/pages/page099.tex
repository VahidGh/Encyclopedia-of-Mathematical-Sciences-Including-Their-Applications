\thispagestyle{fancy}
\fancyhead[LO]{9. 10. Different Representation Forms of Irrational Numbers}

\vspace{0.5cm}

theorem on the \textit{irrationality} of any infinite continued fraction:

\vspace{-0.2cm}
$$\frac{a_1 |}{| b_1} \pm \frac{a_2 |}{| b_2} \pm \cdots \pm \frac{a_\nu |}{| b_\nu} \pm\cdots ,$$
\vspace{-0.2cm}

for the case that the $\frac{a_\nu}{b_\nu}$ are ordinary proper fractions\textsuperscript{59)}. Using this theorem, \textit{Legendre} first extended the irrationality proof to $\pi^2$. Also based on it is, for example, the proof given by \textit{G. Eisenstein}\textsuperscript{60)} for the irrationality of certain series and products occurring in the theory of elliptic functions, such as:

\vspace{-0.2cm}
$$\sum_{\nu=1}^{\infty} \frac{1}{p^{\nu^2}}, \quad \sum_{\nu=1}^{\infty} \frac{(-1)^{\nu - 1}}{p^{\nu^2}}, \quad \sum_{\nu=1}^{\infty} \frac{r^{\nu}}{p^{\nu^2}}, \quad \prod_{\nu=1}^{\infty} (1 - \frac{1}{p^{\nu}})$$
\vspace{-0.2cm}

(where $p$ is a whole, $r$ a rational positive number)\textsuperscript{61)}.

\vspace{0.3cm}
\textbf{10. Continuation.} The extension of the \textit{binomial} theorem to \textit{fractional} exponents\textsuperscript{62)} taught how to represent roots of any degree by infinite series and thereby provided at the same time the first general series type of \textit{immediately} recognizable irrationality. It seems to have remained the only one of this kind for a long time. The \textit{direct} proof for the irrationality of the well-known \textit{$e$-series} that has been included in most textbooks comes only from \textit{J. Fourier}\textsuperscript{63)}. By applying a quite analogous proof method, \textit{Stern}\textsuperscript{64)} showed the irrationality of the series: $\sum_{\nu} p^{-\nu} q^{-m_\nu}$, (where $p$, $q$ are natural numbers, $(m_v)$ an unbounded sequence of natural numbers, for which $m_{\nu+1} - m_\nu$ grows to infinity with $\nu$) and: $\sum_{\nu} \pm (p_1 p_2 \cdots p_\nu)^{-1}$ ,(where $p_1$, $p_2$, $p_3$, ... is an unbounded sequence of increasing natural numbers), as well as some similar, somewhat more general series and infinite products equivalent to them.

\vfill
\leftline{\rule{2in}{0.4pt}}
\vspace{0.2cm}
{
\footnotesize
59) Eléments de géometrie (1794), Note IV. (Also reprinted in the above-cited work by \textit{Rudio} p. 161.) Cf. No. 49.

60) J. f. Math. 27 (1843), p. 193; 28 (1844), p. 39.

61) The further investigations in this direction deal essentially with the separation of irrationalities into \textit{algebraic} and \textit{transcendental}. On this (specifically also on the \textit{transcendence} of $e$ and $\pi$) see I C 2.

62) Around 1666 by \textit{Newton} (Letter to \textit{Oldenburg} of Oct. 24, 1676 — see Opuscula, Ed. Castillioneus, 1 (1644), p. 328). \textit{N}. found the theorem in question merely by induction. The first rigorous, purely elementary proof (i.e., without use of differential calculus) was given by \textit{Euler}: Nov. Comment. Petrop. 19 (1774), p. 103.

63) According to \textit{Stainville}, Mélanges d'analyse (1815), p. 339.

64) J. f. Math. 37 (1848), p. 95; 95 (1883), p. 197.

}