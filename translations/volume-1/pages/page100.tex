\thispagestyle{fancy}

\vspace{0.5cm}

\textit{W. L. Glaisher}\textsuperscript{65)} pointed out that one recognizes quite immediately the irrationality of the series considered by \textit{Eisenstein} $\sum p^{-\nu^2}$, $\sum (-1)^{\nu-1}p^{-\nu^2}$ and the more general: $\sum n_\nu \cdot p^{-m_\nu}$ (where $m_\nu$, $n_\nu$ are natural numbers satisfying certain conditions) if one interprets them as \textit{systematic} (obviously \textit{non}-periodic) \textit{fractions} with base $p$. He also proves, with the help of continued fraction developments, the irrationality of various other series, which essentially coincide with those treated by \textit{Stern}.

A unique representation of \textit{every} proper-fractional \textit{irrational number}, modeled after the exponential series, by the series $\sum_{\nu=1}^{\infty} \frac{m_\nu}{\nu!}$ (where $m_\nu$ is a natural number $< \nu$) has been given by \textit{Cyp. Stephanos}\textsuperscript{66)}; the sum of the series yields a \textit{rational} number if and only if from some specific $\nu$ onwards, throughout $m_\nu = \nu-1$. Incidentally, this representation appears only as a special case of one given earlier by \textit{G. Cantor}\textsuperscript{67)}. Another likewise unique representation of \textit{all} numbers lying between 0 and 1 by series of the form:
$$\frac{1}{m_1 + 1} + \sum_{\nu=1}^{\infty} \frac{1}{m_1(m_1 + 1) \cdots m_\nu(m_\nu + 1)}$$
comes from \textit{J. Lüroth}\textsuperscript{68)}. The \textit{rational} numbers always yield \textit{periodic}, the \textit{irrational} however \textit{non-periodic} series of this kind - \textit{vice versa}.

Finally, there belongs here also a unique representation communicated by \textit{G. Cantor}\textsuperscript{69)} of all numbers lying above 1 by infinite products of the form: $\prod_{\nu=1}^{\infty}(1 + \frac{1}{m_\nu})$, where the $m_\nu$ are natural numbers and $m_{\nu+1} \geq m_\nu^2$. Here the \textit{irrational} numbers are characterized by the fact that for infinitely many values of $\nu$: $m_{\nu+1} > m_\nu^2$, while for every \textit{rational} number, from a certain value $\nu$ onwards, throughout the relation $m_{\nu+1} = m_\nu^2$ holds\textsuperscript{70)}.

\vfill
\leftline{\rule{2in}{0.4pt}}
\vspace{0.2cm}
{
\footnotesize
65) Philosophical Magazine 45 (London 1873), p. 191.

66) Bull. de la S. M. d. F. 7 (1879), p. 81. A function-theoretical application of this representation method in \textit{G. Darboux}, Ann. de l'École norm. (2), 7 (1879), p. 200.

67) Z. f. Math. 14 (1869), p. 124.

68) Math. Ann. 21 (1883), p. 411. There \textit{L}. also gives one application each to function theory and set theory.

69) Z. f. Math. 14 (1869), p. 152.

70) A representation of \textit{special} irrationalities by infinite products,

}