\thispagestyle{fancy}
\fancyhead[LO]{11. The Geometric Origin of the Concept of Limit}

\vspace{0.5cm}

\begin{center}
    \textbf{II. Concept of Limit.}
\end{center}



\textbf{11. The Geometric Origin of the Concept of Limit.} The more general concept of \textit{limit} or \textit{limiting value} of a somehow defined, unbounded in number \textit{set of numbers}, which stands in closest relation to the concept of irrational number, has emerged from the principle of \textit{exhaustion}\textsuperscript{71)} already used by \textit{Euclid} and \textit{Archimedes} in connection with the application of the \textit{infinite} belonging only to more recent times. The \textit{exhaustion} principle appears among the ancients in the form of a purely \textit{apagogic} proof method useful for comparing surfaces and bodies, the core of which can be formulated as follows\textsuperscript{72)}: "Two geometric magnitudes $A$, $B$ are equal to each other if it can be shown that under the assumption $A > B$ the difference $A - B$, and under the assumption $A < B$ the difference $B - A$ would be smaller than any magnitude of the same kind as $A$, $B$." The conception of a spatial structure bounded by a continuously curved line or surface as a polygon or polyhedron with \textit{"infinitely many"} and \textit{"infinitely small"} sides is hardly found before the 16th century. Here too, the above-already cited \textit{M. Stifel} may well be considered the first who defined the \textit{circle} as an \textit{infinite-polygon} and, even more precisely, in a sense as the \textit{last} (thus in our terminology as the \textit{"limit"}) of all possible polygons with finite number of sides\textsuperscript{73)}. But while he concluded from this precisely the \textit{impossibility} of representing the ratio of circumference and diameter by a \textit{rational} or \textit{irrational} number\textsuperscript{73a)}, \textit{Joh. Kepler}, proceeding from analogous

\vfill
\leftline{\rule{2in}{0.4pt}}
\vspace{0.2cm}
{
\footnotesize
as the first example of which appears the well-known \textit{Wallis} formula for $\pi$ (see No. 41 Eq. [52]), \textit{Ch. A. Vandermonde} gave, Mém. de l'Acad., Paris 1772. (In the German edition of \textit{V.'s Abhandl}. aus der reinen Math. [Berlin 1888], p. 67.)

71) Cf. Art. \textit{"Exhaustion"} in \textit{Klügel's} W. B., 2, p. 152. A more critical presentation is given by \textit{Hermann Hankel} in \textit{Ersch} and \textit{Gruber's} Encyklopädie, Sect. I, Vol. 90, Art. \textit{"Grenze"}.

72) \textit{Stolz}, Zur Geometrie der Alten. Math. Ann. 22 (1883), p. 514. Allg. Arithm. I, p. 24.

73) L.c. Fol. 224a. Def. 7. 8: "Recte igitur describitur circulus mathematicus esse polygonia infinitorum laterum. \textit{Ante} circulum mathematicum sunt omnes polygoniae numerabilium laterum, quemadmodum ante numerum infinitum sunt omnes numeri dabiles."

73a) L.c. Fol. 224b. Def. 12. If one considers that \textit{Stifel} did not yet have the \textit{general} concept of irrational number (cf. No. 2), the above apparently false conclusion may be regarded not only as perfectly logical, but even as a characteristic sign of the (approaching modern conception) arithmetically-

}