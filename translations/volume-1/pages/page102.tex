\thispagestyle{fancy}

\vspace{0.5cm}

views, arrived at useful formulas for the \textit{cubature} of bodies of rotation\textsuperscript{74)}. Shortly thereafter appeared \textit{Bonav. Cavalieri's Geometry of Indivisibles}\textsuperscript{75)}, which, going considerably beyond \textit{Kepler's} special investigations, notwithstanding the somewhat mystical nature of those "indivisibles", is usually regarded as the first fundamental presentation of \textit{a general scientific method of exhaustion}\textsuperscript{76)}.

\vspace{0.5cm}

\textbf{12. The Arithmetization of the Concept of Limit.} \textit{John Wallis}\textsuperscript{77)} arrived at an \textit{arithmetic} formulation of the \textit{concept of limit}, as it is essentially still common today, by abandoning the cumbersome \textit{apagogic} procedure of the ancients and translating \textit{Cavalieri's direct geometric} method into the \textit{arithmetic} - in meaning and in today's expression approximately as follows:

A number $a$ is considered as the \textit{limit} of an unboundedly continuable sequence of numbers $a_\nu$ ($\nu = 0, 1, 2, \ldots$ in inf.), if the difference $a - a_\nu$ becomes \textit{arbitrarily small}\textsuperscript{78)} with sufficiently increasing values of $\nu$.

This \textit{definition}, which completely fixes the \textit{arithmetic} relation of that \textit{limit} $a$ to the numbers $a_\nu$ as soon as the number $a$ is \textit{known} or at least its \textit{existence} is established from the outset, \textit{does not} yet provide a \textit{criterion} to possibly infer the \textit{existence} of a limit from the nature of the numbers $a_\nu$. In this respect, one repeatedly took refuge in \textit{geometric} ideas and analogies, from which one then believed to be able to infer without further ado the \textit{existence} of the limit in question\textsuperscript{79)}. Thus, for example, in the \textit{quadrature} of curvilinearly bounded 

\vfill
\leftline{\rule{2in}{0.4pt}}
\vspace{0.2cm}
{
\footnotesize
That conclusion agrees perfectly with our current view that the \textit{rectification} of a curved line cannot be defined at all \textit{without} the \textit{general concept} of irrational number. Cf. No. 11, 12.

74) Nova stereometria doliorum. Linz 1615. (Cf. \textit{M. Chasles}, Histoire de la Géométrie (2de éd. 1875), p. 56. \textit{M. Cantor}, Gesch. der Math. 2, p. 750.)

75) Geometria indivisibilibus continuorum nova quadam ratione promota. Bologna 1635. (Details in \textit{Klügel}, 1, Art. \textit{"Cavalieri's Method des Untheilbaren"}. \textit{M. Cantor} l.c. p. 759.)

76) \textit{H. Hankel} l.c. p. 189. \textit{Chasles} l.c. p. 57.

77) Arithmetica Infinitorum (1655), Prop. 43, Lemma. (In the complete edition of \textit{W}.'s works - Opera omnia, Oxoniae, 1695. 1, p. 383.) Cf. \textit{M. Cantor}, 2, p. 823.

78) "Quolibet assignabili minor." L.c. Prop. 40.

79) I deliberately omit here again all attempts to establish the concept of limit epistemologically and psychologically, as belonging to the \textit{philosophy} of mathematics (thus according to VI A 2 a, 3 a).

}
