\thispagestyle{fancy}
\fancyhead[LO]{13. The Criterion for the Existence of a Limit Value}

\vspace{0.5cm}

plane pieces, in the \textit{rectification} of curve arcs (with the help of the quadrature or rectification of a series of unboundedly approximated polygons), one regarded the \textit{existence} of a definite \textit{area or length number} as something \textit{self-evident}, existing \textit{a priori} on the basis of \textit{geometric} intuition\textsuperscript{80)}. The \textit{decisive turn} toward eliminating this inadequate conception is marked by \textit{Cauchy's} definition and existence proof\textsuperscript{81)} for the \textit{definite integral} of a continuous function; with this, indeed, not only is the \textit{necessity} made clear for the first time to explicitly prove \textit{arithmetically} the \textit{existence} of an \textit{area number}, but this proof is actually delivered \textit{at least in the main part}, i.e., it is shown that for the definition of that \textit{area number}, sequences of numbers are available which fulfill the \textit{criterion} required for the \textit{existence} of a definite \textit{limit} (to be discussed more closely immediately)\textsuperscript{82)}. Although \textit{Cauchy} lacks (and indeed not only at the relevant point, but generally in his works) the \textit{proof} that this \textit{criterion} is actually \textit{sufficient} for the existence of a definite \textit{limit}, one can nevertheless say that through \textit{Cauchy's} mentioned achievement, the true \textit{arithmetic} nature of the general limit problem has been sharply characterized for the first time and the way has been shown for its final resolution.

\vspace{0.5cm}

\textbf{13. The Criterion for the Existence of a Limit Value.} The mentioned \textit{criterion} for the existence of a definite \textit{limit}, in its \textit{basic form}, i.e., for a simple, unboundedly continuable series of real numbers (simply-infinite \textit{sequence of numbers}, simply-\textit{countable}\textsuperscript{83)} set of numbers) and in connection

\vfill
\leftline{\rule{2in}{0.4pt}}
\vspace{0.2cm}
{
\footnotesize
80) In stereometry, the analogous difficulty already arises with the cubature of the \textit{pyramid}; cf. \textit{R. Baltzer}, Die Elemente der Mathematik 2 (1883), p. 229. \textit{Stolz}, Math. Ann. 22 (1883), p. 517.

81) Both are already found in the "Résumé des leçons données à l'école polytechnique sur le calcul infinitésimal" (Paris 1823), p. 81 (not first, as is often assumed, in the "Leçons sur le calcul différentiel et intégral" 2, p. 2, published by \textit{M. Moigno} 1840-44).

82) For the full rigor of the proof, the recognition of the \textit{uniform} continuity of a simply continuous function would still be required - which, however, does not weigh essentially in the context at hand. Cf. II A 1.

83) Cf. I A 5, No. 2. In the present article, essentially only the \textit{limit values of countable sets of numbers} are dealt with, since the limit values of \textit{uncountable}, especially \textit{continuous} sets of numbers (cf. I A 5, No. 2, 13, 16) belong to \textit{analysis} (II A, B). Of course, this separation cannot be strictly maintained with regard to the historical development of the different limit value considerations cannot always be strictly maintained (as e.g. above

}