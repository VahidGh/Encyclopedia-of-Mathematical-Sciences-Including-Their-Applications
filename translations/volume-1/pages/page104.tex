\thispagestyle{fancy}

\vspace{0.5cm}

with the \textit{definition of limit} given above, is as follows:

For the unbounded sequence of numbers $(a_\nu)$ to possess a definite \textit{limit} (a definite \textit{limit value} or \textit{limes}) $a$, in symbols\textsuperscript{84)}:

$$a = \lim a_\nu \quad (\nu=\infty) \quad \text{or:} \quad \lim_{\nu=\infty} a_\nu = a$$

it is \textit{necessary} and \textit{sufficient} that $a_{n+\varrho} - a_n$ becomes \textit{arbitrarily small} for \textit{a sufficiently large} value of $n$ and \textit{every} value of $\varrho$\textsuperscript{85)}.

The sequence of numbers $(a_\nu)$ is then called \textit{convergent}.

That the above condition is a \textit{necessary} one follows directly from the \textit{definition} of the \textit{limit} and may well have been known since one has dealt with such limit values at all. That it is also \textit{sufficient} was regarded as \textit{self-evident} until recent times, but \textit{was never explicitly proven}. The merit of having first emphasized this necessity belongs to \textit{Bolzano}\textsuperscript{86)}, who at least \textit{attempted} to provide the corresponding proof for the special case of series convergence\textsuperscript{87)}. This is, however, inadequate, as is also a proof given by \textit{Herm. Hankel} (relating to the more general case of \textit{arbitrary} sets of numbers)\textsuperscript{88)}.

\vfill
\leftline{\rule{2in}{0.4pt}}
\vspace{0.1cm}
{
\footnotesize
regarding the citation about the definite integral or in the following remarks about the proof of the limit value criterion).

84) The \textit{symbol $\lim$}, which has become completely indispensable to us today, seems to me to have been first used by \textit{Simon L'Huilier} (Exposition élément. des calculs supérieurs, Berlin 1786 - also under the title: Principiorum calc. diff. et integr. expositio, Tübingen 1795). It probably became generally common only since \textit{Cauchy} (Anal. algébr. p. 13) (i.e., since 1821; in the great Traité de calc. diff. et integr. by \textit{Lacroix}, 1810-1819, each individual limit transition is still laboriously designated with words). The above-mentioned work of \textit{L'Huilier} (awarded a prize by the Berlin Academy as the solution to a prize question posed in 1784), which in the historical presentations known to me is by no means appreciated according to merit, contains the first rigorous presentation of the concept of limit based on the Euclidean theory of proportions and the method of exhaustion.

85) This theorem with its transfer to \textit{arbitrary} (e.g., continuous) sets of numbers - designated by \textit{Du Bois-Reymond} as the \textit{"general convergence principle"} (Allg. Funct.-Theorie, pp. 6, 260) is the actual \textit{fundamental theorem of the entire analysis} and should stand at the head of every rational textbook of analysis with sufficient emphasis on its fundamental character.

86) "Rein analytischer Beweis des Lehrsatzes, etc." Prague 1817. Cf. \textit{Stolz}, Math. Ann. 18 (1881), p. 259.

87) Cf. also No. 21 of this article.

88) \textit{Ersch} u. \textit{Gruber} l.c. p. 193; Math. Ann. 20 (1882), p. 106. Cf. \textit{Stolz} l.c. p. 260, footnote.

}