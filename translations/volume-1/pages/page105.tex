\thispagestyle{fancy}
\fancyhead[LO]{14. The Infinitely Large and Infinitely Small}

\vspace{0.5cm}

Since the correctness of the theorem in question depends essentially and exclusively on the \textit{well-defined existence of irrational numbers}, the first rigorous proofs of it naturally coincide with the emergence of the \textit{arithmetic} theories of irrational numbers and the associated revision and sharpening of the older \textit{geometrizing} mode of explanation (cf. No. 7). In \textit{Cantor}, the theorem appears as a quite direct \textit{consequence} resulting from the \textit{definition} of irrational numbers, as is sharply emphasized by himself \textsuperscript{89)}. \textit{Dedekind} has also, in connection with \textit{his} theory of irrational numbers, provided a complete proof of it (for the more general case of \textit{arbitrary} sets of numbers)\textsuperscript{90)}. The latter has been somewhat simplified by \textit{U. Dini}\textsuperscript{91)} and then adapted by \textit{Du Bois-Reymond} to the view he represents\textsuperscript{92)}. Other modifications of that proof have been given by \textit{Stolz}, \textit{J. Tannery}, \textit{C. Jordan}\textsuperscript{93)} and \textit{P. Mansion}\textsuperscript{94)}.

If the sequence of numbers $(a_\nu)$ is \textit{monotonic}\textsuperscript{95)}, i.e., never \textit{de-} or never \textit{in}creasing, then for its \textit{convergence} the condition suffices that the $a_\nu$ remain numerically below a fixed number (example: the systematic fractions). One can also take this \textit{simplest} form of convergent sequences of numbers as the starting point for the theory of irrational numbers and limit values\textsuperscript{96)}. But then, in order to be able to define subtraction and division, one always needs \textit{two} such sequences (one never \textit{de-} and one never \textit{in}creasing)\textsuperscript{97)}.

\vspace{0.1cm}

\textbf{14. The Infinitely Large and Infinitely Small.} If the terms of an unbounded sequence of well-defined numbers $(a_\nu)$ have the property that, no matter how large a positive number $G$ is prescribed, from a certain index $\nu$ onwards throughout: $a_\nu > G$ (or $a_\nu < -G$), then one says that the \textit{limit value}

\vspace{-0.2cm}
\leftline{\rule{2in}{0.4pt}}
\vspace{0.1cm}
{
\footnotesize
89) Math. Ann. 21 (1883), p. 124.

90) L.c. p. 30.

91) Fondamenti per la teorica etc. p. 27.

92) Allg. Funct.-Theorie p. 260.

93) Cf. my remarks in the Münch. Sitzber. 27 (1897), pp. 357, 358. - \textit{Stolz} has also proven the existence of the limit value through its representation in systematic form: Allg. Arithm. 1, p. 115 ff. (cf. No. 9 of this article).

94) Mathesis 5 (1885), p. 270.

95) This expression comes from \textit{C. Neumann}: "Über die nach Kreis-, Kugel- und Cylinder-Funct. fortschr. Entw.", Leipzig 1881, p. 26.

96) Cf. \textit{Mansion}, Mathesis 5, p. 193.

97) \textit{Bachmann} l.c. pp. 12, 13. Cf. No. 4, Note 20. If one chooses even more special sequences of numbers for the definition of irrational numbers, e.g., the systematic fractions, a difficulty already arises in the definition of addition and multiplication

}