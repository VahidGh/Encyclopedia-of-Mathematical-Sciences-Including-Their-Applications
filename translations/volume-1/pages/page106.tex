\thispagestyle{fancy}

\vspace{0.5cm}

of the $a_\nu$ is positive (negative) \textit{infinity}, in symbols:

\vspace{-0.2cm}
$$\lim_{\nu=\infty} a_\nu = +\infty \quad \text{(or} \quad \lim_{\nu=\infty} a_\nu = -\infty\text{).}$$

The sequence of numbers $(a_\nu)$ is then called \textit{properly divergent}.

This statement, according to today's conception, is to be regarded as a \textit{definition of the infinite}\textsuperscript{98)}, while older analysts used to view it as a provable \textit{theorem}\textsuperscript{99)}; in truth, however, any such proof would have to amount to a mere \textit{circular reasoning} as long as no other \textit{mathematically tangible definition} of the \textit{infinite} existed\textsuperscript{100)} (which has been the case only since very recent times - see somewhat further below).

Based on the definition given above, among the numbers $a_\nu$, no matter how large $\nu$ may be assumed, \textit{none is infinitely large}; nevertheless, one uses the \textit{expression}: the numbers $a_\nu$ \textit{become infinitely large} with \textit{unboundedly} increasing values of $\nu$. The \textit{infinite}, which in this form of definition appears merely as a \textit{variably-finite}, thus as a \textit{becoming}, not a \textit{become}, is designated as \textit{potential}\textsuperscript{101)} or \textit{improper}\textsuperscript{102)} \textit{infinite}.

But also \textit{independently} of any such \textit{process of becoming}, the \textit{infinite} can be strictly arithmetically defined as an \textit{actual} or \textit{proper infinite}. \textit{B. Bolzano}\textsuperscript{103)} has emphasized as a peculiar characteristic of an \textit{infinite} set of elements that the elements which form merely a certain \textit{part} of that set can be \textit{uniquely-invertibly} assigned to the elements of the \textit{total} set (e.g., to the \textit{total} set of numbers $0 \leq y \leq 12$ the \textit{partial} set of numbers $0 \leq x \leq 5$ on the basis of the stipulation: $5y = 12x$). \textit{G. Cantor} has formulated the same property to the effect that in an \textit{infinite} set, and only in such a set, a \textit{part} of the set can possess the \textit{same cardinality} as itself\textsuperscript{104)}.

\vspace{-0.1cm}
\leftline{\rule{2in}{0.4pt}}
\vspace{0.2cm}
{
\footnotesize
98) Approximately since \textit{Cauchy}: Analyse algébr. pp. 4, 27.

99) See e.g. \textit{Jac. Bernoulli}, Positiones arithmeticae de seriebus infinitis (1689), Prop. II (Opera, Genevae 1744, 1, p. 379).

100) Also what e.g. \textit{DuBois-Reymond} says in his Allg. Functionen-Theorie p. 69 ff. about the distinction of the \textit{"infinite"} from the \textit{"unbounded"} appears untenable. Cf. my remarks Münch. Sitzber. 1897, p. 322, footnote 1.

101) The \textit{Infinitum potentia} or \textit{syncategorematic infinite} of the philosophers, in contrast to the \textit{Infinitum actu} or \textit{categorematic} (actual) \textit{infinite} to be mentioned immediately.

102) According to \textit{G. Cantor}, Math. Ann. 21 (1883), p. 546.

103) Paradoxien des Unendlichen. Leipzig 1851. § 20.

104) Journ. f. Math. 84 (1878), p. 242.

}