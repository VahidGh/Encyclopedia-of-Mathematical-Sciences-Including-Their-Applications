\thispagestyle{fancy}
\fancyhead[LO]{14. The Infinitely Large and Infinitely Small}

\vspace{0.5cm}

Independent of the two mentioned\textsuperscript{105)}, \textit{Dedekind} has actually elevated this property to the \textit{definition} of the \textit{infinite}, i.e. (retaining the \textit{Cantorian} terminology just used): A set is called \textit{infinite} if it contains a \textit{sub}set of \textit{equal cardinality}; in the opposite case, it is called \textit{finite}\textsuperscript{106)}. \textit{Dedekind} then \textit{proves} the \textit{existence of infinite} sets\textsuperscript{107)}, derives from this the concept of the \textit{natural sequence of numbers} and finally that of the \textit{number} of a \textit{finite} set.

Conversely, \textit{G. Cantor}, regarding the concept of \textit{number} for \textit{finite} sets in the usual way as something given \textit{a priori}, has transferred this concept to \textit{infinite} sets and has thereby been led to the establishment of a consistently developed \textit{system of properly-infinite} (\textit{"supra-finite"} or \textit{"transfinite"}) numbers\textsuperscript{108)}.

In \textit{arithmetic}, the \textit{infinite} always appears only as \textit{improperly-infinite}, thus as \textit{variably-finite} whose absolute value is not bound to any upper limit. In \textit{function theory}, especially for complex variables, it has nevertheless proven expedient to introduce, besides this \textit{improperly-infinite}, also a \textit{properly-infinite} in such a way that to all possible \textit{finite} values of which a variable is capable, \textit{the value $\infty$ is added like a single, definite one} (geometrically represented by a definite point)\textsuperscript{109)}.

\vfill
\leftline{\rule{2in}{0.4pt}}
\vspace{0.2cm}
{
\footnotesize
105) Cf. the preface to the 2nd edition of the work: Was sind und was sollen die Zahlen? Braunschweig 1895. (First edition 1887.)

106) L.c. No. 64. \textit{D}. thereby designates two sets of \textit{"equal cardinality"} (thus those whose elements can be uniquely-invertibly assigned to each other) as \textit{"similar"} (or also more explicitly as such that can be similarly mapped into each other). Another, in a certain respect even simpler definition of the infinite is given by \textit{D}. in the above-cited preface, p. XVII. Cf. also \textit{Franz Meyer}, Zur Lehre vom Unendlichen. Antr.-Rede, Tübingen 1889; \textit{C. Cram}, Wundts Philos. Studien 21 (1895), p. 1; \textit{E. Schröder}, Nova acta Leop. 71 (1898), p. 303.

107) Similarly, as already \textit{Bolzano} l.c. § 13.

108) Math. Ann. 21 (1883), p. 545 ff. The relevant treatise also contains a historical-critical discussion of the concept of infinity provided with numerous citations. More on \textit{transfinite} numbers see I A 5, No. 3 ff.

109) This proper \textit{infinite} of function theory can by no means always be replaced without further ado by the \textit{improper infinite}; in other words: the behavior of a function $f(x)$ \textit{for all possible arbitrarily large} values of $x$ need by no means yet determine that for the value $x = \infty$. If one sets e.g. 

$$f(x) = \lim_{n=\infty} f_n(x), \quad  where: f_n(x) = \frac{n}{n+x}$$

}