\thispagestyle{fancy}

\vspace{0.5cm}

The so-called \textit{infinitely small} behaves somewhat differently. If $\lim a_\nu = 0$, one often uses the \textit{expression}: the numbers $a_\nu$ \textit{become} \textit{infinitely small} with \textit{unboundedly} increasing values of $\nu$\textsuperscript{110)}. Wherever in \textit{arithmetic}, \textit{function theory}, \textit{geometry} the so-called \textit{infinitely small} appears, it is always only a \textit{becoming infinitely small}, thus according to the terminology used above an \textit{improperly-infinitely small}\textsuperscript{111)}. Although it has recently been possible to establish self-consistent systems of \textit{properly-infinitely small "magnitudes"}\textsuperscript{112)}, these are merely \textit{systems of symbols} with \textit{purely formally} defined laws, which partly differ from those valid for real numbers. Such fictitious \textit{properly-infinitely small magnitudes} have no direct relation to \textit{real numbers}; they find no place in proper \textit{arithmetic} and \textit{analysis} and cannot, like real numbers, serve to \textit{describe} \textit{geometric magnitude} relations without contradiction. In particular, from the possibility of such arithmetic constructions, the \textit{existence of infinitely small geometric magnitudes} (e.g., line elements) \textit{cannot} be inferred. \textit{G. Cantor} has rather explicitly shown that from the assumption of numbers that are numerically \textit{smaller} than \textit{any} positive number, precisely the \textit{non-existence of infinitely small line segments} can be inferred\textsuperscript{113)}.

\vspace{0.3cm}

\textbf{15. Upper and Lower Limits.} From an \textit{improperly divergent}, i.e.

\vspace{0.1cm}
\leftline{\rule{2in}{0.4pt}}
\vspace{0.1cm}
{
\footnotesize
so one has for \textit{every arbitrarily large} finite $x$ without exception:

\vspace{-0.3cm}
$$f(x) = 1,$$
\vspace{-0.5cm}

whereas: 

\vspace{-0.4cm}
$$f_n(\infty) = 0, \quad thus \; also: \quad f(\infty) = 0.$$
\vspace{-0.3cm}

In general, the behavior of a function $f(x)$ for that \textit{value} or \textit{point} $x = \infty$ is defined by that of $f(\frac{1}{x})$ for $x = 0$. Cf. II B 1. Another type of the properly-infinite ("the infinitely distant line") has proven expedient in projective \textit{geometry}.

110) \textit{Cauchy}, Anal. algébr. p. 4, 26.

111) With the \textit{properly-infinite} $x = \infty$ of function theory corresponds \textit{not} a \textit{properly-infinitely} small value $x$, but the value $x = 0$.

112) \textit{O. Stolz} has, using \textit{Du Bois-Reymond's} investigations, constructed two different systems of properly-infinitely small magnitudes: Ber. d. naturw.-medic. Vereins, Innsbruck 1884, p. 1 ff. 37 ff. Allg. Arithm. 1, p. 205 ff. Cf. I A 5, No. 17. On \textit{P. Veronese's} \textit{"Infiniti und Infinitesimi attuali"} cf. I A 5, footnote 103, 107. - A detailed historical-critical presentation of the theory of infinitely small magnitudes is given in \textit{G. Vivanti's} work: Il concetto d'infinitesimo, Mantova 1894.

113) Z. f. Philos. 91, p. 112. Cf. also \textit{O. Stolz}, Math. Ann. 31 (1888), p. 601. \textit{G. Peano}, Rivista di Mat. 2 (1872), p. 58.

}