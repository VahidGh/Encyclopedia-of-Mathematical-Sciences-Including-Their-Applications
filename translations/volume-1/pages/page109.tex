\thispagestyle{fancy}
\fancyhead[LO]{15. Upper and Lower Limits}

\vspace{0.5cm}

neither convergent nor properly divergent sequence of numbers $(a_\nu)$, \textit{two convergent} or \textit{properly divergent} sequences of numbers $(a_{m_\nu})$, $(a_{n_\nu})$ of the following nature can always be extracted:

If one sets
\vspace{-0.5cm}
\begin{align}
    \lim_{\nu=\infty} a_{m_\nu} = A, \quad \lim_{\nu=\infty} a_{n_\nu} = a,
\end{align}
\vspace{-0.5cm}

(where $A > a$ and $A$, $a$ either represent definite numbers or can also be $A = +\infty$, $a = -\infty$), then from the sequence $(a_\nu)$ \textit{no} sequence can be extracted which possesses a \textit{greater limit} than $A$ or a \textit{smaller limit} than $a$. $A$ is accordingly called the \textit{greatest} or \textit{upper}, $a$ the \textit{smallest} or \textit{lower limit} of the $a_\nu$, in symbols\textsuperscript{114)}:
\vspace{-0.5cm}
\begin{align}
    \lim_{\nu=\infty} \sup a_\nu = A, \quad \lim_{\nu=\infty} \inf a_\nu = a,
\end{align}
\vspace{-0.5cm}

or more briefly\textsuperscript{115)}:
\vspace{-0.5cm}
\begin{align}
    \overline{\lim_{\nu=\infty}} a_\nu = A, \quad \underline{\lim_{\nu=\infty}} a_\nu = a.
\end{align}
\vspace{-0.5cm}

By means of this generalization of the concept of limit, the \textit{convergent} and \textit{properly divergent} sequences of numbers appear as that limiting case in which \textit{upper and lower limits coincide}, so that:

\vspace{-0.5cm}
\begin{align}
    \overline{\lim_{\nu=\infty}} a_\nu = \underline{\lim_{\nu=\infty}} a_\nu = \lim_{\nu=\infty} a_\nu.
\end{align}
\vspace{-0.5cm}

The concept of the \textit{upper and lower limits} is already found by \textit{Cauchy}\textsuperscript{116)}, who has made an extremely important application of it specifically in the theory of series\textsuperscript{117)}. \textit{Du Bois-Reymond} has introduced for the \textit{upper and lower limits} the designation \textit{upper and lower indeterminacy bounds}\textsuperscript{118)} and is therefore often falsely regarded as the \textit{inventor} of the \textit{concept} thus designated. Nevertheless, one can say that he was the first to explicitly emphasize the great and general significance of that concept for the theory of series and functions and to give occasion for its consistent application\textsuperscript{119)}.

\vfill
\leftline{\rule{2in}{0.4pt}}
\vspace{0.2cm}
{
\footnotesize
114) According to \textit{Pasch}, Math. Ann. 30 (1887), p. 134.

115) According to a notation recently introduced by me: Münch. Sitzber. 28 (1898), p. 62. The occasional use of the notation $\overline{\underline{\lim_{\nu=\infty}}} a_\nu$ is meant to indicate that in the relevant context \textit{either the upper or the lower limit} may be taken.

116) Anal. algébr. p. 132, 151 etc. "la plus grande des limites". \textit{C}. designates the upper limit as "la limite vers laquelle tend la plus grande valeur". "La plus petite des limites" in \textit{N. H. Abel}: Oeuvres 2, p. 198.

117) Cf. No. 23.

118) Antritts-Progr. d. Univ. Freiburg (1871), p. 3. Münch. Abh. 12, I. Abth. (1876), p. 125. Allg. Funct.-Th. p. 266.

119) Cf. especially the above-cited essay by \textit{Pasch}.

}