\thispagestyle{fancy}

\vspace{0.5cm}

\textbf{16. Upper and Lower Bounds.} Related to the concept of the \textit{upper (lower) limit}, yet to be sharply distinguished from it, is the concept of the \textit{upper (lower) bound}\textsuperscript{122)} first noted by \textit{Bolzano}\textsuperscript{120)}, but especially emphasized by \textit{Weierstrass} (in his lectures)\textsuperscript{121)}: Every sequence of numbers $(a_\nu)$ with terms remaining finite (i.e., contained between two definite numbers) possesses a definite \textit{upper} and \textit{lower bound} $G$, $g$, i.e., one has for \textit{every} $\nu$: $g \leq a_\nu \leq G$ and for at least one value each $\nu = m$, $\nu = n$: $G - \varepsilon < a_m \leq G, \quad g \leq a_n < g + \varepsilon \quad \text{for arbitrarily small prescribed positive } \varepsilon.$ If there is a term $a_m = G$ (possibly also several or even infinitely many), then the \textit{upper bound} of the $a_\nu$ is also called their\textit{maximum}\textsuperscript{123)}. If there is \textit{no} such term, then there must be infinitely many terms $a_{m_\nu}$ for which: $G - \varepsilon < a_{m_\nu} < G$, i.e., in this case the \textit{upper bound} $G$ is simultaneously the \textit{upper limit} of the $a_\nu$. This obviously also occurs when for \textit{infinitely many} values of $\nu$ the relation $a_\nu = G$ holds.

The analogous applies regarding the lower bound $g$.

If the $a_\nu$ \textit{do not} remain below a certain \textit{positive} or above a certain \textit{negative} number, then $G = +\infty$ or $g = -\infty$. Also in this case, the upper or lower \textit{bound} appears simultaneously as the upper or lower \textit{limit}\textsuperscript{124)}.

\vfill
\leftline{\rule{2in}{0.4pt}}
\vspace{0.2cm}
{
\footnotesize
120) Beweis des Lehrs. etc. p. 41. Cf. \textit{Stolz}, Math. Ann. 18 (1881), p. 257.

121) \textit{Pincherle} l.c. p. 242 ff.

122) \textit{Pasch} l.c. designates what is here (following \textit{Weierstrass}) called upper (lower) \textit{bound} as upper (lower) \textit{barrier}, and uses the expression upper (lower) \textit{bound} for the upper (lower) \textit{limit}. French (and Italian) authors tend to use the expression \textit{limite supérieure} (\textit{limite superiore}) etc. sometimes in one sense, sometimes in the other, which can easily give rise to ambiguities.

123) \textit{Darboux} (Ann. de l'école norm. (2) 4, p. 61) calls the upper (lower) bound: \textit{"la limite maximum (minimum)"} a designation that should not be confused with\textit{ maximum (minimum)}. In the case $a_m = G$, I tend to designate the upper bound even more concisely as the \textit{real maximum} of the $a_\nu$, and call it their \textit{ideal maximum} if \textit{no} term reaches the upper bound $G$ (an assumption that also encompasses the case $G = \infty$). Then one can say: The upper \textit{limit} coincides with the upper \textit{bound} if and only if the latter is an \textit{ideal} or \textit{infinitely often occurring real maximum}. Analogously for the lower bound.

124) \textit{G. Peano} has pointed out that in certain cases (e.g., def. of the definite integral, of rectification, etc.) one operates more easily and precisely with the concept of the upper (lower) \textit{bound} than with that of the \textit{limit}: Ann. di Mat. (2), 23 (1895), p. 153.

}