\thispagestyle{fancy}
\fancyhead[LO]{17. Operations with Limit Values}

\vspace{0.5cm}

\textbf{17. Operations with Limit Values. The number $e = \lim (1 + \frac{1}{\nu})^\nu$.}

If $(a_\nu)$, $(b_\nu)$ are convergent sequences of numbers, then the elementary rules of calculation to be directly connected to the definition of irrational numbers yield the relations:

\vspace{-0.5cm}
\begin{align}
    \left.\begin{tabular}{l}
    $\lim a_\nu \pm \lim b_\nu = \lim (a_\nu \pm b_\nu), \quad \lim a_\nu \cdot \lim b_\nu = \lim (a_\nu b_\nu),$ \\
    $\displaystyle\frac{\lim a_\nu}{\lim b_\nu} = \lim (\frac{a_\nu}{b_\nu})$ \textsuperscript{125)}
    \end{tabular}\right\}
\end{align}


(where in the last equation the case $\lim b_\nu = 0$ is to be excluded), and in general:

\vspace{-0.5cm}
\begin{align}
f(\lim a_\nu, \lim b_\nu, \lim c_\nu, \ldots) = \lim f(a_\nu, b_\nu, c_\nu, \ldots),
\end{align}

when $f$ denotes any combination of the 4 operations (excluding division by 0).

If the calculation symbol $f$ contains other requirements, e.g., extraction of roots, then eq. (6) is valid as a \textit{definition} equation, provided the right side \textit{converges}. With the help of this principle, in particular the theory of fractional and irrational powers and their inversions, the logarithms, can be consistently and rigorously founded\textsuperscript{126)}.

The distinguished arithmetic properties which the \textit{natural} logarithms (i.e., those with base $e$) have over all others are based on the relations:

\vspace{-0.5cm}
\begin{align}
\lim (1 + \frac{1}{\nu})^\nu = e, \quad \lim (1 + \frac{a}{\nu})^\nu = e^a
\end{align}

($a$ an arbitrary real number). While the latter appear in \textit{Euler}\textsuperscript{127)} only in the context that the equality of the limit values on the left with the series serving to \textit{define} $e, e^a$ is derived (in a manner certainly inadequate by today's concepts), \textit{Cauchy}\textsuperscript{128)} has directly proven the \textit{existence} of those limit values and \textit{based} on them the \textit{definition} of exponential quantities and natural logarithms - a method that has since passed into most textbooks of analysis\textsuperscript{129)}.

\vfill
\leftline{\rule{2in}{0.4pt}}
\vspace{0.2cm}
{
\footnotesize
125) From now on, I always write just $\lim$ instead of $\lim_{\nu=\infty}$, as far as a misunderstanding seems excluded.

126) Cf. \textit{Stolz}, Allg. Arithm. p. 125-148.

127) Introductio in anal. inf. 1 § 115-122.

128) Résumé des leçons etc. (1823), p. 2.

129) Thereby the definition of the power with arbitrary (possibly irrational) exponents is usually presupposed as already known.

}