\thispagestyle{fancy}
\fancyhead[LO]{19. Gradation of Becoming Infinite and Zero}

\vspace{0.5cm}

\textbf{19. Gradation of Becoming Infinite and Zero.} On the investigation of quotients of the form $\frac{\infty}{\infty}$ (i.e., of $\lim \frac{a_\nu}{b_\nu}$ where $\lim a_\nu = \infty, \lim b_\nu = \infty$) is based the \textit{gradation} of becoming infinite of sequences of numbers (or of functions). If $\lim a_\nu = +\infty$, $\lim b_\nu = +\infty$, then, \textit{if} $\lim \frac{a_\nu}{b_\nu}$ \textit{exists at all}\textsuperscript{135)}, the following three cases are to be distinguished:

\vspace{-0.5cm}
\begin{align}
    \lim \frac{a_\nu}{b_\nu} = 0, \quad \lim \frac{a_\nu}{b_\nu} = g > 0, \quad \lim \frac{a_\nu}{b_\nu} = \infty,
\end{align}
\vspace{-0.5cm}

for which \textit{Du Bois-Reymond} has introduced the notations\textsuperscript{136)}

\vspace{-0.7cm}
\begin{align}
    a_\nu \prec b_\nu, \quad a_\nu \sim b_\nu, \quad a_\nu \succ b_\nu,
\end{align}
\vspace{-0.7cm}

in words:

$a_\nu$ becomes infinite of lower order (weaker, slower) than $b_\nu$

$a_\nu$ becomes infinite of the same order (equally) as $b_\nu$

$a_\nu$ becomes infinite of higher order (stronger, faster) than $b_\nu$:

or more briefly:

\vspace{-0.5cm}
\begin{center}
    $a_\nu$ is \textit{infinitarily} smaller than $b_\nu$
    
    $a_\nu$ is \textit{infinitarily} equal to $b_\nu$
    
    $a_\nu$ is \textit{infinitarily} greater than $b_\nu$
\end{center}

I am accustomed to apply the notation $a_\nu \sim b_\nu$ and the corresponding expression also when it is only established that $\underline{\lim} \frac{a_\nu}{b_\nu}$ and $\overline{\lim} \frac{a_\nu}{b_\nu}$ are \textit{both} finite and different from zero (thus: 

\vspace{-0.3cm}
$$0 < g \le \underline{\overline{\lim}} \frac{a_\nu}{b_\nu} \le G < \infty)$$

and have added to the above notations also the following\textsuperscript{137)}:

\vspace{-0.5cm}
\begin{align}
    \quad a_\nu \, \underline{\underline{\sim}} \, g \cdot b_\nu, \quad if: \quad \lim \frac{a_\nu}{b_\nu} = g. 
\end{align}

If $(M_\nu)$ is monotonically increasing, $\lim M_\nu = \infty$\textsuperscript{138)}, then one has:

\vspace{-0.5cm}
\begin{align}
    \cdots \prec (\log_2 M_\nu)^{p"} \prec (\log_2 M_\nu)^{p'} \prec M_\nu^p  \prec (e^{M_\nu})^{p_1} \prec (e^{e^{M_\nu}})^{p_2} \prec \cdots,
\end{align}


\vfill
\leftline{\rule{2in}{0.4pt}}
\vspace{0.2cm}
{
\footnotesize
135) This need not even be the case when $a_\nu$, $b_\nu$ are \textit{both monotonic}, see e.g. \textit{Stolz}, Math. Ann. 14, p. 232 and cf. No. 29, 30 of this article.

136) Ann. di Mat. Ser. II 4 (1870), p. 339. The development and utilization of the algorithm defined in (11) (12) forms the content of \textit{Du Bois-Reymond's infinitary calculus}.

137) Math. Ann. 35 (1890), p. 302.

138) In the following, the symbol $(M_\nu)$ shall once and for all represent a sequence of numbers of this kind.

}