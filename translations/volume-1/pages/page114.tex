\thispagestyle{fancy}

\vspace{0.5cm}

where $p^{(\chi)}$, $p$, $p_\chi$ denote entirely arbitrary (e.g., also \textit{increasing}) positive numbers and $\log_\chi M_\nu$ represents the $\chi$-fold iterated logarithm\textsuperscript{139)}. One can thus, starting from an arbitrarily chosen \textit{"infinite"} $\lim M_\nu$, establish a scale unbounded on both sides of ever \textit{weaker} or ever \textit{stronger "infinites"}, so-called \textit{order types} of the infinite. This scale can be thickened in infinitely many \textit{ways}\textsuperscript{140)}. One is also not restricted to logarithms and exponential functions in its formation; but they are the \textit{analytically simplest} functions of this kind. One can, however, also construct sequences of numbers or functions that become infinite more weakly (strongly) not only than any specific \textit{individual}, but than \textit{all possible} iterated logarithms\textsuperscript{141)} (exponential functions). The analogous also holds for \textit{any arbitrary} scale of such order types\textsuperscript{142)}.

In connection with No. 14, it may be noted that these\textit{"different types"} of \textit{infinite} are by no means \textit{proper} infinites in the sense specified there. The so-called infinitary relations of the form (12) are merely compilations of an unbounded number of relations between \textit{finite} numbers that are not bound to any upper limit\textsuperscript{143)}.

The analogous considerations can be made regarding becoming zero or infinitely small. Only naturally in the case $\lim a_\nu = 0$, $\lim b_\nu = 0$, the relation $a_\nu \prec b_\nu$ has the meaning: $a_\nu$ becomes infinitely small of \textit{higher} order (stronger, faster) than $b_\nu$, and so on\textsuperscript{144)}.

\vspace{0.2cm}
\textbf{20. Limit Values of Doubly Infinite Sequences of Numbers.} The \textit{limit values of doubly infinite} sequences of numbers have, to my knowledge, not yet been explicitly treated in the literature; one has only investigated \textit{special} forms of such limit values (double series) and limit values of \textit{functions}

\vfil
\leftline{\rule{2in}{0.4pt}}
\vspace{0.1cm}
{
\footnotesize
139) One was led to the consideration of such iterated logarithms (and, as a natural complement, to that of iterated exponential quantities) through investigations on series convergence; cf. No. 26 of this article. \textit{Abel} was, as far as I could determine, the first who made use of the iterated logarithms in this sense: Oeuvres compl. Ed. Sylow-Lie 1, p. 400; 2, p. 200. Scales of similar form as (14) are found first in \textit{A. de Morgan}, Diff. and integr. calculus (London 1839) p. 323.

140) \textit{Du Bois-Reymond} l.c. p. 341.

141) \textit{Du Bois-Reymond}, J. f. Math. 76 (1873), p. 88.

142) \textit{Du Bois-Reymond}, Math. Ann. 8, p. 365, footnote. \textit{Pincherle}, Mem. Acad. Bologn. (4), 5 (1884), p. 739. \textit{J. Hadamard}, Acta math. 18 (1894), p. 331.

143) Cf. my remarks in the Münch. Sitzber. 27 (1897), p. 307.

144) More on \textit{"types of infinity"} see I A 5, No. 17.

}