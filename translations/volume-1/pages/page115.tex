\thispagestyle{fancy}
\fancyhead[LO]{21. Convergence and Divergence of Infinite Series}

\vspace{0.5cm}

\textit{of two variables}, of which \textit{at least one} (in the infinite series $\sum f_\nu(x)$) appears as \textit{continuously} variable. Since the characteristic of the possibilities coming into question here emerges most simply in sequences of numbers of the form $a_{\mu\nu}$ ($\mu = 0, 1, 2, \ldots$; $\nu = 0, 1, 2, \ldots$)\textsuperscript{145)}, I have recently briefly compiled the most important theorems about such limit values\textsuperscript{146)}. As a \textit{criterion} for the existence of a finite or positively infinite $\lim_{\mu,\nu=\infty} a_{\mu\nu}$ appears thereby a condition of the form: $|a_{\mu+\varrho, \nu+\sigma} - a_{\mu\nu}| \le \varepsilon$ or $> G$ for $\mu \geq m$, $\nu \geq n$. Hereby the existence of $\lim_{\nu=\infty} a_{\mu\nu}$ for any specific $\mu$ and $\lim_{\mu=\infty} a_{\mu\nu}$ for any specific $\nu$ is in no way prejudiced. On the other hand, there obviously exist under all circumstances: $\underline{\lim_{\nu=\infty}} a_{\mu\nu}$, $\overline{\lim_{\nu=\infty}} a_{\mu\nu}$ ($\mu = 0, 1, 2, \ldots$), $\underline{\lim_{\mu=\infty}} a_{\mu\nu}$,  $\overline{\lim_{\mu=\infty}} a_{\mu\nu}$ ($\nu = 0, 1, 2, \ldots$), and the main theorem holds:

\vspace{-0.5cm}
\begin{align}
    (15) \quad \lim_{\mu,\nu=\infty} a_{\mu\nu} = \lim_{\mu=\infty} (\underline{\overline{\lim_{\nu=\infty}}} a_{\mu\nu}) = \lim_{\nu=\infty} (\underline{\overline{\lim_{\mu=\infty}}} a_{\mu\nu}),
\end{align}

if the \textit{first} of these limit values (in the broader sense) \textit{exists}.

\vspace{0.4cm}
\begin{center}
    \textbf{Part Two. Infinite Series, Products, Continued Fractions and Determinants.}
    
\vspace{0.2cm}
\textbf{III. Infinite Series.}
\end{center}

\textbf{21. Convergence and Divergence.} The simplest type of lawfully defined sequences of numbers is formed by the \textit{infinite series} $(s_\nu)$, in which each term $s_\nu$ is generated from the preceding one by a simple \textit{addition}, so that:

\vspace{-0.5cm}
$$s_\nu = s_{\nu-1} + a_\nu = a_0 + a_1 + \cdots + a_\nu .$$

One then says the \textit{infinite series} $\sum_{\nu=0}^{\infty} a_\nu$ is \textit{convergent}, properly or improperly \textit{divergent}, according to whether the sequence of numbers $(s_\nu)$ \textit{converges} or properly or improperly \textit{diverges}. If $\lim s_\nu = s$ (where $s$ is a definite number incl. 0), then $s$ is called the \textit{sum} of the series\textsuperscript{148)}.

\vfill
\leftline{\rule{2in}{0.4pt}}
\vspace{0.2cm}
{
\footnotesize
145) This applies, e.g., also regarding the fundamental concept of uniform convergence. Cf. II A 1.

146) Münch. Ber. 27 (1897), p. 103 ff.

147) L.c. p. 105.

148) Some authors initially designate $s$ only as the \textit{limit value} of the series and use the expression \textit{sum} only when $\lim s_\nu$ is \textit{commutative}, thus the series converges \textit{absolutely} (cf. No. 31). On the meaning of the symbol $\sum_{\nu=-\infty}^{\infty} a_\nu$ cf. No. 59, footnote 448.

}