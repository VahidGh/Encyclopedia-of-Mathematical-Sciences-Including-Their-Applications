\thispagestyle{fancy}

\vspace{0.5cm}

One also frequently uses the expression that the \textit{sum} of the series is \textit{infinitely large} or \textit{indeterminate} (it \textit{oscillates}) when $(s_\nu)$ \textit{properly} or \textit{improperly} diverges. As a \textit{necessary and sufficient} condition for the \textit{convergence} of the series, it follows from No. 13:

The quantity $|s_{n+\varrho} - s_n| \equiv |a_{n+1} + a_{n+2} + \cdots + a_{n+\varrho}|$ must become \textit{arbitrarily small} solely through the choice of $n$ for \textit{every} $\varrho$.

Although the introduction of infinite series dates back to the 17th century\textsuperscript{149)} and their treatment occupies an exceedingly broad space in the mathematical literature of the 18th, one will search in vain for such a \textit{criterion} of convergence\textsuperscript{150)}. If one \textit{asked} at all about the \textit{convergence} of a series development obtained through any formal operations (which in itself was already an exception), one considered the determination that $\lim a_\nu = 0$ already sufficient, although \textit{Jac. Bernoulli} had already demonstrated the \textit{divergence} of the harmonic series $\sum \frac{1}{\nu}$ \textsuperscript{151)}. Even \textit{J. L. Lagrange} in his treatise on the solution of literal equations by series still stands completely on this standpoint\textsuperscript{152)}.

\vfill
\leftline{\rule{2in}{0.4pt}}
\vspace{0.2cm}
{
\footnotesize
149) On the earlier developmental history of the theory of infinite series cf. \textit{Reiff} l.c.

150) \textit{Reiff} (p. 119) seems to me to err when he interprets a passage in \textit{Euler} (Comm. Petrop. 7, 1734, p. 150) to mean that the latter actually already knew the convergence condition in the (\textit{Cauchy}ian) form: $\lim_{n=\infty} (s_{n+\varrho} - s_n) = 0$. The relevant passage in \textit{Euler} only states that a series \textit{diverges} when: $\lim_{n=\infty} |s_{kn} - s_n| > 0$.

151) Pos. arithm. de seriebus 1689. Prop. XVI (Opera omnia 1, p. 392). \textit{B}. gives there \textit{two} proofs, and designates his brother \textit{Johann} as the originator of the \textit{first} (based on the so-called \textit{Bernoullian paradox} $\sum_{\nu=2}^{\infty} \frac{1}{\nu} = \sum_{\nu=1}^{\infty} \frac{1}{\nu}$). The \textit{second} (with the help of the inequality 

$$\frac{1}{a} + \frac{1}{a+1} + \cdots + \frac{1}{a^2} < \frac{1}{a} + \frac{a^2 - a}{a^2} = 1)$$ 

is in principle still the usual one today.

152) Berl. Mem. 24 (1770). Oeuvres 3, p. 61. "... pour qu'une série puisse être regardée comme représentant réellement la valeur d'une quantité cherchée, il faut qu'elle soit convergente à son extrémité, c'est à dire que ses derniers termes soient infiniment petits, de sorte que l'erreur puisse devenir moindre qu'aucune quantité donnée". The convergence investigation that follows is limited to showing that the individual series terms eventually converge to zero. After this, it can hardly seem surprising that, for example, in the 1st volume of \textit{Klügel}'s W. B. printed in 1803, p. 555, one still finds the following definition: "A series is convergent if its terms in their sequence become ever smaller. The sum of the terms then approaches ever more closely the value of the quantity which is the sum of the

}