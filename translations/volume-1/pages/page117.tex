\thispagestyle{fancy}
\fancyhead[LO]{22. The Convergence Criteria of Gauss and Cauchy}

\vspace{0.5cm}

And the introduction of the \textit{remainder term} of the \textit{Taylor} series occurs in \textit{Lagrange} by no means with the intention of proving its \textit{convergence} (this is not touched upon with a single word as something self-evident), but merely to be able to \textit{estimate} the error bound when truncating the series at a finite point\textsuperscript{153)}.

The first essentially rigorous formulation of the necessary and sufficient condition for the convergence of a series is usually attributed to \textit{Cauchy}\textsuperscript{154)}. \textit{Herm. Hankel}\textsuperscript{155)} and \textit{O. Stolz}\textsuperscript{156)} have, however, pointed out that the same can be found already some years before \textit{Cauchy} in \textit{Bolzano}\textsuperscript{157)}. The latter's version, which (apart from the notation) agrees exactly with the one given above, appears even more precise than that given by \textit{Cauchy}, which does not exclude the possibility of a misunderstanding\textsuperscript{158)}. Since \textit{Bolzano}'s writings received little attention until very recent times, it must nevertheless be said that \textit{Cauchy} is to be regarded as the actual founder of an exact general theory of series\textsuperscript{159)}.

\vspace{0.5cm}
\textbf{22. The Convergence Criteria of Gauss and Cauchy.} The \textit{true criterion} for the convergence and divergence of a series indicated above is usable for determining convergence or divergence only in a few cases (e.g., for the geometric progression, for series of the form $\sum(a_\nu - a_{\nu+1})$, for the harmonic series). This circumstance led to the establishment of more convenient

\vfill
\leftline{\rule{2in}{0.4pt}}
\vspace{0.2cm}
{
\footnotesize

series continued to infinity."

153) Théorie des fonctions (1797). Oeuvres 9, p. 85.

154) Anal. algébr. (also 1821), p. 125.

155) \textit{Ersch u. Gruber}, Art. Grenze, p. 209.

156) Math. Ann. 18 (1881), p. 259.

157) Beweis des Lehrsatzes etc. 1817.

158) This applies to an even greater extent to a later formulation appearing in the Anc. exerc. 2 (1827), p. 221: $\lim_{n=\infty} (s_{n+\varrho} - s_n) = 0$, which has indeed been misunderstood and consequently contested. Cf. my note in the Münch. Sitzber. 27 (1897), p. 327. \textit{N. H. Abel}, who expresses himself almost verbatim the same way in his treatise on the binomial series (J. f. Math. 1, 1826, p. 313), gives in a note dating from 1827 but only found in his estate (Oeuvres 2, p. 197) a formulation agreeing with ours, free from objection.

159) \textit{K. F. Gauss} in his investigation of the hypergeometric series (1812), which admittedly provides the \textit{first example} of exact convergence investigation, does not go into \textit{general} convergence questions.

}