\thispagestyle{fancy}

\vspace{0.5cm}

manageable \textit{convergence} and \textit{divergence} criteria, i.e., conditions which, while \textit{not necessary} for convergence or divergence, prove to be \textit{sufficient}. The first criteria of this kind come from \textit{Gauss}\textsuperscript{160)} and relate to series with all positive terms $a_\nu$, for which:

\vspace{-0.2cm}
$$\frac{a_{\nu+1}}{a_\nu} = \frac{\nu^m + A\nu^{m-1} + B\nu^{m-2} + \cdots}{\nu^m + a\nu^{m-1} + b\nu^{m-2} + \cdots} .$$

The series \textit{diverges} when $A - a \geq -1$, it \textit{converges} when $A - a < -1$\textsuperscript{161)}. The \textit{divergence} in the case $A - a > 0$ or $A - a = 0$ is directly concluded from the fact that the terms of the series grow to infinity or approach a finite limit different from zero. On the other hand, in the case $A - a < 0$, the \textit{divergence} or \textit{convergence} is obtained with the help of the principle of \textit{series comparison} (i.e., the term-by-term comparison of the series to be investigated with another \textit{series already recognized as divergent or convergent}, e.g., through direct summation), which is to be regarded as \textit{fundamental} for all further convergence investigations.

\vspace{0.3cm}
\textbf{23. Continuation.} After \textit{Cauchy} had established that the \textit{convergence} of a series with \textit{positive} and \textit{negative} terms is assured if the series of \textit{absolute values} converges\textsuperscript{162)}, it was primarily a matter of developing convergence criteria for series with all \textit{positive terms}. By comparison with the geometric progression, he first obtained the two \textit{fundamental criteria of the first and second kind}\textsuperscript{163)}, namely:

\vspace{0.1cm}
(I) $\sum a_\nu$ \textit{ diverges}, when  $\overline{\lim} \sqrt[\nu]{a_\nu} > 1$ ; \textit{ converges}, when  $\overline{\lim} \sqrt[\nu]{a_\nu} < 1$ ,

(II) $\sum a_\nu$ \textit{ diverges}, when  $\lim \frac{a_{\nu + 1}}{a_\nu} > 1$ ; \textit{ converges}, when  $\lim \frac{a_{\nu + 1}}{a_\nu} < 1$ .
\vspace{0.1cm}

What is to be emphasized is the sharp distinction in the formulation of these two criteria; for (I), the nature of the \textit{upper} limit of $\sqrt[\nu]{a_\nu}$ is already sufficient to decide, with the exclusion of the \textit{single} case $\overline{\lim} \sqrt[\nu]{a_\nu} = 1$

\vspace{-0.1cm}
\leftline{\rule{2in}{0.4pt}}
\vspace{0.1cm}
{
\footnotesize
160) See the just cited treatise: Opera 3, p. 139.

161) An extension of these criteria to the case of \textit{complex} $a_\nu$ has been given by \textit{Weierstrass}: J. f. Math. 51 (1856), p. 22 ff.

162) Anal. algébr. p. 142. The formulation of the proof is admittedly inadequate. More rigorous: Résumé analyt. p. 39.

163) L.c. p. 133, 134. We designate a criterion, following \textit{Du Bois-Reymond} (J. f. Math. 76, p. 61), as one of the \textit{first} or \textit{second} kind, depending on whether it depends exclusively on $a_\nu$ or on $\frac{a_{\nu+1}}{a_\nu}$.

}