\thispagestyle{fancy}
\fancyhead[LO]{23. Cauchy's Convergence Criteria}

\vspace{0.5cm}

the divergence or convergence; for (II), only the case is explicitly considered where \textit{a definite} $\lim \frac{a_{\nu+1}}{a_\nu}$ \textit{exists}, i.e., besides the case $\lim \frac{a_{\nu+1}}{a_\nu} = 1$, all those remain unresolved where \textit{no definite limit} exists\textsuperscript{164)}. This superiority of criterion (I) over (II) has been specially emphasized by \textit{Cauchy}\textsuperscript{165)}, and he has further shown how it can serve to \textit{precisely fix} \textit{the convergence interval}\textsuperscript{166)} (the \textit{radius of convergence}\textsuperscript{167)}) of a power series $\sum a_\nu x^\nu$ in \textit{every} case\textsuperscript{168)}. For the possible resolution of the case which the application of criterion (I) leaves undecided, \textit{Cauchy} proves an auxiliary theorem about the \textit{simultaneous} divergence and convergence of the series $\sum a_\nu$ and $\sum 2^\nu \cdot a_{2^\nu - 1}$ (if $a_{\nu+1} \le a_\nu$), infers from it the divergence of the series $\sum \frac{1}{\nu^{1+\varrho}}$ for $\varrho \leq 0$, the convergence for $\varrho > 0$, and derives from it a sharpened \textit{criterion of the first kind}:

\vfill
\leftline{\rule{2in}{0.4pt}}
\vspace{0.2cm}
{
\footnotesize
164) Somewhat more completely, one can formulate (II) as follows: $\sum a_\nu$ \textit{diverges} when $\underline{\lim} \frac{a_{\nu+1}}{a_\nu} > 1$, \textit{converges} when $\overline{\lim} \frac{a_{\nu+1}}{a_\nu} < 1$. The question remains undecided when simultaneously: 

$$\underline{\lim} \frac{a_{\nu+1}}{a_\nu} \leq 1 , \quad \quad \overline{\lim} \frac{a_{\nu+1}}{a_\nu} \geq 1 .$$

165) L.c. p. 135 "le \textit{premier} de ces théorèmes etc."

166) L.c. p. 151. Résumés analyt. (1833), p. 46.

167) L.c. p. 286. Rés. analyt. p. 113. Exerc. d'Anal. 3 (1844), p. 390.

168) It is peculiar that this result, extremely important for function theory (to which \textit{Cauchy} himself evidently attached great value), seems to have been completely overlooked or fallen into oblivion in many cases. Only a few years ago it was rediscovered by \textit{J. Hadamard} (J. de Math. (4) 8 [1892], p. 107) and has since then often been cited as the \textit{"Hadamard theorem"}. -- On the other hand, despite the flawlessly correct formulation of criterion (I) and the explicit emphasis on its more \textit{special} character, the opinion has partly formed that the three assumptions $\lim \frac{a_{\nu+1}}{a_\nu} < 1$, $= 1$, $> 1$ exhaust \textit{all} possibilities coming into consideration, or that at least the \textit{convergence} of $\sum a_\nu$ in the case of the \textit{non}-existence of a definite $\lim \frac{a_{\nu+1}}{a_\nu}$ appears as a \textit{special curiosity} (cf. my remarks Math. Ann. 35 [1890], p. 308). And accordingly, in many textbooks (even belonging to the most recent times), the \textit{whole theory of power series} is founded on the much too special assumption that $\lim |\frac{a_{\nu+1}}{a_\nu}|$ \textit{exists}.

}