\thispagestyle{fancy}

\vspace{0.5cm}
 
\vspace{-0.3cm}
\begin{align}
    \raisebox{0ex}{
        \begin{tabular}[b]{l} 
            $\lim \frac{\lg \frac{1}{a_\nu}}{\lg \nu}$ 
        \end{tabular}
        }
    \left\{ 
    \begin{tabular}{l}
    $< 0: \textit{ Divergence},$ \\
    $> 0: \textit{ Convergence}.$
    \end{tabular}
    \right.
\end{align}

Elsewhere\textsuperscript{169)}, \textit{Cauchy} shows that the divergence or convergence of the series $\sum_{m=\nu}^{\infty} f(\nu)$ under certain conditions coincides with that of the integral $\int_m^{\infty} f(x)dx$, and from this obtains the \textit{criterion pair}:

\vspace{-0.3cm}
\begin{align}
    \left\{ 
    \begin{tabular}{l}
    $\lim \nu \cdot a_\nu > 0: \textit{ Divergence},$ \\
    $\lim \nu^{1+\varrho} \cdot a_\nu = 0: \textit{ Convergence}, \quad (\varrho > 0),$
    \end{tabular}
    \right.
\end{align}

which, incidentally, is easily recognized as essentially \textit{equivalent} to the \textit{disjunctive double criterion} (16) and could have been derived more simply directly from the behavior of the series $\sum \frac{1}{\nu^{1+\varrho}}$ ($\varrho \geq 0$). More important, it seems to me, is that \textit{Cauchy} here for the first time proves the \textit{divergence} of $\sum \frac{1}{\nu \lg \nu}$, the \textit{convergence} of $\sum \frac{1}{\nu (\lg \nu)^{1+\varrho}}$ for $\varrho > 0$, whereby the path for the further sharpening of criteria (16) and (17) appears directly indicated.

\vspace{0.3cm}
\textbf{24. Kummer's General Criteria.} The criteria of \textit{J. L. Raabe}, \textit{J. M. C. Duhamel}, \textit{de Morgan}, \textit{Bertrand}, \textit{P. O. Bonnet}, \textit{M. G. v. Paucker} (whose publication falls in the period from 1832-1851 and which will be discussed later) provide merely such \textit{sharpenings} of \textit{Cauchy}'s criteria, to which they also essentially adhere in form and method of derivation.

While all the criteria mentioned so far have a \textit{special} character, insofar as they are consistently based on the comparison of $a_\nu$ with one of the \textit{special} sequences $a^\nu$, $\nu^p$, $\nu \cdot (\lg \nu)^p$ etc., \textit{E. E. Kummer}\textsuperscript{170)} has derived the following \textit{convergence} criterion of surprisingly \textit{general} character: $\sum a_\nu$ \textit{converges} if there exists any positive sequence of numbers $(p_\nu)$\textsuperscript{171)} such that:

\begin{align}
\lim \lambda_\nu \equiv \lim (P_\nu \cdot \frac{a_\nu}{a_{\nu+1}} - P_{\nu + 1} > 0.
\end{align}

\vfill
\leftline{\rule{2in}{0.4pt}}
\vspace{0.2cm}
{
\footnotesize
169) Anc. Exerc. 2 (1827), p. 221 ff. The theorem on the connection of the integral with the series is already found in geometric form in \textit{Colin Mac Laurin} (Treatise of fluxions 1742, p. 289). On the transformation of this criterion by \textit{B. Riemann}, cf. No. 36.

170) J. f. Math. 13 (1835), p. 171 ff.

171) \textit{Kummer} adds the additional condition: $\lim p_\nu \cdot a_\nu = 0$, which is, however, in truth superfluous, as \textit{Dini} first showed in a work to be mentioned immediately.

}