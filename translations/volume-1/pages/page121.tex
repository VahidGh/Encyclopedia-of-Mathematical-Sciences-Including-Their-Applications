\thispagestyle{fancy}
\fancyhead[LO]{25. The Convergence Theories of Dini, du Bois-Reymond and Pringsheim}

\vspace{0.5cm}

At the same time, \textit{K}. shows that $\sum a_\nu$ \textit{diverges} when:

\vspace{-0.5cm}
\begin{align}
    \lim P_\nu \cdot a_\nu = 0,\textsuperscript{172)} \quad \lim \lambda_\nu = 0, \quad \lim \frac{P_\nu \cdot a_\nu}{\lambda_\nu} > 0,
\end{align}
\vspace{-0.5cm}

and demonstrates that there always actually \textit{exist} (infinitely many) sequences of numbers $(P_\nu)$ which satisfy one of the criteria (18) (19); but to be able to determine them in each \textit{case}, one would have to be oriented in advance about the \textit{convergence} and \textit{divergence} of $\sum a_\nu$.

\vspace{0.3cm}
\textbf{25. The Theories of Dini, du Bois-Reymond and Pringsheim.} Significant generalizations of the whole theory of convergence criteria are then brought by \textit{Dini}'s extensive treatise, initially directly connecting to \textit{Kummer}'s investigation: \textit{"Sulle serie a termini positivi}"\textsuperscript{173)}, which, however, does not seem to have found the deserved dissemination.

\textit{Du Bois-Reymond}'s "New Theory of Convergence and Divergence of Series with Positive Terms"\textsuperscript{174)} seems to have originated quite independently of \textit{Dini}'s work. Although his investigation methods and main results are not essentially different from those of \textit{Dini}, he goes beyond \textit{Dini} in \textit{principle} through the expressed tendency "to give the theory of convergence and divergence, through more rigorous foundation and through appropriate connection of its theorems, the character of a mathematical theory which it has lacked until now." Since, however, \textit{Du Bois-Reymond} does not seem to me to have achieved this goal by any means\textsuperscript{175)}, I have taken up the problem posed by him anew and resolved it as\textsuperscript{176)}: Rules of the greatest possible generality are derived from the completely uniformly implemented, most obvious principle of series comparison, which not only include all previously known criteria as special cases\textsuperscript{177)}, but also

\vfill
\leftline{\rule{2in}{0.4pt}}
\vspace{0.2cm}
{
\footnotesize
172) Here this condition is \textit{essential}.

173) Pisa 1867 (Tipogr. Nistri). Also: Ann. dell'Univ. Tosc. 9 (1867), p. 41-76.

174) J. f. Math. 76 (1873), p. 61-91.

175) Cf. my critical remarks Math. Ann. 35 (1890), p. 298.

176) Math. Ann. 35 (1890), p. 297-394. Addendum to this: Math. Ann. 39 (1891), p. 125. An extract of this theory can be found: Math. Pap. Congr. Chicago [1896] 1893, p. 305-329.

177) An exception is \textit{Kummer}'s \textit{divergence} criterion, because it depends not, like all other criteria, on \textit{one}, but on \textit{three} conditions. The same, however, is made completely dispensable by the more general divergence criteria of the 2nd kind. Cf. my treatise l.c. p. 365, footnote.

}