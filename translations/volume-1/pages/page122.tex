\thispagestyle{fancy}

\vspace{0.5cm}

make their scope and their more or less hidden connection more clearly recognizable. In particular, \textit{Kummer}'s convergence criterion of the \textit{second kind}, which has so far stood completely apart in its generality, appears as a natural member of this theory and finds its complete analogue among the criteria of the \textit{first kind}.

\vspace{0.3cm}
\textbf{26. The Criteria of First and Second Kind.} I denote by $d_\nu \equiv D_\nu^{-1}$ or $c_\nu \equiv C_\nu^{-1}$ the general term of a series recognized as \textit{divergent} or \textit{convergent}, and by $a_\nu$ that of a series to be judged. Then the \textit{main form} of the criteria of the \textit{first and second kind} emerges:

\vspace{-0.5cm}
\begin{align}
    \left\{ 
    \begin{tabular}{l}
    $\lim D_\nu \cdot a_\nu > 0: \textit{ Divergence},$\\ 
    $\lim C_\nu \cdot a_\nu < \infty: \textit{ Convergence}\textsuperscript{178)}.$
    \end{tabular}
    \right.
\end{align}
\vspace{-0.5cm}

\vspace{-0.5cm}
\begin{align}
    \left\{ 
    \begin{tabular}{l}
    $\lim (D_\nu \cdot \frac{a_\nu}{a_{\nu+1}} - D_{\nu+1}) < 0: \textit{ Divergence},$\\ 
    $\lim  (C_\nu \cdot \frac{a_\nu}{a_{\nu+1}} - C_{\nu+1}) > 0: \textit{ Convergence}.$
    \end{tabular}
    \right.
\end{align}

One can give these criteria manifold \textit{other forms} if one compares not $a_\nu$ directly with $d_\nu$, $c_\nu$, but $F(a_\nu)$ with $F(d_\nu)$, $F(c_\nu)$, where $F$ is understood to be a \textit{monotonic} function. On this is based in particular the transformation of the \textit{criterion pairs} (20) into \textit{disjunctive double criteria}, in which a \textit{single} expression decides on divergence and convergence.

\textit{If} for any specific choice of $D_\nu$, $C_\nu$ one of those criteria fails in such a way that the equality sign appears in place of the signs $ < or > $, then there arises the \textit{possibility} of obtaining \textit{more effective} criteria if one introduces instead of $D_\nu$, $C_\nu$ such $\overline{D_\nu}$, $\overline{C_\nu}$ which satisfy the condition: $\overline{D_\nu} \prec D_\nu$ or $\overline{C_\nu} \succ C_\nu$, in which case the series $\sum \overline{D_\nu}^{-1}$ or $\sum \overline{C_\nu}^{-1}$ is said to be \textit{more weakly} divergent or convergent than $\sum D_\nu{}^{-1}$ or $\sum C_\nu{}^{-1}$\textsuperscript{179)}. But such $D_\nu$, $C_\nu$ can be produced not only in unlimited number, but \textit{all possible ones} with the help of the following theorems:

If $0 < M_\nu < M_{\nu+1}$, $\lim M_\nu = \infty$, then each of the three expressions

\vfill
\leftline{\rule{2in}{0.4pt}}
\vspace{0.2cm}
{
\footnotesize
178) The notation: $< \infty$ means: \textit{not} $\infty$, thus \textit{under a finite bound}. Furthermore, note that $\lim$ here stands in the sense of $\overline{\underline{\lim}}$, i.e., there need not exist \textit{a definite limit} of the nature in question.

179) The concept of \textit{"weaker"} divergence and convergence can be formulated more generally. Cf. l.c. p. 319, 327.

}