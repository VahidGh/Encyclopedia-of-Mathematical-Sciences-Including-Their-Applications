\thispagestyle{fancy}
\fancyhead[LO]{26. The Criteria of First and Second Kind}

\vspace{0.5cm}

\vspace{-0.5cm}
\begin{align}
    (a) \quad M_{\nu+1} - M_\nu, \quad (b) \quad \frac{M_{\nu+1} - M_\nu}{M_\nu}, \quad (c) \quad \frac{M_{\nu+1} - M_\nu}{M_{\nu+1}}
\end{align}
\vspace{-0.3cm}

represents a $d_\nu$, and conversely, every $d_\nu$ can be represented in the form (a), (b), and in the case $d_\nu < 1$ also in the form (c)\textsuperscript{180)}.

Furthermore, the expression:

\vspace{-0.5cm}
\begin{align}
    \frac{M_{\nu+1} - M_\nu}{M_{\nu+1} \cdot M_\nu}
\end{align}
\vspace{-0.5cm}

represents a $c_\nu$ -- \textit{vice versa}.

The series in question diverge or converge \textit{more weakly} the \textit{more slowly} $M_\nu$ increases with $\nu$\textsuperscript{181)}.

By introducing $M_\nu^\varrho$ ($0 < \varrho < 1$) instead of $M_\nu$, one recognizes with the help of the relation:

\vspace{-0.5cm}
\begin{align}
    \frac{M_{\nu+1}^\varrho - M_\nu^\varrho}{M_{\nu+1}^\varrho \cdot M_\nu^\varrho} \sim \frac{M_{\nu+1} - M_\nu}{M_{\nu+1} \cdot M_\nu^\varrho} \preceq \frac{M_{\nu+1} - M_\nu}{M_{\nu+1}^{1+\varrho}}
\end{align}
\vspace{-0.3cm}

each of these terms as the general term of a \textit{convergent} series\textsuperscript{182)}.

Then the substitution of $\lg_\chi M_\nu$ ($\chi = 1, 2, 3, \ldots$) and (24), if one sets:

\vspace{-0.5cm}
\begin{align}
    x \cdot \lg_1 x \cdot \lg_2 x \cdots \lg_\chi x = L_\chi(x),
\end{align}
\vspace{-0.5cm}

with the help of elementary infinitary relations, provides the two indefinitely continuable sequences:

\vspace{-0.5cm}
\begin{align}
    (a) \, \frac{M_{\nu+1} - M_\nu}{L_\chi(M_\nu)}, \quad (b) \, \frac{M_{\nu+1} - M_\nu}{L_\chi(M_{\nu+1}) \cdot (\lg_\chi M_{\nu+1})^\varrho} \, (\varrho > 0, \, \chi=1,2,3,\cdots)
\end{align}
\vspace{-0.3cm}

as general terms of constantly \textit{more weakly} diverging or converging series. These expressions contain for $\chi = 0$ the

\vfill
\leftline{\rule{2in}{0.4pt}}
\vspace{0.2cm}
{
\footnotesize
180) The comparison of expressions (22) (b) and (c) with (a) shows directly that for \textit{every divergent} series there exist \textit{more weakly} diverging ones. If one sets $M_{\nu+1} - M_\nu = d_\nu, M_0 = 0$, thus: $M_{\nu+1} = d_0 + d_1 + \ldots + d_\nu = s_\nu$, it follows: With the series $\sum d_\nu$ also \textit{diverges} $\sum \frac{d_\nu}{s_{\nu+1}}$ (theorem of \textit{Abel}: J. f. Math. 3 [1828], p. 81) and $\sum \frac{d_\nu}{s_\nu}$ (\textit{Dini} l.c. p. 8).

181) One can directly designate $M_\nu$ as the \textit{measure} of divergence or convergence of $\sum (M_{\nu+1} - M_\nu)$ or $\sum \frac{M_{\nu+1} - M_\nu}{M_{\nu+1} \cdot M_\nu}$. Cf. \textit{Du Bois-Reymond} l.c. p. 64.

182) From this follows, with application of the notation used immediately before, that $\sum \frac{d_\nu}{s_\nu^{1+\varrho}}$ \textit{converges}. This theorem is also found already in \textit{Abel} (in the posthumous note mentioned above: 2, p. 198), additionally in \textit{Dini} (l.c. p. 8).

}