\thispagestyle{fancy}

\vspace{0.5cm}

(22), (24), if one further sets $L_0(x) = \lg_0(x) = x$.

Also, one can in the denominator of expression (26b) replace $M_{\nu+1}$ without further ado by $M_\nu$ if one introduces the restriction $M_{\nu+1} \sim M_\nu$, which proves expedient for the formation of criteria.

\vspace{0.3cm}
\textbf{27. Continuation.} According to this, the \textit{main form of all possible criteria of the first kind} is contained in the two relations:

\vspace{-0.5cm}
\begin{align}
    \left\{ 
    \begin{tabular}{l}
    $\lim \frac{M_\nu}{M_{\nu+1} - M_\nu} \cdot a_\nu > 0: \textit{ Divergence},$\\ 
    $\lim \frac{M_{\nu+1} \cdot M_\nu}{M_{\nu+1} - M_\nu} \cdot a_\nu < \infty: \textit{ Convergence},$
    \end{tabular}
    \right.
\end{align}
\vspace{-0.2cm}

and the relations:

\vspace{-0.5cm}
\begin{align}
    \left\{ 
    \begin{tabular}{l}
    $\lim \frac{L_\chi(M_\nu)}{M_{\nu+1} - M_\nu} \cdot a_\nu > 0: \textit{ Divergence},$\\ 
    $\lim \frac{L_\chi(M_{\nu}) \cdot \lg_{\chi}^{\varrho} M_\nu}{M_{\nu+1} - M_\nu} \cdot a_\nu < \infty: \textit{ Convergence},$
    \end{tabular}
    \right. \raisebox{-0.5ex}{\bigg(\begin{tabular}[b]{l} $M_{\nu+1} \sim M_\nu$ \\ $\varrho > 0$ \end{tabular}\bigg)}
\end{align}
\vspace{-0.2cm}

for $\chi = 0, 1, 2, \ldots$ represent a \textit{scale} of \textit{increasingly} effective criteria.

The special choice $M_\nu = \nu$ then provides for $\chi = 0$ the \textit{Cauchy} criterion (17), for $\chi = 1, 2, \ldots$ that series which was first established by \textit{de Morgan}\textsuperscript{183)}, later by \textit{Bonnet}\textsuperscript{184)}.

The criteria (28) can also be replaced by the following scale of disjunctive criteria\textsuperscript{185)}:

\vspace{-0.5cm}
\begin{align}
\left\{
\begin{array}{l}
(a) \quad \lim \frac{\lg \frac{M_{\nu+1} - M_\nu}{a_\nu}}{M_\nu} \begin{cases} < 0 & \textit{Divergence}, \\ > 0 & \textit{Convergence}, \end{cases} \\[1em]
(b) \quad \lim \frac{\lg \frac{M_{\nu+1} - M_\nu}{L_\chi(M_\nu) \cdot a_\nu}}{L_{\chi+1}(M_\nu)} \begin{cases} < 0 & \textit{Divergence}, \\ > 0 & \textit{Convergence}. \end{cases} \raisebox{-0.5ex}{$\quad (\chi = 0, 1, 2, \ldots).$}
\end{array}
\right.
\end{align}

If one again specializes $M_\nu = \nu$, then (a) provides the \textit{Cauchy} fundamental criterion (I), (b) for $\chi = 0$ the \textit{Cauchy} criterion (16), for $\chi = 1, 2, \ldots$ a series first derived by \textit{Bertrand}\textsuperscript{186)}.

\vfill
\leftline{\rule{2in}{0.4pt}}
\vspace{0.2cm}
{
\footnotesize
183) Diff. and Integr. Calc. (1839), p. 326. \textit{De Morgan} derives from this yet another seemingly more general criterion form, whose scope is, however, exactly the same, as \textit{Bertrand} and \textit{Bonnet} (J. de Math. 7, p. 48; 8, p. 86) have shown.

184) J. de Math. 8 (1843), p. 78.

185) Derived in somewhat different form by \textit{Dini} l.c. p. 14.

186) J. de Math. 7 (1842), p. 37. -- A more elementary derivation is given by \textit{Paucker} (J. f. Math. 42 [1851], p. 139) and \textit{Cauchy} (C. R. 1856, 2me sem., p. 638),

}