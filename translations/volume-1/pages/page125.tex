\thispagestyle{fancy}
\fancyhead[LO]{27. The Criteria of First and Second Kind}

\vspace{0.5cm}

Finally, the \textit{convergence} criterion contained in (a) permits the following generalization:

\vspace{-0.7cm}
\begin{align}
    \lim \frac{\lg P_\nu \cdot a_\nu}{S_\nu} < 0: \textit{ Convergence},\textsuperscript{187)}
\end{align}
\vspace{-0.5cm}

where $(P_\nu)$ can mean \textit{any arbitrary positive} sequence of numbers and 

\vspace{-0.5cm}
$$S_\nu = P_0 + P_1 + \cdots + P_\nu.$$
\vspace{-0.5cm}

This \textit{most general convergence criterion of the first kind} then forms the analogue to \textit{Kummer's convergence criterion of the second kind}.

By substituting the general expression (23) for $C_{\nu}^{-1}$ in the \textit{convergence} criterion of the \textit{second} kind (21), the remarkable result emerges that the same can also be brought to the form:

\vspace{-0.5cm}
\begin{align}
    \lim (D_\nu \cdot \frac{a_\nu}{a_{\nu+1}} - D_{\nu+1}) > 0: \textit{ Convergence}.
\end{align}
\vspace{-0.3cm}

Since \textit{every arbitrary} positive sequence of numbers $(P_\nu)$ must belong either to the type $(D_\nu)$ or to the type $(C_\nu)$, one finds by combination of (31) with the \textit{convergence} criterion (21) directly \textit{Kummer}'s convergence criterion (18), with the \textit{divergence} criterion (21) the \textit{disjunctive criterion of the second kind}:

\vspace{-0.5cm}
\begin{align}
    \lim (D_\nu \cdot \frac{a_\nu}{a_{\nu+1}} - D_{\nu+1}) \begin{cases} < 0: & \textit{Divergence}, \\ > 0: & \textit{Convergence}, \end{cases}
\end{align}
\vspace{-0.2cm}

into which one need only substitute from (22a), (26a):

\vspace{-0.5cm}
\begin{align}
    D_\nu = \frac{1}{M_{\nu+1} - M_\nu} \quad or \quad D_\nu = \frac{L_\chi(M_\nu)}{M_{\nu+1} - M_\nu} \quad (\chi = 0,1,2,\cdots)
\end{align}
\vspace{-0.3cm}

to obtain \textit{scales} of increasingly effective\textsuperscript{188)} criteria. For $M_\nu = \nu$ there results from this in succession the \textit{Cauchy fundamental criterion} (II), \textit{Raabe}'s\textsuperscript{189)}

\vfill
\leftline{\rule{2in}{0.4pt}}
\vspace{0.2cm}
{
\footnotesize
 occasion rightfully claims the fundamental idea and the methods used for himself.

187) Written differently: $\lim (P_\nu \cdot a_\nu)^{\frac{1}{s_\nu}} < 1$.

188) On the character (not so immediately visible here as with the criteria of the first kind) of the successive \textit{sharpenings} to be achieved, cf. my treatise l.c. p. 364.

189) Z. f. Phys. u. Math, von \textit{Baumgartner} u. \textit{Ettingshausen} 10 (1832), p. 63. Rediscovered by \textit{Duhamel}, J. de Math. 4 (1839), p. 214. Cf. also 6 (1841), p. 85. The criterion in question can be brought to the form: 

\vspace{-0.3cm}
$$\lim \nu \cdot (\frac{a_\nu}{a_{\nu+1}} - 1) \begin{cases} < 1: & \textit{Divergence}, \\ > 1: & \textit{Convergence}. \end{cases}$$
\vspace{-0.5cm}

}