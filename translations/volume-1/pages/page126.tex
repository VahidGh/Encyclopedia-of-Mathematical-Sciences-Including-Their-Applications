\thispagestyle{fancy}

\vspace{0.5cm}

and (apart from an un-essential difference in form) a criterion sequence likewise established by \textit{Bertrand}\textsuperscript{190)}.

Besides the \textit{main form} (32) of the \textit{disjunctive criterion of the second kind}, I have emphasized the following as particularly simple and of equal scope:

\vspace{-0.3cm}
\begin{align}
    \lim D_{\nu+1} \lg \frac{D_\nu a_\nu}{D_{\nu+1} a_{\nu+1}} \begin{cases} < 0: & \textit{Divergence} \\ > 0: & \textit{Convergence} \end{cases}
\end{align}

Here too it proves admissible to replace the $D_\nu$ in the \textit{convergence} criterion by the terms of a completely \textit{arbitrary} positive sequence of numbers $(P_\nu)$, so that a criterion of the same generality as \textit{Kummer}'s results.

\vspace{0.3cm}
\textbf{28. Other Criterion Forms.} The theory of \textit{criteria of the first and second kind} may be considered completely closed. If nevertheless from time to time "new" such criteria keep appearing, these involve either the rediscovery of long-known criteria or special formations of subordinate importance.

On the other hand, the unlimited possibility of further general criterion formations emerges if one compares instead of $a_\nu$ or $\frac{a_\nu}{a_{\nu+1}}$ some other, suitably chosen combinations $F(a_\nu, a_{\nu+1}, \ldots)$ with the corresponding ones of the $d_\nu$ or $c_\nu$. On this principle are based the criteria of the \textit{third kind} (difference criteria)\textsuperscript{191)} established by me, as well as the \textit{"extended criteria of the second kind"}, in which, instead of the quotients of two \textit{consecutive} terms, those of two \textit{arbitrarily distant} terms or also those of two \textit{groups of terms} are taken into consideration. I arrive by the latter path to the following \textit{extended main criterion of the second kind}:

\vspace{-0.3cm}
\begin{align}
\begin{cases} \lim_{x=\infty} \frac{(M_{x+h} - M_x) \cdot f(M_{x+h})}{(m_{x+h} - m_x) \cdot f(m_x)} > 1: & \textit{Divergence } \text{of the series} \sum f(\nu), \\ \lim_{x=\infty} \frac{(M_{x+h} - M_x) \cdot f(M_x)}{(m_{x+h} - m_x) \cdot f(m_{x+h})} < 1: & \textit{Convergence } \text{of the series} \sum f(\nu) \end{cases}
\end{align}

\vfill
\leftline{\rule{2in}{0.4pt}}
\vspace{0.2cm}
{
\footnotesize
190) J. de Math. 7, p. 43. Cf. also: \textit{Bonnet}, J. de Math. 8, p. 89 and \textit{Paucker}, J. f. Math. 42, p. 143. -- The \textit{Gauss} criteria can be derived with the help of \textit{Raabe}'s and the \textit{first Bertrand} criterion, as \textit{B}. has shown l.c. p. 52; incidentally also with the help of \textit{Kummer}'s criteria (\textit{Kummer} l.c. p. 178). The analogous holds for the somewhat more general case: $\frac{a_{\nu+1}}{a_\nu} = 1 + \frac{c_1}{\nu} + \frac{c_2}{\nu^2} + \cdots$. \textit{Schlömilch}, Z. f. Math. 10 (1865), p. 74.

191) L.c. p. 379.

}