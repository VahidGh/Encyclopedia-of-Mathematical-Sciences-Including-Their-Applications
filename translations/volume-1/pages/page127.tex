\thispagestyle{fancy}
\fancyhead[LO]{29. Scope of the Criteria of First and Second Kind}

\vspace{0.5cm}

when $M_x > m_x$ and $M_x$, $m_x$ are monotonically increasing, $f(x)$ a monotonically decreasing function of the positive variable $x$. From the same result for $h = 1$ the criteria derived by \textit{G. Kohn}\textsuperscript{192)}, for $\lim h = 0$ the criteria of \textit{Ermakoff}\textsuperscript{193)} distinguished by formal simplicity and great scope:

\vspace{-0.5cm}
\begin{align}
    \lim_{x=\infty} \frac{M'_x \cdot f(M_x)}{m'_x \cdot f(m_x)} \begin{cases} > 1: & \textit{Divergence}, \\ < 1: & \textit{Convergence}, \end{cases}
\end{align}
\vspace{-0.3cm}

The latter I have recently generalized in such a way that $f(x)$ no longer needs to be assumed \textit{monotonic}\textsuperscript{194)}.

\vspace{0.2cm}
\textbf{29. Scope of the Criteria of First and Second Kind.} The field of application of any criterion of the \textit{second kind} is naturally a noticeably \textit{narrower} one than that of the corresponding (i.e., formed with the same $D_\nu$, $C_\nu$) criterion of the \textit{first kind}\textsuperscript{195)}. \textit{Cauchy} has, on the basis of the limit theorem mentioned in No. 18, Eq. (9), more precisely established the connection between his fundamental criteria of the first and second kind. The relevant result can be generalized in the following way: If the disjunctive criterion of the \textit{second kind} (32) for $D_{\nu}^{-1} = M_{\nu+1} - M_\nu$ provides a \textit{decision} or \textit{fails} by the appearance of the \textit{limit value zero}, then the same holds for the criterion of the \textit{first kind} (29a). On the other hand, the latter \textit{can} still \textit{provide} a decision when the \textit{former} fails through the appearance of \textit{indeterminate limits}\textsuperscript{196)}.

The limits for the scope of the ordinary \textit{criterion pairs of the first kind} (20) result from the observation that they fail not only when directly:

\vspace{-0.4cm}
$$(A) \quad \lim D_\nu \cdot a_\nu = 0, \quad \lim C_\nu \cdot a_\nu = \infty,$$
\vspace{-0.4cm}

but also when those limit values \textit{do not exist} at all and \textit{simultaneously}:

\vspace{-0.4cm}
$$(B) \quad \underline{\lim} \, D_\nu \cdot a_\nu = 0, \quad \overline{\lim} \, C_\nu \cdot a_\nu = \infty.$$
\vspace{-0.4cm}

\vfill
\leftline{\rule{2in}{0.4pt}}
\vspace{0.2cm}
{
\footnotesize
192) Archiv f. Math. 67 (1882), p. 82, 84.

193) \textit{Darboux} Bulletin 2 (1871), p. 250; 18 (1883), p. 142. The criterion resulting for $M_x = e^x$, $m_x = 1$: $\lim \frac{e^x f(e^x)}{f(x)}  > or < 1$ possesses, for example, the same scope as the \textit{entire scale} of logarithmic criteria.

194) Chicago Papers p. 328. There also a shorter proof based on the theory of definite integrals (improvement of that originally given by \textit{W. Ermakoff}) and more precise establishment of the relationship between $\sum_{m=\nu}^{\infty} f(\nu)$ and $\int_{m}^{\infty} f(x)dx$.

195) Cf. l.c. p. 308.

196) \textit{Pringsheim} l.c. p. 376.

}