\thispagestyle{fancy}

\vspace{0.5cm}

If one chooses, as happens with the criteria exclusively applied in practice, the $D_\nu$, $C_\nu$ \textit{monotonically} increasing\textsuperscript{197)}, their applicability obviously extends only to such $a_\nu$ which either directly decrease \textit{monotonically} or at least \textit{"essentially" monotonically}, i.e., so that any fluctuations remain within certain bounds.

\vspace{0.3cm}
\textbf{30. The Boundary Regions of Divergence and Convergence.} The first example of a \textit{convergent} series for which the ordinary logarithmic (\textit{Bonnet}) scale \textit{fails} according to the mode of Eq. (A) was constructed by \textit{Du Bois-Reymond}\textsuperscript{198)}. I have then given a somewhat more general series type with a more transparent formation law, which simultaneously also provides \textit{divergent} series of the nature in question\textsuperscript{199)}. The method used here can, as \textit{Hadamard} has shown\textsuperscript{200)}, easily be transferred to any \textit{arbitrary} criterion scale.

But there are also infinitely many \textit{monotonic} $a_\nu$ for which an arbitrarily chosen criterion scale must completely fail in the sense of equations (B). The investigations I have undertaken in this direction\textsuperscript{201)} lead to the following general theorem: "However strongly $\sum C_\nu^{-1}$ may \textit{converge}, there always exist monotonic \textit{divergent} series $\sum a_\nu$ for which: $\underline{\lim} C_\nu a_\nu = 0$. However \textit{slowly} $m_\nu$ may grow to infinity with $\nu$, there always exist monotonic \textit{convergent} series $\sum a_\nu$ for which: $\overline{\lim} \, \nu \cdot m_\nu \cdot a_\nu = \infty$; on the other hand, one always has: $\lim \nu \cdot a_\nu = 0$." There exists therefore no $M_\nu$ of \textit{arbitrarily high infinity} such that $\lim M_\nu \cdot a_\nu > 0$ forms a \textit{necessary} condition for the \textit{divergence} of $\sum a_\nu$. On the other hand, the relation $\lim \nu \cdot a_\nu = 0$ does indeed form a \textit{necessary}\textsuperscript{202)}

\vfill
\leftline{\rule{2in}{0.4pt}}
\vspace{0.2cm}
{
\footnotesize
197) E.g., 
$$ D_\nu = \nu , \quad \nu \lg \nu , \quad \ldots $$ 
$$ C_\nu = \nu^{1+\varrho}, \quad \nu \cdot (\lg \nu)^{1+\varrho} , \quad \ldots$$ 

(\textit{Bonnet} criteria: see No. 27).

198) J. f. Math. 76 (1873), p. 88.

199) L.c. p. 353 ff.

200) Acta Math. 18 (1894), p. 325.

201) L.c. p. 347, 356. Math. Ann. 37 (1890), p. 600. Münch. Ber. 26 (1896), p. 609 ff.

202) That the same always \textit{suffices} for convergence was claimed by \textit{Th. Olivier} (J. f. Math. 2, 1827, p. 34), refuted by \textit{Abel} (l.c. 3, p. 79. Oeuvres 1, p. 399) by pointing to the series $\sum \frac{1}{\nu \lg \nu}$. \textit{Kummer} has, on the other hand, shown that that condition always suffices

}