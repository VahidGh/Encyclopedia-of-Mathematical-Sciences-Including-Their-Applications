\thispagestyle{fancy}
\fancyhead[LO]{31. Conditional and Unconditional Convergence}

\vspace{0.5cm}

condition for the \textit{convergence} of $\sum a_\nu$, but \textit{no}\textsuperscript{203)} relation of the form: $\lim \nu \cdot m_\nu \cdot a_\nu = 0$ with \textit{arbitrarily weak infinity} of $\lim m_\nu$. In other words: There exists, even when one restricts oneself to the consideration of \textit{monotonic}\textsuperscript{204)} $a_\nu$, \textit{no boundary} of \textit{divergence} at all, i.e., \textit{no} sequence of numbers $(c_\nu)$ such that from some definite $\nu$ onward \textit{constantly} $a_\nu > c_\nu$ would have to hold if $\sum a_\nu$ \textit{diverges}. And while every sequence of numbers of the form $(\frac{\varepsilon}{\nu})$, where $\varepsilon > 0$, forms a \textit{boundary} of \textit{convergence} (i.e., from some definite $\nu$ onward \textit{constantly} $a_\nu < \frac{\varepsilon}{\nu}$ must hold if $\sum a_\nu$ is to \textit{converge}), no sequence of numbers of the form $(\frac{\varepsilon_\nu}{\nu})$ does, however \textit{slowly} $\varepsilon_\nu$ with $\frac{1}{\nu}$ may approach \textit{zero}.

According to this, the fiction of a \textit{"boundary between convergence and divergence"} introduced by \textit{Du Bois-Reymond}\textsuperscript{205)} rests from the outset on a false fundamental conception. But even if one understands the same in an essentially narrower sense, namely as a presumptive boundary between any two \textit{definite} divergent and convergent scales, such as: $\frac{1}{L_{\chi}(\nu)}$ and $\frac{1}{L_{\chi}(\nu) \cdot (\lg_\chi^\nu)^\varrho} \quad (\chi = 1, 2, 3, \cdots ; \varrho < 0) $, it appears untenable, as I have attempted to demonstrate in detail\textsuperscript{206)}.

\vspace{0.5cm}
\textbf{31. Conditional and Unconditional Convergence.} A series with \textit{positive} and \textit{negative} terms $u_\nu$ is called \textit{absolutely} convergent if $\sum |u_\nu|$ \textit{converges}; that under this assumption it actually always \textit{converges itself} as well was, as already noted in No. 23, proved by \textit{Cauchy}. That there also exist \textit{convergent} series $\sum u_\nu$ for which $\sum |u_\nu|$ \textit{diverges} was already shown by the

\vfill
\leftline{\rule{2in}{0.4pt}}
\vspace{0.2cm}
{
\footnotesize
for convergence when $\frac{a_\nu}{a_{\nu+1}}$ can be expanded in ascending powers of $\frac{1}{\nu}$ (J. f. Math. 13, 1835, p. 178).

203) One frequently finds the \textit{false theorem} (cf. my Conv.-Theory l.c. p. 343) that \textit{generally}: $\lim D_\nu \cdot a_\nu = 0$ forms a \textit{necessary} condition for \textit{convergence}, while for $D_\nu \succ \nu$ in truth only: $\underline{\lim} D_\nu \cdot a_\nu = 0$ needs to hold. This uniquely correct formulation is already given by \textit{Abel} in the posthumous note cited above: Oeuvres 2, p. 198.

204) For \textit{non-monotonic} $a_\nu$ the existence of such convergence and divergence boundaries appears excluded \textit{a priori}; see my Conv.-Theory l.c. p. 344, 357.

205) Münch. Abh. 12 (1876), p. XV. Math. Ann. 11 (1877), p. 158 ff.

206) Münch. Ber. 27 (1897), p. 203 ff.

}