\thispagestyle{fancy}

\vspace{0.5cm}

example of the \textit{Leibniz} series\textsuperscript{207)}: $\frac{\pi}{4} = 1 - \frac{1}{3} + \frac{1}{5} - \frac{1}{7} + \cdots$. Also \textit{Leibniz} proved generally the \textit{convergence} of every series of the form $\sum (-1)^\nu \cdot a_\nu$ (where $a_\nu \geq a_{\nu+1} > 0$, $\lim a_\nu = 0$)\textsuperscript{208)}. To such series, specifically to $\sum (-1)^\nu \cdot \frac{1}{\nu + 1}$, \textit{Cauchy} attached the important observation\textsuperscript{209)} that \textit{their convergence depends essentially on the arrangement of the terms}, such that they become \textit{divergent} under certain rearrangements. With this he uncovered that property which one is accustomed to designate today as \textit{conditional} convergence. \textit{Lej.-Dirichlet} added\textsuperscript{210)} that under certain rearrangements the \textit{convergence} is indeed preserved, but the \textit{sum} undergoes a change; and he has particularly sharply emphasized that an \textit{absolutely} convergent series always converges \textit{unconditionally}, i.e., independently of the arrangement of the terms, to the same sum\textsuperscript{211)}. Through \textit{Cauchy} and \textit{Dirichlet} it had at most been proved that \textit{certain} non-absolutely converging series converge only \textit{conditionally}; that this must in truth be the case for \textit{every} non-absolutely converging series was first taught by a theorem proved by \textit{Riemann}\textsuperscript{212)}, according to which two \textit{arbitrary divergent} series of the form $\sum a_\nu$, $\sum (-b_\nu)$ ($a_\nu > 0$, $b_\nu > 0$, $\lim a_\nu = \lim b_\nu = 0$) can be combined into a \textit{convergent} series with an \textit{arbitrarily prescribed sum}\textsuperscript{213)}. With this the complete equivalence of \textit{absolute} and \textit{unconditional}, \textit{non-absolute}

\vfill
\leftline{\rule{2in}{0.4pt}}
\vspace{0.2cm}
{
\footnotesize
207) De vera proportione circuli ad quadratum circumscriptum. Acta erud. Lips. 1682. (Opera, Ed. Dutens 3, p. 140.) The series is already found in \textit{James Gregory}. Cf. \textit{Reiff} l.c. p. 45. \textit{M. Cantor} 3, p. 72.

208) Letter to \textit{Joh. Bernoulli}, Jan. 1, 1714. (Commerc. epist. 2, p. 329.)

209) Résumé anal. p. 57.

210) Berl. Abh. 1837, p. 48. (Ges. W. 1, p. 318.)

211) This was explicitly proved probably for the first time by \textit{W. Scheibner}: Über unendliche Reihen und deren Konvergenz. Gratulationsschrift, Lpzg. 1860, p. 11. The expression \textit{"unconditional"} convergence probably stems from \textit{Weierstrass} (J. f. Math. 51 [1856], p. 41). Individual German and almost all French and English authors designate the \textit{conditionally} convergent series as \textit{semiconvergent}. This expression is in itself little fitting (for the addition \textit{"semi"} designates not so much a special \textit{mode} as rather the partial \textit{negation} of convergence) and appears also for the reason little recommendable because it (or the synonymous \textit{half-convergent} series, série \textit{demi-convergente}) has already acquired a completely different meaning following the precedent of \textit{Legendre} (Exerc. de calc. intégr. 1, p. 267). Cf. No. 38.

212) Gött. Abh. 13 (1867). (Ges. W. p. 221.)

213) \textit{Dini} has noted that one can in analogous manner also produce proper or improper \textit{divergence}; Ann. di Mat. (2) 2 (1868), p. 31.

}