\thispagestyle{fancy}
\fancyhead[LO]{32. Value Changes of Conditionally Convergent Series}

\vspace{0.5cm}

and \textit{conditional} series convergence definitively established.

\vspace{0.5cm}
\textbf{32. Value Changes of Conditionally Convergent Series.} For the \textit{change} which the harmonic series 

$$\sum_{\nu=0}^{\infty} (-1)^{\nu} \cdot \frac{1}{\nu+1} = \lg 2$$ 
\vspace{0.3cm}

undergoes if one lets $q$ negative terms follow every $p$ positive terms, the value $\frac{1}{2} \lg \frac{p}{q}$ was found by \textit{Mart. Ohm} (with the help of integral calculus)\textsuperscript{214)}. A direct generalization of this result is formed by the theorem proved by \textit{Schlömilch}\textsuperscript{215)} that the series $\sum (-1)^{\nu} \cdot a_{\nu+1}$ undergoes the value change $(\lim \nu \cdot a_\nu) \cdot \frac{1}{2} \lg \frac{p}{q}$ under analogous rearrangement. I have investigated in completely general manner\textsuperscript{216)} what value changes a convergent series composed of the two divergent components $\sum a_\nu$, $\sum (-b_\nu)$ undergoes when the \textit{relative frequency} of the $a_\nu$ and $(-b_\nu)$ (while maintaining the original order within the two individual groups $(a_\nu)$ and $(b_\nu)$) is changed in an arbitrarily prescribed manner, and conversely, what such rearrangement is required to produce an arbitrarily prescribed value change. The investigation of \textit{"singular series remainders"} of the form: $\lim_{n=\infty} \sum_{\nu=n+1}^{n+\varphi(n)} a_\nu$ required for this and completely feasible for the case $\lim \frac{a_{\nu+1}}{a_\nu} = 1$ teaches that the value changes in question depend not on the special formation law of the $a_\nu$, but solely on their behavior for $\lim \nu = \infty$: If $\lim \sum_{\nu=n+1}^{n+\varphi(n)} a_\nu = a$ (finite), then also $\lim \sum_{\nu=n+1}^{n+\varphi(n)} a'_\nu = a$, if $a'_\nu \cong a_\nu$; on the other hand $\lim \sum_{\nu=n+1}^{n+\varphi(n)} a'_\nu = 0$ or $= \infty$, if $a'_\nu \prec a_\nu$ or $\succ a_\nu$ respectively. The $\varphi(n)$ required for producing a certain remainder value (incl. 0 and $\infty$) (i.e., ultimately the \textit{rearrangement law} leading to a certain \textit{value change})

\vfill
\leftline{\rule{2in}{0.4pt}}
\vspace{0.2cm}
{
\footnotesize
214) De nonnullis seriebus summandis. Antr.-Programm, Berlin 1839. An elementary derivation in \textit{H. Simon}, Die harm. Reihe. Dissert. Halle 1886.

215) Z. f. Math. 18 (1873), p. 520.

216) Math. Ann. 22 (1883), p. 455 ff.

}