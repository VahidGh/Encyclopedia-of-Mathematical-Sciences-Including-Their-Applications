\thispagestyle{fancy}

\vspace{0.5cm}

depends then in precisely specifiable manner on the infinitary nature of the $a_\nu$. If $a_\nu \succ \frac{1}{\nu}$, the series sum undergoes the change $0$, $a$, $\infty$, according as $\lim \varphi(n) \cdot a_n = 0$, $a$, $\infty$. The analogous holds in the case: $a_\nu \cong \frac{g}{\nu}$, with the single difference that the change, if $\lim \varphi(n) \cdot a_n = a$, here takes the value: $\frac{1}{g} \lg(1 + ag)$. If finally $a_\nu \prec \frac{1}{\nu}$, the \textit{two} assumptions $\lim \varphi(n) \cdot a_n = 0$ and $= a$ yield \textit{no} value change; in the case: $\lim \varphi(n) \cdot a_n = \infty$ there then results a definite finite or infinitely large change, according to the particular manner of the becoming infinite of $\lim \varphi(n) \cdot a_n$.\textsuperscript{217)}

A somewhat more general type of rearrangements which leave the sum of a conditionally converging series \textit{unchanged} has been considered by \textit{E. Borel}\textsuperscript{218)}.

\vspace{0.3cm}
\textbf{33. Criteria for Possibly Only Conditional Convergence.} For establishing the \textit{simple}, i.e., possibly only \textit{conditional} convergence of a series with positive and negative terms, one possesses no general criteria. The \textit{measure of term decrease} is here completely irrelevant for judging convergence, as the \textit{Leibniz} criterion for alternating series (No. 31) shows: $\sum (-1)^\nu \cdot a_\nu$ converges even when the $a_\nu$ approach zero monotonically \textit{arbitrarily slowly}. A useful aid in many cases is given by the transformation originating from \textit{Abel} ("partial summation")\textsuperscript{219)}:

\vspace{-0.5cm}
\begin{align}
    \sum_{\nu=0}^{n} u_\nu v_\nu = u_n V_n - \sum_{\nu=0}^{n-1} (u_{\nu} - u_{\nu+1}) \cdot V_\nu + u_n V_n
\end{align}

\vspace{-0.3cm}
$$(where: V_\nu = v_0 + v_1 + \cdots + v_\nu),$$ 

which for $\lim n = \infty$ provides the following convergence theorem: "If $\sum (u_\nu - u_{\nu+1})$ is absolute and $\sum v_\nu$ is convergent at all, then $\sum u_\nu v_\nu$ converges at least in the prescribed arrangement. This also holds when $\sum v_\nu$ oscillates within finite bounds, provided that $\lim u_\nu = 0$." The application of \textit{Abel}'s transformation for such convergence considerations originates from \textit{Dirichlet}\textsuperscript{220)}, the above theorem in somewhat more special formulation from \textit{Dedekind}\textsuperscript{221)};

\vfill
\leftline{\rule{2in}{0.4pt}}
\vspace{0.2cm}
{
\footnotesize
217) Details l.c. p. 496 ff.

218) Bull. d. Sc. (2) 14 (1890), p. 97.

219) J. f. Math. 1 (1826), p. 314. Oeuvres 1, p. 222.

220) Vorl. über Zahlentheorie, herausgeg. von \textit{R. Dedekind}, 3. Aufl. (1879), § 101.

221) Ibid., Supplem. 9, § 143.

}