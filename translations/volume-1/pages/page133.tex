\thispagestyle{fancy}
\fancyhead[LO]{33. Criteria for Possibly Only Conditional Convergence}

\vspace{0.5cm}

The here given is found along with some simple modifications in \textit{Du Bois-Reymond}\textsuperscript{222)}.

From this theorem follows, for example, immediately the convergence of $\sum a_\nu \cdot \cos \nu x$ (excl. $x = 0 \pm 2k\pi$) and $\sum a_\nu \cdot \sin \nu x$, first proved otherwise by \textit{Malmsten}\textsuperscript{223)}, when the $a_\nu$ approach zero \textit{monotonically}\textsuperscript{224)}, as well as that of some other trigonometric series\textsuperscript{225)}. Also the convergence proof for the \textit{Fourier} series can be reduced to it under certain simplifying assumptions\textsuperscript{226)}.

The \textit{Abel} transformation in connection with the convergence of the series $\sum \frac{M_{\nu+1} - M_\nu}{M_{\nu+1} \cdot M_\nu^\varrho}$ emphasized in No. 26 has been used by me\textsuperscript{227)} to obtain a very general criterion for judging the so-called \textit{Dirichlet} series: $\sum k_\nu \cdot M_\nu^{-\varrho}$ ($k_\nu$ arbitrary, $\varrho > 0$). Special cases of the same have been found earlier by \textit{Dedekind}\textsuperscript{228)} and \textit{O. Hölder}\textsuperscript{229)} by other methods.

A useful generalization of the ordinary convergence theorem for alternating series results from the \textit{Weierstrass} convergence investigations\textsuperscript{230)}. According to this, $\sum (-1)^\nu \cdot a_\nu$ still converges \textit{conditionally} when $\frac{a_\nu}{a_\nu+1} = 1 + \frac{\chi}{\nu} + \frac{\lambda}{\nu^2} + \cdots$ and $0 < \chi \leq 1$.\textsuperscript{231)}

The most important category of series which generally need converge only \textit{conditionally} are the \textit{Fourier series}\textsuperscript{232)}. The general investigations concerning their convergence and divergence are based on the representation of $s_n$ by a

\vfill
\leftline{\rule{2in}{0.4pt}}
\vspace{0.2cm}
{
\footnotesize
222) Antr.-Programm, p. 10.

223) With the unnecessary restriction $\lim \frac{a_{\nu+1}}{a_\nu} = 1$. Nova acta Upsal. 12 (1844), p. 255. Without that restriction and simpler: \textit{Hj. Holmgren}, J. de Math. 16 (1851), p. 186.

224) \textit{G. Björling} (J. de Math. 17 (1852), p. 470) erroneously considers the conditions: $a_\nu > 0$, $\lim a_\nu = 0$ already sufficient.

225) \textit{Du Bois-Reymond} l.c. p. 12, 17.

226) Ibid. p. 13.

227) Math. Ann. 37 (1886), p. 41.

228) Vorl. über Zahlentheorie, Suppl. 9, § 144.

229) Math. Ann. 20 (1882), p. 545.

230) J. f. Math. 51 (1856), p. 29; Werke 1, p. 185.

231) This also holds for \textit{complex} $a_\nu$, if the real part of $\chi$ satisfies the condition given in the text. For $\chi > 1$, $\sum a_\nu$ \textit{converges absolutely} (most simply by \textit{Raabe}'s criterion), for $\chi \leq 0$ it \textit{diverges}. Cf. also \textit{Stolz}, Allg. Arithm. 1, p. 268.

232) Cf. II A8.

}