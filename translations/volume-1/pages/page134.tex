\thispagestyle{fancy}

\vspace{0.5cm}

definite integral and its discussion for $\lim n = \infty$.

\vspace{0.3cm}
\textbf{34. Addition and Multiplication of Infinite Series.} For the \textit{addition} or \textit{subtraction} of two convergent series there results immediately from the relation: $\lim a_\nu \pm \lim b_\nu = \lim (a_\nu \pm b_\nu)$ (No. 17, Eq. (15)) the rule:

\vspace{-0.5cm}
\begin{align}
    \sum_{\nu=0}^{\infty} u_\nu \pm \sum_{\nu=0}^{\infty} v_\nu = \sum_{\nu=0}^{\infty} (u_\nu \pm v_\nu).
\end{align}

For \textit{multiplication} \textit{Cauchy} established the theorem:

\vspace{-0.5cm}
\begin{align}
(\sum_{\nu=0}^{\infty} u_\nu) \cdot (\sum_{\nu=0}^{\infty} v_\nu) = \sum_{\nu=0}^{\infty} w_\nu \quad (w_\nu = u_0 v_\nu + u_1 v_{\nu - 1} + \cdots + u_\nu v_0),
\end{align}

under the assumption that $\sum u_\nu$, $\sum v_\nu$ \textit{converge absolutely}\textsuperscript{233)}, with the explicit indication that the formula can \textit{fail} for \textit{non-absolutely} converging series\textsuperscript{234)}. \textit{Abel} has shown that the same is \textit{valid} as soon as (besides the series $\sum u_\nu$, $\sum v_\nu$ naturally assumed \textit{convergent}) the series $\sum w_\nu$ \textit{converges} at all\textsuperscript{235)}. Since this convergence proof (apart from the case of \textit{absolute} convergence of $\sum u_\nu$, $\sum v_\nu$ settled by \textit{Cauchy}) must be provided specially each time, it appears by no means superfluous that \textit{F. Mertens} has extended the validity of the multiplication theorem (39) to the case that only \textit{one} of the two series $\sum u_\nu$, $\sum v_\nu$ converges \textit{absolutely}\textsuperscript{236)}. The case that \textit{both} series converge only \textit{conditionally} has been considered by me in detail\textsuperscript{237)}. If one of the two series, say $\sum u_\nu$, possesses the property that $\sum |u_\nu + u_{\nu+1}|$ converges, then the condition $\lim w_\nu = 0$ appears as \textit{necessary and sufficient} for the validity of formula (39); from this result in particular simple criteria for the case

\vfill
\leftline{\rule{2in}{0.4pt}}
\vspace{0.2cm}
{
\footnotesize
233) Anal. algébr. p. 147.

234) Ibid. p. 149, probably the first place where the \textit{different} behavior of \textit{absolutely} and \textit{non-absolutely} converging series is emphasized.

235) J. f. Math. 1 (1826), p. 318. (Oeuvres 1, p. 226.) \textit{Abel}'s proof is based on the consideration of the series $\sum u_\nu x^\nu$, $\sum v_\nu x^\nu$ for $\lim x = 1$, thus on a \textit{continuous} limit transition. A proof \textit{without} use of this aid belonging to function theory has been given by \textit{E. Cesàro}: Bull. d. Sc. (2) 14 (1890), p. 114. Similarly \textit{Jordan}, Cours d'Anal. 1, p. 282.

236) J. f. Math. 79 (1875), p. 182. Other proof by \textit{W. V. Jensen}, Nouv. Corresp. math. 1879, p. 430.

237) Math. Ann. 21 (1883), p. 327.

}