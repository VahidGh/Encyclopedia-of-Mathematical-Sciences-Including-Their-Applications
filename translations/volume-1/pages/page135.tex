\thispagestyle{fancy}
\fancyhead[LO]{35. Double Series}

\vspace{0.5cm}

of two \textit{alternating} series with \textit{monotonic} terms\textsuperscript{238)}. Under the more general assumption that $\sum u_\nu$ becomes \textit{absolutely} convergent when one groups the $u_\nu$ in groups of $p_\nu$ terms ($p_\nu$ constant or variable, but remaining finite), I have given a \textit{sufficient} condition which includes the \textit{Cesàro} and \textit{Mertens} theorem as a special case. For the case $p_\nu = 2$, \textit{A. Voss}\textsuperscript{239)} then, for arbitrary \textit{constant}\textsuperscript{240)} and \textit{finitely-variable}\textsuperscript{241)} $p_\nu$, \textit{F. Cajori} established the \textit{necessary and sufficient} conditions.

\vspace{0.3cm}
\textbf{35. Double Series.} The addition formula (38) is indeed directly transferable to an arbitrary \textit{finite} number of series:

\vspace{-0.1cm}
$$\sum_{\nu=0}^{\infty} u_{\nu}^{(\mu)} \quad (\mu = 0,1,\ldots m),$$

but not to the case $m = \infty$, i.e., for the validity of the relation:

\vspace{-0.5cm}
\begin{align}
\sum_{\mu=0}^{\infty} (\sum_{\nu=0}^{\infty} u_{\nu}^{(\mu)}) = \sum_{\nu=0}^{\infty} (\sum_{\mu=0}^{\infty} u_{\nu}^{(\mu)})
\end{align}

it appears by no means \textit{sufficient} that the \textit{left} side has a definite meaning, thus \textit{converges} absolutely. \textit{Cauchy} has shown that Eq. (40) holds when also $\sum_{\mu=0}^{\infty} (\sum_{\nu=0}^{\infty} |u_{\nu}^{(\mu)}|)$ converges\textsuperscript{242)}; the multiplication theorem (39) for two absolutely convergent series proves to be a special case of this theorem\textsuperscript{243)}. At the same time \textit{Cauchy} connected to the consideration of a \textit{doubly-infinite scheme} of terms $u_{\nu}^{(\mu)}$ (where perhaps the index $\mu$ may characterize the \textit{rows}, the index $\nu$ the \textit{columns}) the concept of the \textit{double series}. Setting $\sum_{\mu=0}^{m} \sum_{\nu=0}^{n} u_{\nu}^{(\mu)} = s_n^{(m)}$, the \textit{double series} $\sum_{\mu,\nu=0}^{\infty} u_{\nu}^{(\mu)}$ formed from the terms $u_{\nu}^{(\mu)}$ is called \textit{convergent} and $s$ its \textit{sum}, when in the

\vfill
\leftline{\rule{2in}{0.4pt}}
\vspace{0.2cm}
{
\footnotesize
238) An application to the multiplication of two \textit{trigonometric series} in Math. Ann. 26 (1886), p. 157.

239) Math. Ann. 24 (1884), p. 42.

240) Am. J. of Math. 15 (1893), p. 339.

241) N. Y. Bull. (2), 1 (1895), p. 180. \textit{Cajori} gives a brief analysis of the results found by me and \textit{Voss}: N. Y. Bull. 1 (1892), p. 184.

242) Anal. algébr. p. 541. A more general form of a sufficient condition valid for \textit{power series}, originating from \textit{Weierstrass} (Werke 2, p. 205), is essentially \textit{function-theoretic} in nature. Cf. II B 1.

243) \textit{Cauchy} l.c. p. 542.

}
